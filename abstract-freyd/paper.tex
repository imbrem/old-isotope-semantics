%% Commands for TeXCount
%TC:macro \cite [option:text,text]
%TC:macro \citep [option:text,text]
%TC:macro \citet [option:text,text]
%TC:envir table 0 1
%TC:envir table* 0 1
%TC:envir tabular [ignore] word
%TC:envir displaymath 0 word
%TC:envir math 0 word
%TC:envir comment 0 0

\documentclass[acmsmall,screen,review]{acmart}

\usepackage{syntax}
\renewcommand{\syntleft}{\normalfont\itshape}
\renewcommand{\syntright}{\normalfont\itshape}

\usepackage{prftree}

\newcounter{todos}
\newcommand{\TODO}[1]{{
  \stepcounter{todos}
  \begin{center}\large{\textcolor{red}{\textbf{TODO \arabic{todos}:} #1}}\end{center}
}}
\newcommand{\sorry}{\textcolor{red}{\textbf{sorry}}}

% Math fonts
\newcommand{\mc}[1]{\ensuremath{\mathcal{#1}}}
\newcommand{\mb}[1]{\ensuremath{\mathbf{#1}}}
\newcommand{\ms}[1]{\ensuremath{\mathsf{#1}}}

% Syntax atoms
\newcommand{\lbl}[1]{{`#1}}
\newcommand{\lto}{\Rightarrow}
\newcommand{\ctt}{\ms{tt}}
\newcommand{\cff}{\ms{ff}}

% Syntax
\newcommand{\letexpr}[3]{\ensuremath{\ms{let}\;#1 = #2\;\ms{in}\;#3}}
\newcommand{\letstmt}[3]{\ensuremath{\ms{let}\;#1 = #2; #3}}
\newcommand{\brb}[2]{\ms{br}\;#1\;#2}
\newcommand{\lbrb}[2]{\brb{\lbl{#1}}{#2}}
\newcommand{\ite}[3]{\ms{if}\;#1\;\{#2\}\;\ms{else}\;\{#3\}}
\newcommand{\ewhere}[2]{\ms{then}\;#1\;\ms{where}\;#2}
\newcommand{\where}[2]{#1\;\ms{where}\;#2}
\newcommand{\wbranch}[3]{#1(#2) \lto #3}
\newcommand{\lwbranch}[3]{\wbranch{\lbl{#1}}{#2}{#3}}
\newcommand{\bsplice}[3]{#1(#2)\;\{#3\}}
\newcommand{\lbsplice}[3]{\bsplice{\lbl{#1}}{#2}{#3}}
\newcommand{\csplits}[3]{#1 \mapsto #2;#3}
\newcommand{\cwk}[2]{#1 \mapsto #2}
\newcommand{\lwk}[2]{#1 \rightsquigarrow #2}
\newcommand{\tlin}[2]{\ms{lin}_{#2}(#1)}
\newcommand{\ltlin}[3]{\ms{lin}_{#3}(#1^{#2})}
\newcommand{\thyp}[3]{#1: {#2}^{#3}}
\newcommand{\lhyp}[3]{#1[#2](#3)}
\newcommand{\llhyp}[3]{\lhyp{\lbl{#1}}{#2}{#3}}
\newcommand{\rle}[1]{{\scriptsize\textsf{#1}}}
\newcommand{\taff}{\ms{a}}
\newcommand{\trel}{\ms{r}}
\newcommand{\tint}{\infty}
\newcommand{\hasty}[5]{#1 \vdash_{#2} #3: {#4}^{#5}}
\newcommand{\haslb}[3]{#1 \vdash #2 \rhd #3}
\newcommand{\lhaslb}[3]{#1 \vdash #2 \rhd #3}
\newcommand{\issubst}[3]{#1: #2 \mapsto #3}
\newcommand{\lbsubst}[3]{#1: #2 \rightsquigarrow #3}
\newcommand{\exprletsubst}[2]{{#1};{#2}}
\newcommand{\stmtletsubst}[2]{{#1};{#2}}
\newcommand{\mhole}[1]{{#1}^?}
\newcommand{\lhole}[1]{?#1}
\newcommand{\mhasty}[6]{#1;#2 \vdash_{#3} #4: {#5}^{#6}}
\newcommand{\mhaslb}[4]{#1;#2 \vdash #3 \rhd #4}
\newcommand{\mlhaslb}[4]{#1;#2 \vdash #3 \rhd #4}
\newcommand{\tyhole}[5]{#1: #2 \mapsto_{#3} {#4}^{#5}}
\newcommand{\blkhole}[3]{#1: #2 \mapsto #3}
\newcommand{\cfghole}[3]{#1: #2 \mapsto #3}
\newcommand{\substctx}[2]{{#1}^{#2}}
\newcommand{\substlbs}[2]{{#1}^{#2}}
\newcommand{\restrictsubst}[2]{{#1}_{#2}}
\newcommand{\subsubst}[2]{#1 \subseteq #2}
\newcommand{\isrw}[3]{#1: #2 \mapsto #3}
% \newcommand{\strictlbsubst}[3]{#1: #2 \rightsquigarrow_= #3}

% Denotational semantics
\newcommand{\dnt}[1]{\llbracket{#1}\rrbracket}
\newcommand{\ednt}[1]{\left\llbracket{#1}\right\rrbracket}
\newcommand{\upg}[2]{{#1}^{\uparrow #2}}

% Branding
\newcommand{\isotopessa}{\ms{isotope_{SSA}}}

%% Rights management information.  This information is sent to you
%% when you complete the rights form.  These commands have SAMPLE
%% values in them; it is your responsibility as an author to replace
%% the commands and values with those provided to you when you
%% complete the rights form.
\setcopyright{acmcopyright}
\copyrightyear{2018}
\acmYear{2018}
\acmDOI{XXXXXXX.XXXXXXX}

%%
%% These commands are for a JOURNAL article.
% \acmJournal{JACM}
% \acmVolume{37}
% \acmNumber{4}
% \acmArticle{111}
% \acmMonth{8}

%%
%% Submission ID.
%% Use this when submitting an article to a sponsored event. You'll
%% receive a unique submission ID from the organizers
%% of the event, and this ID should be used as the parameter to this command.
%%\acmSubmissionID{123-A56-BU3}

%%
%% The majority of ACM publications use numbered citations and
%% references.  The command \citestyle{authoryear} switches to the
%% "author year" style.
%%
%% If you are preparing content for an event
%% sponsored by ACM SIGGRAPH, you must use the "author year" style of
%% citations and references.
%% Uncommenting
%% the next command will enable that style.
%%\citestyle{acmauthoryear}

\begin{document}

\title{Freyd Categories are SSA}

\author{Neel Krishnaswami}
\email{nk480@cl.cam.ac.uk}
\orcid{0000-0003-2838-5865}

\author{Jad Ghalayini}
\email{jeg74@cl.cam.ac.uk}
\orcid{0000-0002-6905-1303}

%%
%% The code below is generated by the tool at http://dl.acm.org/ccs.cfm.
%% Please copy and paste the code instead of the example below.
%%
\begin{CCSXML}
<ccs2012>
 <concept>
  <concept_id>00000000.0000000.0000000</concept_id>
  <concept_desc>Do Not Use This Code, Generate the Correct Terms for Your Paper</concept_desc>
  <concept_significance>500</concept_significance>
 </concept>
 <concept>
  <concept_id>00000000.00000000.00000000</concept_id>
  <concept_desc>Do Not Use This Code, Generate the Correct Terms for Your Paper</concept_desc>
  <concept_significance>300</concept_significance>
 </concept>
 <concept>
  <concept_id>00000000.00000000.00000000</concept_id>
  <concept_desc>Do Not Use This Code, Generate the Correct Terms for Your Paper</concept_desc>
  <concept_significance>100</concept_significance>
 </concept>
 <concept>
  <concept_id>00000000.00000000.00000000</concept_id>
  <concept_desc>Do Not Use This Code, Generate the Correct Terms for Your Paper</concept_desc>
  <concept_significance>100</concept_significance>
 </concept>
</ccs2012>
\end{CCSXML}

\ccsdesc[500]{Do Not Use This Code~Generate the Correct Terms for Your Paper}
\ccsdesc[300]{Do Not Use This Code~Generate the Correct Terms for Your Paper}
\ccsdesc{Do Not Use This Code~Generate the Correct Terms for Your Paper}
\ccsdesc[100]{Do Not Use This Code~Generate the Correct Terms for Your Paper}

%%
%% Keywords. The author(s) should pick words that accurately describe
%% the work being presented. Separate the keywords with commas.
\keywords{TODO PUT KEYWORDS HERE}

% \received{20 February 2007}
% \received[revised]{12 March 2009}
% \received[accepted]{5 June 2009}

\maketitle

\TODO{this}

\section{Related Work}

\TODO{\cite{promonad}}

\TODO{\cite{linear-state-usage}}

\TODO{\cite{ssa-is-fun}}

\TODO{\cite{sparky}}

\bibliographystyle{ACM-Reference-Format}
\bibliography{references}

\end{document}
\endinput
