%% Commands for TeXCount
%TC:macro \cite [option:text,text]
%TC:macro \citep [option:text,text]
%TC:macro \citet [option:text,text]
%TC:envir table 0 1
%TC:envir table* 0 1
%TC:envir tabular [ignore] word
%TC:envir displaymath 0 word
%TC:envir math 0 word
%TC:envir comment 0 0

\documentclass[acmsmall,screen,review]{acmart}

\usepackage{syntax}
\renewcommand{\syntleft}{\normalfont\itshape}
\renewcommand{\syntright}{\normalfont\itshape}

\usepackage{prftree}

\usepackage{listings}
\usepackage{xcolor}
\usepackage{subcaption}
\usepackage{fancyvrb}

\definecolor{codegreen}{rgb}{0,0.6,0}
\definecolor{codegray}{rgb}{0.5,0.5,0.5}
\definecolor{codepurple}{rgb}{0.58,0,0.82}
\definecolor{backcolour}{rgb}{0.95,0.95,0.92}

\lstdefinestyle{mystyle}{
%    backgroundcolor=\color{backcolour},   
    commentstyle=\color{codegreen},
    keywordstyle=\color{magenta},
    numberstyle=\tiny\color{codegray},
    stringstyle=\color{codepurple},
    basicstyle=\ttfamily\footnotesize,
    breakatwhitespace=false,         
    breaklines=true,                 
    captionpos=b,                    
    keepspaces=true,                 
    numbers=left,                    
    numbersep=5pt,                  
    showspaces=false,                
    showstringspaces=false,
    showtabs=false,                  
    tabsize=2
}

\lstset{style=mystyle}

\newcounter{todos}
\newcommand{\TODO}[1]{{
  \stepcounter{todos}
  \begin{center}\large{\textcolor{red}{\textbf{TODO \arabic{todos}:} #1}}\end{center}
}}
\newcommand{\sorry}{\textcolor{red}{\textbf{sorry}}}

\newcommand{\todo}[1]{\stepcounter{todos} \textcolor{red}{TODO \arabic{todos}:} #1}

% Math fonts
\newcommand{\mc}[1]{\ensuremath{\mathcal{#1}}}
\newcommand{\mb}[1]{\ensuremath{\mathbf{#1}}}
\newcommand{\ms}[1]{\ensuremath{\mathsf{#1}}}

% Math
\newcommand{\nats}{\mathbb{N}}

% Syntax atoms
\newcommand{\lbl}[1]{{`#1}}
\newcommand{\lto}{\Rightarrow}
\newcommand{\ctt}{\ms{tt}}
\newcommand{\cff}{\ms{ff}}

% Syntax
\newcommand{\letexpr}[3]{\ensuremath{\ms{let}\;#1 = #2\;\ms{in}\;#3}}
\newcommand{\letstmt}[3]{\ensuremath{\ms{let}\;#1 = #2; #3}}
\newcommand{\brb}[2]{\ms{br}\;#1\;#2}
\newcommand{\lbrb}[2]{\brb{\lbl{#1}}{#2}}
\newcommand{\ite}[3]{\ms{if}\;#1\;\{#2\}\;\ms{else}\;\{#3\}}
\newcommand{\ewhere}[2]{\ms{then}\;#1\;\ms{where}\;#2}
\newcommand{\where}[2]{#1\;\ms{where}\;#2}
\newcommand{\wbranch}[3]{#1(#2) \lto #3}
\newcommand{\lwbranch}[3]{\wbranch{\lbl{#1}}{#2}{#3}}
\newcommand{\bsplice}[3]{#1(#2)\;\{#3\}}
%\newcommand{\lbsplice}[3]{\bsplice{\lbl{#1}}{#2}{#3}}
\newcommand{\csplits}[3]{#1 \mapsto #2;#3}
\newcommand{\cwk}[2]{#1 \mapsto #2}
\newcommand{\lwk}[2]{#1 \rightsquigarrow #2}
\newcommand{\tlin}[2]{#2 \subseteq \ms{lin}(#1)}
\newcommand{\ltlin}[3]{#3 \subseteq \ms{lin}(#1) \cap #2}
\newcommand{\thyp}[3]{#1: {#2}^{#3}}
\newcommand{\lhyp}[3]{#1[#2](#3)}
\newcommand{\llhyp}[3]{\lhyp{\lbl{#1}}{#2}{#3}}
\newcommand{\rle}[1]{{\scriptsize\textsf{#1}}}
\newcommand{\taff}{{\{\ms{a}\}}}
\newcommand{\trel}{{\{\ms{r}\}}}
\newcommand{\tint}{{\{\ms{a}, \ms{r}\}}}
\newcommand{\hasty}[5]{#1 \vdash_{#2} #3: {#4}^{#5}}
\newcommand{\haslb}[3]{#1 \vdash #2 \rhd #3}
\newcommand{\lhaslb}[3]{#1 \vdash #2 \rhd #3}
\newcommand{\issubst}[3]{#1: #2 \mapsto #3}
\newcommand{\lbsubst}[3]{#1: #2 \rightsquigarrow #3}
\newcommand{\exprletsubst}[2]{{#1};{#2}}
\newcommand{\stmtletsubst}[2]{{#1};{#2}}
\newcommand{\mhole}[1]{{#1}^?}
\newcommand{\lhole}[1]{?#1}
\newcommand{\mhasty}[6]{#1;#2 \vdash_{#3} #4: {#5}^{#6}}
\newcommand{\mhaslb}[4]{#1;#2 \vdash #3 \rhd #4}
\newcommand{\mlhaslb}[4]{#1;#2 \vdash #3 \rhd #4}
\newcommand{\tyhole}[5]{#1: #2 \mapsto_{#3} {#4}^{#5}}
\newcommand{\blkhole}[3]{#1: #2 \mapsto #3}
\newcommand{\cfghole}[3]{#1: #2 \mapsto #3}
\newcommand{\substctx}[2]{{#1}^{#2}}
\newcommand{\substlbs}[2]{{#1}^{#2}}
\newcommand{\restrictsubst}[2]{{#1}_{#2}}
\newcommand{\subsubst}[2]{#1 \subseteq #2}
\newcommand{\isrw}[3]{#1: #2 \mapsto #3}
\newcommand{\mbind}{\mathbin{{>}\hspace{-0.1em}{>}\hspace{-0.1em}{=}}}
% \newcommand{\strictlbsubst}[3]{#1: #2 \rightsquigarrow_= #3}

% Denotational semantics
\newcommand{\dnt}[1]{\llbracket{#1}\rrbracket}
\newcommand{\ednt}[1]{\left\llbracket{#1}\right\rrbracket}
\newcommand{\upg}[2]{{#1}^{\uparrow #2}}

% Weak memory
\newcommand{\bufloc}[1]{\overline{#1}}

% Branding
\newcommand{\isotopessa}{\ms{isotope_{SSA}}}

%% Rights management information.  This information is sent to you
%% when you complete the rights form.  These commands have SAMPLE
%% values in them; it is your responsibility as an author to replace
%% the commands and values with those provided to you when you
%% complete the rights form.
\setcopyright{acmcopyright}
\copyrightyear{2023}
\acmYear{2023}
\acmDOI{XXXXXXX.XXXXXXX}

%%
%% These commands are for a JOURNAL article.
% \acmJournal{JACM}
% \acmVolume{37}
% \acmNumber{4}
% \acmArticle{111}
% \acmMonth{8}

%%
%% Submission ID.
%% Use this when submitting an article to a sponsored event. You'll
%% receive a unique submission ID from the organizers
%% of the event, and this ID should be used as the parameter to this command.
%%\acmSubmissionID{123-A56-BU3}

\begin{document}

\title{Effectful Semantics for Substructural SSA}

\author{Neel Krishnaswami}
\email{nk480@cl.cam.ac.uk}
\orcid{0000-0003-2838-5865}

\author{Jad Ghalayini}
\email{jeg74@cl.cam.ac.uk}
\orcid{0000-0002-6905-1303}

\begin{abstract}
  Static single-assignment form (SSA) is one of the most widely used
  compiler intermediate representations. In this paper, we build on
  the observation that SSA can be seen as a language of basic blocks
  with arguments (or first-order procedures tail-calling one another)
  to give a type theory for SSA programs. Notably, the type system and
  equational theory of our language is designed to support
  substituting impure terms for variables, which makes it easy to
  validate compiler optimisations such as hoisting assignments out of
  loops.

  We give a categorical axiomatisation of our type theory and show
  that our equational theory is sound with respect to it. Our
  categorical semantics generalises the notions of premonoidal
  category and Freyd category to support substructural features,
  including both support for substructural types, as well as
  F\"{u}hrmann's notion of central, copyable and discardable
  morphisms.

  Finally, we exhibit a number of concrete models of our categorical
  axiomatisation, including in particular a model of TSO-style weak
  memory based on the semantics of \citet{sparky}. This demonstrates
  that our approach is strong enough to connect machine models of
  concurrency to compiler IRs supporting a rich collection of sound
  rewrites.
\end{abstract}

\begin{CCSXML}
  <ccs2012>
  <concept>
  <concept_id>10003752.10010124.10010131.10010133</concept_id>
  <concept_desc>Theory of computation~Denotational semantics</concept_desc>
  <concept_significance>500</concept_significance>
  </concept>
  <concept>
  <concept_id>10003752.10010124.10010131.10010137</concept_id>
  <concept_desc>Theory of computation~Categorical semantics</concept_desc>
  <concept_significance>500</concept_significance>
  </concept>
  <concept>
  <concept_id>10003752.10003790.10011740</concept_id>
  <concept_desc>Theory of computation~Type theory</concept_desc>
  <concept_significance>500</concept_significance>
  </concept>
  <concept>
  <concept_id>10003752.10003790.10011742</concept_id>
  <concept_desc>Theory of computation~Separation logic</concept_desc>
  <concept_significance>300</concept_significance>
  </concept>
  </ccs2012>
\end{CCSXML}

\ccsdesc[500]{Theory of computation~Denotational semantics}
\ccsdesc[500]{Theory of computation~Categorical semantics}
\ccsdesc[500]{Theory of computation~Type theory}
\ccsdesc[300]{Theory of computation~Separation logic}

%%
%% Keywords. The author(s) should pick words that accurately describe
%% the work being presented. Separate the keywords with commas.
\keywords{SSA, Categorical Semantics, Elgot Structure, Effectful Category}

% \received{20 February 2007}
% \received[revised]{12 March 2009}
% \received[accepted]{5 June 2009}

\maketitle


\section{Introduction}

Static Single Assignment, or SSA, has been a cornerstone of compiler design ever
since its introduction by \cite{ssa-original,rosen-gvn-1988} in the late 1980s.
Older compiler intermediate representations permitted the same local variable to
be assigned multiple times, whereas SSA requires each variable bindings to be
immutable -- they cannot be modified after initialisation. This makes many
dataflow analyses easier to compute, and (just as in functional languages) makes
it much easier to justify compiler optimisations in terms of equational
rewrites.

In fact, the relationship between SSA form and functional programming has been
known for decades: \citet{kelsey-ssa-cps} showed that SSA can be seen as a
subset of continuation-passing style, and \citet{appel-ssa} built on this to
observe that every SSA program can be viewed as a collection of first-order
functions which tail-call one another. This makes the equivalence of the
functional view to the traditional $\phi$-node presentation very easy to see. In
the functional style, each tail call lists its arguments. In contrast, a
$\phi$-node collects together the arguments from every callsite, which lets the
tail-calls/jumps to omit their arguments. Thus, the main difference between SSA
and a functional program is where in the program text the arguments to a tail
call are listed.

This "basic blocks with arguments" presentation of SSA has become very popular
in industry, with compiler backends such as Cranelift~\cite{cranelift},
MLIR~\cite{mlir}, and SIL~\cite{SIL} all using this representation. This tight
connection might suggest that the semantics of SSA representations is trivial:
we can just interpret an SSA program as a corresponding functional program, and
then rewrite the SSA programs using the equational theory of functional
programs.

While this is a perfectly sound thing to do, it is not strong enough to write a
good compiler. Compiler IRs like SSA are fundamentally about effectful programs,
and the equational theory of effectful functional programs is very weak. Instead
of a full substitution rule, the equational theory of impure functional
languages only justifies substituting values for variables (the $\beta$-value
rule). This weak equational rule suffices to validate some basic compiler
optimisations such as constant propagation and copy propagation, but is not
strong enough to justify many other essential compiler optimisations. For
example, hoisting a load out of a loop or conditional involves moving an
effectful term out of its original position in the program (see Equation
\ref{eqn:hoist}). As another example, dead store elimination removes an
effectful term from the program altogether.

%TODO: note that dead store elimination requires refinement, which we're not
%covering in this paper...

These examples illustrate that many optimisations require taking
\emph{effectful} terms and moving them around and substituting them for
variables. Since this is not a generally valid program transformation, we need a
finer classification of programs than merely pure and effectful to judge when a
substitution is allowed.

A useful taxonomy was introduced by \citet{fuhrmann-direct-1999}, classifying
effects based on (1) whether or not the order of effects mattered, (2) whether
it was sound to discard an effectful operation, and (3) whether it was sound to
duplicate an effectful operation. Suppose we substitute \texttt{print("hello")}
for $x$ in the program below:
\begin{equation}
\begin{BVerbatim}[baseline=c]
let x = print("hello");
let y = print("world");
let z = x;   
...
\end{BVerbatim}
\qquad \qquad \centernot\implies \qquad \qquad
\begin{BVerbatim}[baseline=c]
let y = print("world");
let x = print("hello");
...
\end{BVerbatim}
\end{equation}
In this case, substituting a term for a variable changes the order the I/O
actions are performed, which changes the program's observable behaviour. On the
other hand, suppose we have an effect which increments a counter. Then the same
transformation is sound, since \(x\) is used exactly once:
\begin{equation}
\begin{BVerbatim}[baseline=c]
let x = increment(a);
let y = increment(b);
let z = x;
...
\end{BVerbatim}
\qquad \qquad \implies \qquad \qquad
\begin{BVerbatim}[baseline=c]
let y = increment(b);
let z = increment(a);
...
\end{BVerbatim}
\end{equation}
F\"{u}hrmann called such commutative effects \emph{central}, which we also call
\emph{linear}. It can also matter whether an effect is performed at all or not.
For example, consider:
\begin{equation}
\begin{BVerbatim}[baseline=c]
let x = may_loop();
return 5 
\end{BVerbatim}
\qquad \qquad \centernot\implies \qquad \qquad
\begin{BVerbatim}[baseline=c]
return 5
\end{BVerbatim}
\end{equation}
In this example, even though $x$ is not used, the call to \texttt{may_loop()}
cannot be erased, as the observable behaviour of the program can change if it
goes into an infinite loop. On the other hand, if the effectful operation is
merely nondeterministic, then it is safe to discard results:
\begin{equation}
\begin{BVerbatim}[baseline=c]
let x = rand();
return 5 
\end{BVerbatim}
\qquad \qquad \implies \qquad \qquad
\begin{BVerbatim}[baseline=c]
return 5
\end{BVerbatim}
\end{equation}
F\"{u}hrmann called these operations \emph{discardable}. In this paper, we use
his terminology, and also call such effects \emph{affine}. Now consider the
following example: 
\begin{equation}
\begin{BVerbatim}[baseline=c]
let x = rand();
let y = x;
let z = x;
return y == z 
\end{BVerbatim}
\qquad \qquad \centernot\implies \qquad \qquad
\begin{BVerbatim}[baseline=c]
let y = rand();
let z = rand();
return y == z
\end{BVerbatim}
\end{equation}
In this case, substituting for $x$ duplicates the call to \texttt{rand()}, and
can change this from a program which always returns true to one which can return
either true or false. On the other hand, if the effect is the possibility of
nontermination, then the substitution is sound. 
\begin{equation}
\begin{BVerbatim}[baseline=c]
let x = may_loop();
let y = x;
let z = x;
return y == z 
\end{BVerbatim}
\qquad \qquad \implies \qquad \qquad
\begin{BVerbatim}[baseline=c]
let y = may_loop();
let z = may_loop();
return y == z
\end{BVerbatim}
\end{equation}
F\"{u}hrmann called such operations \emph{copyable}. In this paper, we use his
terminology, and also call such effects \emph{relevant}. With these distinction
in place, we can \emph{define} a pure expression as one which is all three of
central, copyable, and discardable.

In addition to working with effects, compiler intermediate representations also
have to track accesses to memory. This information is typically gathered via
ad-hoc alias analyses, but a compositional alternative to these analyses is to
track this information via substructural type systems such as linear and affine
types, which associate aliasing and ownership information with individual
substructural values. Having aliasing information justifies important
optimisations such as loop hoisting: 
\begin{equation}
\begin{BVerbatim}[baseline=c]
while (e) {
  x = load(a);
  body
}
\end{BVerbatim}
\qquad \qquad \implies \qquad \qquad
\begin{BVerbatim}[baseline=c]
x = load(a);
while(e) {
  body
}
\end{BVerbatim}
\label{eqn:hoist}
\end{equation}
This transformation is only sound if the body of the loop \texttt{body} does not
access \texttt{a}, which is precisely the kind of frame condition that linear
types can track. Moreover, since we already want to track whether effectful
\emph{expressions} can be duplicated or discarded, the very same machinery can
be reused to track whether \emph{values} are linear/affine/relevant.

Historically, compiler writers have treated the semantics of their IR quite
informally, since the "intended model" of the language is a simple sequential
machine operating over a flat array of bytes. Unfortunately,
over time both the intended model and the actual hardware have drifted away from
this simple mental model. Compiler frameworks have to model concurrent accesses
to memory in order to compile multithreaded code, and furthermore the IR's
memory model has to be sound with respect to the actual weak memory models
supported by hardware.

As a result, it is now much harder to judge whether a particular optimisation is
sound or not. Moreover, different kinds of hardware such as x86, GPUs, and ARM
all offer different memory models. This means that we need an abstract semantics
for compiler IRs, which can permit a separation of concerns. This will let
compiler writers validate their optimisations against the abstract semantics,
and will let us separately show that a particular machine model faithfully
implements the semantics.

In this paper, we show that this is possible. We give a type system for SSA
which uses substructural types to both track effects at a fine-grained level,
and to track ownership of values. We then rigorously justify a variety of
optimisations by reference to an abstract categorical axiomatisation of this
language. We then show that our categorical axioms have a variety of concrete
models, particularly including one of TSO-style weak memory originally
introduced by \citet{sparky}. This lets us argue that a host of SSA-based
optimisations are sound with respect to the x86 memory model.

\paragraph{Contributions} 

\begin{itemize}
\item We give a type system for SSA-style programs, which has a full
substitution theorem and equational theory. This language includes substructural
types, both for controlling effects and tracking ownership in values, and our
strong substitution principle makes formulating inlining and rewriting
optimisations easier than in the traditional CFG-based formulations. 
\item We give an axiomatic semantics for our language, by showing that it has a
compositional intepretation in any effectful category with an Elgot structure.
We use this structure to give generic proofs of the soundness of substitution
and rewriting properties.
\item Our abstract semantics justifies a variety of powerful optimisation
techniques, including E-graph rewriting, inlining, loop fusion, hoisting, and
strength reduction. 
\item We show how Elgot monads give rise to instances of our categorical
semantics. We use this to derive semantics from a general trace monad, and show
that various monad transformers such as state transformers preserve the needed
Elgot structure. 
\item We use these tools to demonstrate the existence of a variety of concrete
models satisfying these categorical axioms. Starting first from simple languges
such as state plus printing, we also show how more challenging semantics such as
the TSO semantics for weak memory also fit into our framework. This thus gives
rise to an SSA-based IR with support for weak memory operations, which is fully
semantically-justified.
\item Our weak memory semantics and our results about Elgot monads are
formalised in the Lean theorem prover.
\end{itemize}

\section{SSA Syntax}

\TODO{terms vs expression/blocks/label-sets}

\TODO{standardize "branching to" wordmark? Is wordmark the word here?}

In a standard SSA-based compiler, to represent a function, we begin by
decomposing it into basic blocks, which can be viewed as \textit{label},
followed by a sequence of \textit{instructions}, each defining a
\textit{variable}, followed by a \textit{terminator}. A program written in this
manner being in "SSA form" simply means that every variable name is defined
exactly once. For example, the program in Figure \ref{fig:3-addr} has 3 basic blocks, one of which is a distinguished \textit{entry block}, but is \textit{not} in SSA form, since the variables \(m, n\) and \(i\) are each modified through subsequent assignments (\(t\) is recomputed on every iteration of the loop, but since it only has one definition, it \textit{is} in SSA form). To convert this program to SSA form, we would traditionally use \(\phi\)-functions, or, equivalently, annotate each basic block with arguments, as in Figure \ref{fig:bb-arg}.

\begin{figure}
  \centering
  \begin{subfigure}[t]{0.33\textwidth}
    \begin{BVerbatim}[baseline=t, gobble=3]
    'entry:
      m = 0
      n = 1
      brz i 'exit 'loop
    'loop:
      t = add m n
      m = n
      n = t
      i = sub i 1
      brz i 'exit 'loop
    'exit:
      ret m
    \end{BVerbatim}
    \caption{3-address code}
    \label{fig:3-addr}
  \end{subfigure}
  \begin{subfigure}[t]{0.32\textwidth}
    \begin{BVerbatim}[baseline=t, gobble=3]
      'entry:
      m0 = 0
      n0 = 1
      brz N 
        'exit (m0) 
        'loop (i, m0, n0)
    'loop(i0, m1, n1):
      m2 = n1
      n2 = add m1 n1
      i1 = sub i0 1
      brz z1 
        'exit (m2) 
        'loop (i1, m2, n2)
    'exit(m3):
      ret m3
    \end{BVerbatim}
    \caption{Basic blocks with arguments}
    \label{fig:bb-arg}
  \end{subfigure}
  \begin{subfigure}[t]{0.33\textwidth}
    \begin{BVerbatim}[baseline=t, gobble=3]
    let m0 = 0;
    let n1 = 1;
    if eq i 0 {
      br 'exit (m0)
    } else {
      br 'loop (i, m0, n0)
    } where
    'loop(i0, m1, n1) =>
      let m2 = n1;
      let n2 = add m1 n1;
      let i1 = sub i0 1;
      if eq i1 0 {
        br 'exit (m2)
      } else {
        br 'loop (i1, m2, n2)
      }
    'exit(m3) => ret m3
    \end{BVerbatim}
    \caption{Our syntax}
    \label{fig:our-syn}
  \end{subfigure}
  \caption{A simple program for computing the \texttt{i}th Fibonacci number,
  represented as 3-address code, basic-blocks with arguments, and in our SSA
  syntax} 
  \Description{}
  \label{fig:ssa-examples}
\end{figure}

For our type theoretic presentation of SSA, we will treat a very slight
generalization of the standard basic blocks with arguments syntax, in that we
will support \textit{nested scopes} and, in particular, \textit{if-statements}.
The program in Figure \ref{fig:bb-arg} translated to our syntax is given in
Figure \ref{fig:our-syn}. These do not introduce any new expressivity: an
if-statement can be viewed as syntactic sugar for a conditional branch to one of
two anonymous basic blocks, while nested scopes simply \textit{restrict} where
we can branch to. 
% In particular, in Section \ref{ssec:ssa-norm}, we will formalize
% interconversion with traditional SSA and show how it is semantically sound to
% simply merge all nested scopes. The point of the slightly modified syntax is
% simply to make some of our metatheoretic results more practical to state.

The grammar for our syntax is given in Figure \ref{fig:ssa-grammar}, while the
typing judgements themselves are given in \ref{fig:ssa-judgements}. We begin by
giving a grammar for \textit{types} \(A, B, C\), parametrized by a set of
\textit{base types} \(X\); all syntactically well-formed types are also
semantically valid. We postulate the existence of a \textit{unit type}
\(\mb{1}\), a \textit{boolean type} \(\mb{2}\), and all tensor products \(A
\otimes B\); we write \(\otimes\) to underscore the fact that types are allowed
to be substructural, assigning every type a linearity \(\ms{lin}(A)\) satisfying
the constraints in Figure \ref{fig:ssa-structural}. In particular, the rules
\rle{unit-lin} and \rle{bool-lin} state that both the unit and boolean types
must be intuitionistic; the rule \rle{pair-lin} says that the linearity of a
pair is simply the intersection of the linearity of its components. The rules
\rle{nil-lin} and \rle{cons-lin} together say that the linearity of a context is
the intersection of the linearity of its variables, where the linearity of a
variable is the intersection of the linearity of its type and its quantity
annotation.

Our grammar proper is split into three main syntactic categories:
\begin{itemize}
  \item \textit{Expressions} \(a, b, c, d, e\), which are typed via the
  judgement \(\hasty{\Gamma}{p}{a}{A}{q}\), which says that the expression \(a\)
  is of type \(A\) with \textit{centrality} \(p = 1\) if \textit{central}, or \(p = 0\)
  otherwise, and \textit{quantity} \(q \subseteq \{\ms{a}, \ms{r}\}\). The rules
  for this judgement are given in Figure \ref{fig:ssa-term-typing}.

  So the centrality $p$ indicates whether it is safe to substitute this expression
  for a variable, and $q$, plus the linearity of $A$, control how many times that
  variable can occur. 
  
  \item \textit{Blocks} \(s, t\), which are typed via the judgement
  \(\haslb{\ms{L}}{t}{\Gamma}\), which says that the block \(t\) is well-formed
  with respect to the context \(\Gamma\) and branches only to labels in
  \(\ms{L}\). The rules for this judgement are given in Figure
  \ref{fig:ssa-block-typing}.

\item \textit{Label-sets} or \textit{control-flow graphs} \(L\), which are
  typed via the judgement \(\lhaslb{\ms{L}}{L}{\ms{K}}\), which says that \(L\)
  defines a control-flow graph with inputs \(\ms{L}\) and outputs \(\ms{K}\)
  (some of which may also be in \(\ms{L}\), indicating recursive control-flow).
  The rules for this judgement are given in Figure \ref{fig:ssa-block-typing}.
\end{itemize}
Here \textit{contexts} \(\Gamma\), consist of a list of variables \(x\) given a
type \(A\) annotated with a \textit{quantity} \(q \subseteq \tint\). A variable
\(x: A^q\) is called \textit{affine} if its type \(A\) is affine (i.e.
\(\tlin{A}{\taff}\)) \textit{and} \(\taff \subseteq q\), in which case it can be
freely ignored. Similarly, \(x\) is \textit{relevant} if its type \(A\) relevant
(i.e. \(\tlin{A}{\trel}\)) \textit{and} \(\trel \subseteq q\), in which case it
can be freely ignored. This is formalized in the structural judgement
\(\csplits{\Gamma}{\Delta}{\Xi}\), which says that \(\Gamma\) can be split into
subcontexts \(\Delta\) and \(\Xi\) (weakening is simply the special case
\(\cwk{\Gamma}{\Delta} \iff \csplits{\Gamma}{\Delta}{\cdot}\)), as per the
typing rules in Figure \ref{fig:ssa-structural}. In particular, the rule
\rle{split-nil} provides a base case by saying that the empty context can be
split into two empty contexts, while the rules \rle{split-left} and
\rle{split-right} send \(x: A^q\) to the left or right subcontext repsectively;
both these rules also allow us to reduce the quantity \(q\) to a quantity \(r
\subseteq q\), "forgetting" some of our capabilities. The rule \rle{split-dup}
allows the duplication of a variable which has a relevant type \(A\)
\textit{and} a relevant quantity \(q\), with the option to weaken the quantity
to \textit{different} quantities \(r\) and \(s\) on the left-hand side and
right-hand sides respectively. Finally, the rule \rle{split-aff} simply allows
us to ignore a variable \(x: A^q\) if \(A\) and \(q\) are affine.

\textit{Label-contexts} \(\ms{L}\) are defined to consist of a list of
\textit{labels} \(\lbl{\ell}\) annotated with a context \(\Gamma\) and an
argument type \(A\). The judgement \(\lwk{\ms{L}}{\ms{K}}\) states that
\(\ms{L}\) is a subcontext of \(\ms{K}\), i.e., that every program which
"branches to \(\ms{L}\)" (i.e., finishes by either nontermination or branching
to one of the labels \(\lbl{\ell}\) with the appropriate preserved variables
\(\Gamma\) and argument \(A\)) can also be viewed as branching to \(\ms{K}\).
The conditions for this are formalized in Figure \ref{fig:ssa-structural} as
follows: the rule \rle{join-nil} provides a base case stating that the empty
label weakens into itself. The rule \rle{join-cons} says that if \(\ms{L}\)
weakens into \(\ms{K}\) and \(\Delta\) weakens into \(\Gamma\), then
\(\lwk{\ms{L}, \llhyp{\ell}{\Gamma}{A}}{\ms{K}, \llhyp{\ell}{\Delta}{A}}\),
allowing us to "forget" any extra variables dropped by \(\cwk{\Gamma}{\Delta}\).
Finally, the rule \rle{join-zero} simply says that if \(\lwk{\ms{L}}{\ms{K}}\),
then \(\ms{L}\) also weakens to any extension of \(\ms{K}\). Note that this
judgement is "backwards," in that the "larger" context is on the \textit{right};
this corresponds to the placement of the label context on the right-hand side of
the typing judgement for blocks, and is by analogy to our semantics for blocks
and label-sets in Section \ref{sec:semantics}.

\begin{figure}
  \begin{center}
    \begin{grammar}
      <\(A, B, C\)> ::= 
      \(X\)
      \;|\; \(\mathbf{1}\)
      \;|\; \(\mathbf{2}\)
      \;|\; \(A \otimes B\)

      <\(a, b, c, e\)> ::= \(x\) 
      \;|\; \(f\;a\)
      \;|\; \((a, b)\) 
      \;|\; \(()\) 
      \;|\; \(\ctt\) 
      \;|\; \(\cff\)
      \;|\; \(\letexpr{x}{a}{e}\)
      \;|\; \(\letexpr{(x, y)}{a}{e}\)
      % \;|\; \(\lbsplice{\ell}{x: A}{t}\)
      
      <\(s, t\)> ::= \(\lbrb{\ell}{a}\) 
      \;|\; \(\ite{e}{s}{t}\)
      \;|\; \(\letstmt{x}{a}{t}\)
      \;|\; \(\letstmt{(x, y)}{a}{t}\)
      \;|\; \(\ewhere{t}{L}\)

      <\(L\)> ::= \(\cdot\) \;|\; \(\lwbranch{\ell}{x: A}{t}, L\)

      <\(\Gamma\)> ::= \(\cdot\) \;|\; \(\Gamma, \thyp{x}{A}{q}\)

      <\(\ms{L}\)> ::= \(\cdot\) \;|\; \(\ms{L}, \lbl{\ell}[\Gamma](x: A)\)

      <\(p\)> ::= 0 \;|\; 1

      <\(q\)> ::= \(\varnothing\) 
      \;|\; \(\taff\) 
      \;|\; \(\trel\) 
      \;|\; \(\tint\)
    \end{grammar}
  \end{center}
  \caption{Grammar for \isotopessa, parametrized over a set of instructions \(f \in \mc{I}\)}
  \Description{Grammar for isotope-SSA}
  \label{fig:ssa-grammar}
\end{figure}

\begin{figure}
  \begin{center}        
    \begingroup
    \renewcommand{\arraystretch}{1.5}
    \setlength{\tabcolsep}{2em}
    \begin{tabular}{rl}
        \multicolumn{1}{c}{Judgment} & \multicolumn{1}{c}{Meaning} \\ \hline
        \(\hasty{\Gamma}{p}{a}{A}{q}\) & \(a\) is a term of type \(A\) in
        context \(\Gamma\) with centrality \(p\) and quantity \(q\) \\
        \(\haslb{\Gamma}{t}{\ms{L}}\) & \(t\) is a block targeting labels
        \(\ms{L}\) in context \(\Gamma\) \\
        \(\lhaslb{\ms{L}}{L}{\ms{K}}\) & The label-set \(L\) sends labels
        \(\ms{L}\) to labels \(\ms{K}\) \\
        \(\csplits{\Gamma}{\Delta}{\Xi}\) & The context \(\Gamma\) splits into
        \(\Delta\) and \(\Xi\) \\
        \(\lwk{\ms{L}}{\ms{K}}\) & The label-set \(\ms{L}\) weakens to the
        label-set \(\ms{K}\) \\
        % \(\tlin{A}{q}\) &
        % The type \(A\) can be used with linearity \(q\) \\
        % \(\ltlin{A}{r}{q}\) &
        % The type \(A\) can be used with linearity \(q \subseteq r\)
        % (i.e. \(\tlin{A}{q} \land q \subseteq r\)) \\
        % \(\tlin{\Gamma}{q}\) &
        % The context \(\Gamma\) has linearity \(q\) \\
        \(\cwk{\Gamma}{\Delta}\) &
        \(\Gamma\) is a weakening of \(\Delta\) 
        (i.e. \(\csplits{\Gamma}{\Delta}{\cdot}\))
    \end{tabular}
    \endgroup
  \end{center}
  \caption{Typing judgements for \isotopessa}
  \Description{Typing judgements for isotope-SSA}
  \label{fig:ssa-judgements}
\end{figure}

\begin{figure}
  \begin{gather*}    
    % \prftree[r]{\rle{base-lin}}{q \subseteq \ms{lin}(X)}{\tlin{X}{q}} \qquad
    \prftree[r]{\rle{unit-lin}}{}{\tlin{\mathbf{1}}{q}} 
    \qquad
    \prftree[r]{\rle{bool-lin}}{}{\tlin{\mathbf{2}}{q}} 
    \qquad
    \prftree[r]{\rle{pair-lin}}{\tlin{A}{q}}{\tlin{B}{q}}{\tlin{A \otimes B}{q}} \\
    \prftree[r]{\rle{nil-lin}}{\tlin{\cdot}{q}} \qquad
    \prftree[r]{\rle{cons-lin}}{\ltlin{A}{r}{q}}{\tlin{\Gamma}{q}}
      {\tlin{\Gamma, \thyp{x}{A}{r}}{q}} \\
    \prftree[r]{\rle{split-nil}}{\csplits{\cdot}{\cdot}{\cdot}} \qquad
    \prftree[r]{\rle{split-left}}
      {\csplits{\Gamma}{\Delta}{\Xi}}
      {r \subseteq q}
      {\csplits{\Gamma, \thyp{x}{A}{q}}{\Delta, \thyp{x}{A}{r}}{\Xi}} \qquad
    \prftree[r]{\rle{split-right}}
      {\csplits{\Gamma}{\Delta}{\Xi}}
      {r \subseteq q}
      {\csplits{\Gamma, \thyp{x}{A}{q}}{\Delta}{\Xi, \thyp{x}{A}{r}}} \\
    \prftree[r]{\rle{split-dup}}
      {\csplits{\Gamma}{\Delta}{\Xi}}
      {\ltlin{A}{q}{\trel}}
      {r, s \subseteq q}
      {\csplits{\Gamma, \thyp{x}{A}{q}}{\Delta, \thyp{x}{A}{r}}{\Xi, \thyp{x}{A}{s}}}
      \qquad
    \prftree[r]{\rle{split-drop}}
      {\csplits{\Gamma}{\Delta}{\Xi}}
      {\ltlin{A}{q}{\taff}}
      {\csplits{\Gamma, \thyp{x}{A}{q}}{\Delta}{\Xi}}
      \\
    \prftree[r]{\rle{join-nil}}{\lwk{\cdot}{\cdot}} \qquad
    \prftree[r]{\rle{join-cons}}
      {\lwk{\ms{L}}{\ms{K}}}
      {\cwk{\Gamma}{\Delta}}
      {\lwk{\ms{L}, \llhyp{\ell}{\Gamma}{A}}{\ms{K}, \llhyp{\ell}{\Delta}{A}}} 
      \qquad
    \prftree[r]{\rle{join-zero}}
      {\lwk{\ms{L}}{\ms{K}}}
      {\lwk{\ms{L}}{\ms{K}, \llhyp{\ell}{\Gamma}{A}}} 
  \end{gather*}
  \caption{Structural rules for \isotopessa}
  \Description{Structural rules for isotope-SSA}
  \label{fig:ssa-structural}
\end{figure}

We can now get to the typing judgement for expressions in Figure
\ref{fig:ssa-term-typing}. The rule \rle{var} is essentially the standard rule
for variables, with the requirement that, other than \(x\), all variables in
\(\Gamma\) are affine. Note that the quantity \(q\) of the \textit{expression}
\(x\) is must satisfy \(q \subseteq q'\), where \(x: A^{q'} \in \Gamma\); this
reflects that we're allowed to \textit{forget} capabilities, but not
\textit{add} them. The \rle{app} rule is also mostly standard: given an
instruction \(f\) which takes in an argument \(A\) and has a return type of
\(B\) and a \textit{central} expression \(\hasty{\Gamma}{1}{a}{A}{q}\), we
conclude that \(\hasty{\Gamma}{p}{f\;a}{B}{q}\). We, of course, require that
\(f\) itself have centrality \(p\) and quantity \(q\), i.e. that \(f \in
\mc{I}_p^q(A, B)\). The \rle{pair} rule says that we can type a pair \((a, b)\)
in \(\Gamma\) if \(\csplits{\Gamma}{\Delta}{\Xi}\) where \(\Delta\) types \(a\)
and \(\Xi\) types \(b\); the quantity of a pair is simply the intersection of
the quantity of its components (this is expressed in the rule by allowing any
quantity which types \textit{both} components), and both components must be
central. Finally, we have the rules \rle{let} and \rle{let2}, corresponding to
let-bindings and destructuring let-bindings respectively. In both cases, we
split the context \(\Gamma\) into subcontext \(\Delta, \Xi\), one of which is
used to type a central subterm \(a\), which is bound to variables \(x\) or
destructured to variables \(x, y\) respectively, in a potentially non-central
term \(e\) which is otherwise typed in \(\Delta\).

\begin{figure}
  \begin{gather*}    
    \prftree[r]{\rle{var}}
      {\cwk{\Gamma}{\thyp{x}{A}{q}}}
      {\hasty{\Gamma}{p}{x}{A}{q}} \qquad
    \prftree[r]{\rle{app}}
      {f \in \mc{I}_p^q(A, B)}
      {\hasty{\Gamma}{1}{a}{A}{q}}
      {\hasty{\Gamma}{p}{f\;a}{B}{q}} \qquad
    \prftree[r]{\rle{pair}}
      {\csplits{\Gamma}{\Delta}{\Xi}}
      {\hasty{\Delta}{1}{a}{A}{q}}
      {\hasty{\Xi}{1}{b}{B}{q}}
      {\hasty{\Gamma}{p}{(a, b)}{A \otimes B}{q}} \\
    \prftree[r]{\rle{unit}}
      {\cwk{\Gamma}{\cdot}}
      {\hasty{\Gamma}{p}{()}{\mb{1}}{q}} \qquad
    \prftree[r]{\rle{true}}
      {\cwk{\Gamma}{\cdot}}
      {\hasty{\Gamma}{p}{\ctt}{\mb{2}}{q}} \qquad
    \prftree[r]{\rle{false}}
      {\cwk{\Gamma}{\cdot}}
      {\hasty{\Gamma}{p}{\cff}{\mb{2}}{q}} \\
    \prftree[r]{\rle{let}}
      {\csplits{\Gamma}{\Delta}{\Xi}}
      {\hasty{\Delta, \thyp{x}{A}{}}{p}{e}{B}{q}}
      {\hasty{\Xi}{1}{a}{A}{q}}
      {\hasty{\Gamma}{p}{\letexpr{x}{a}{e}}{B}{q}} \qquad
    % \prftree[r]{\rle{blk}}
    %   {\haslb{\Gamma}{t}{\llhyp{\ell}{\cdot}{A}}}
    %   {\hasty{\Gamma}{0}{\lbsplice{\ell}{A}{t}}{A}{\varnothing}} 
      \\
    \prftree[r]{\rle{let2}}
      {\csplits{\Gamma}{\Delta}{\Xi}}
      {\hasty{\Delta, \thyp{x}{A}{}, \thyp{y}{B}{}}{p}{e}{C}{q}}
      {\hasty{\Xi}{1}{a}{A \otimes B}{q}}
      {\hasty{\Gamma}{p}{\letexpr{(x, y)}{a}{e}}{C}{q}}
  \end{gather*}
  \caption{Typing rules for \isotopessa expressions}
  \Description{Typing rules for isotope-SSA expressions}
  \label{fig:ssa-term-typing}
\end{figure}

The rules for block typing are given in Figure \ref{fig:ssa-block-typing}.
\rle{br} says that, given a pure term \(a: A\) (that is, central, relevant, and
affine), we can type a block \(\lbrb{\ell}{a}\) which branches to \(\ell\) with
argument \(a\), where \(\ms{L}\) contains the context \(\ell\) with an argument
type of \(A\) and an appropriate set of unused variables. The rule \rle{ite}
typechecks if-then-else statements by stating that, if \(\Gamma\) splits into
\(\Delta\) and \(\Xi\), where \(\Xi\) typechecks a pure, intuitionistic
discriminant \(e: \mb{2}\) and \(\Delta\) typechecks both \(s, t\) to branch
into \(\ms{L}\), then the whole if-statement \(\ite{e}{s}{t}\) also typechecks
as branching into \(\ms{L}\) in constext \(\Gamma\). \rle{let-blk} and
\rle{let2-blk} are quite similar to their expression analogues, except that the
bound expression \(e\) is allowed to be impure (otherwise, programs could only
make use of pure functions!), and the remainder of the context plus the bound
variable is used to type a block rather than a term. Finally, probably the most
interesting rule is \rle{where}, which says that, given a block \(t\) which
branches to \(\ms{L}\) in context \(\Gamma\), and labels \(L\) mapping
\(\ms{L}\) to \(\ms{K}\), \(\ewhere{t}{L}\) branches to \(\ms{K}\). Here, \(L\)
mapping \(\ms{L}\) to \(\ms{K}\) is defined using the typing rules \rle{nil-br},
which says that an empty list of blocks leaves the target label-set unchanged,
and \rle{cons-br}, which says that if \(L\) maps \(\ms{L}\) to either \(\ms{K}\)
or the label \(\lbl{\ell}[\Gamma](A)\), and \(t\) branches to \(\ms{L}\) in the
context \(\Gamma, \thyp{x}{A}{}\), then \(L\) maps \(\ms{L}\) to \(\ms{K}\),
having (potentially recursively) defined \(\lbl{\ell}\).

\begin{figure}
  \begin{gather*}    
    \prftree[r]{\rle{br}}
      {\csplits{\Gamma}{\Delta}{\Xi}}
      {\lwk{\llhyp{\ell}{\Delta}{A}}{\ms{L}}}
      {\hasty{\Xi}{1}{a}{A}{\tint}}
      {\haslb{\Gamma}{\lbrb{\ell}{a}}{\ms{L}}} 
    \\
    \prftree[r]{\rle{ite}}
      {\csplits{\Gamma}{\Delta}{\Xi}}
      {\hasty{\Delta}{1}{e}{\mb{2}}{\tint}}
      {\haslb{\Xi}{s}{\ms{L}}}
      {\haslb{\Xi}{t}{\ms{L}}}
      {\haslb{\Gamma}{\ite{e}{s}{t}}{\ms{L}}} 
    \\
    \prftree[r]{\rle{let-blk}}
      {\csplits{\Gamma}{\Delta}{\Xi}}
      {\haslb{\Delta, \thyp{x}{A}{}}{t}{\ms{L}}}
      {\hasty{\Xi}{0}{a}{A}{q}}
      {\haslb{\Gamma}{\letstmt{x}{a}{t}}{\ms{L}}} 
    \\
    \prftree[r]{\rle{let2-blk}}
      {\csplits{\Gamma}{\Delta}{\Xi}}
      {\haslb{\Delta, \thyp{x}{A}{}, \thyp{y}{B}{}}{t}{\ms{L}}}
      {\hasty{\Xi}{0}{a}{A \otimes B}{q}}
      {\haslb{\Gamma}{\letstmt{(x, y)}{a}{t}}{\ms{L}}} \qquad
    \prftree[r]{\rle{where}}
      {\haslb{\Gamma}{t}{\ms{L}}}
      {\lhaslb{\ms{L}}{L}{\ms{K}}}
      {\haslb{\Gamma}{\ewhere{t}{L}}{\ms{K}}}
    \\
    \prftree[r]{\rle{nil-br}}
      {\lhaslb{\ms{L}}{\cdot}{\ms{L}}} \qquad
    \prftree[r]{\rle{cons-br}}
      {\lhaslb{\ms{L}}{L}{\ms{K}, \llhyp{\ell}{\Gamma}{A}}}
      {\haslb{\Gamma, \thyp{x}{A}{}}{t}{\ms{L}}}
      {\lhaslb{\ms{L}}{L, \lwbranch{\ell}{x: A}{t}}{\ms{K}}}
  \end{gather*}
  \caption{Typing rules for \isotopessa blocks}
  \Description{Typing rules for isotope-SSA blocks}
  \label{fig:ssa-block-typing}
\end{figure}

\subsection{Weakening and Substitution}

We can now state some of the basic metatheoretic properties of \isotopessa. We
begin with weakening, which we state in our setting as follows:
\begin{lemma}[Weakening] \
  \begin{itemize}
    \item If \(\cwk{\Gamma}{\Delta}\) and \(q \subseteq r\),
    \(\hasty{\Delta}{p}{a}{A}{r}\), then \(\hasty{\Gamma}{p}{a}{A}{q}\)
    \item If \(\cwk{\Gamma}{\Delta}\), \(\haslb{\Delta}{t}{\ms{L}}\), then \(\haslb{\Gamma}{t}{\ms{L}}\)
    \item If \(\lwk{\ms{L}}{\ms{K}}\), then:
    \begin{itemize}
      \item If \(\haslb{\Gamma}{t}{\ms{L}}\) then \(\haslb{\Gamma}{t}{\ms{K}}\)
      \item If \(\lhaslb{\ms{W}}{L}{\ms{L}}\) then \(\lhaslb{\ms{W}}{L}{\ms{L}}\)
    \end{itemize}
  \end{itemize}
\end{lemma}
That is, we can do the following and still keep our typing judgements valid
\begin{itemize}
  \item Add unused \textit{affine} variables
  \item \textit{Reduce} the quantity of an expressions type
  \item Add arbitrary unused labels, and remove unused \textit{affine} variables
  from existing labels
\end{itemize}
This can be proved by a relatively straightforward mutual induction.

\begin{figure}
  \begin{center}        
    \begingroup
    \renewcommand{\arraystretch}{1.5}
    \setlength{\tabcolsep}{2em}
    \begin{tabular}{rl}
        \multicolumn{1}{c}{Judgment} & \multicolumn{1}{c}{Meaning} \\ \hline
        \(\issubst{\gamma}{\Theta}{\Gamma}\) &
        \(\gamma\) is a substitution taking \(\Theta\) to \(\Gamma\) \\
        \(\lbsubst{\mc{L}}{\ms{L}}{\ms{K}}\) &
        \(\mc{L}\) is a label substitution taking \(\ms{L}\) to \(\ms{K}\) \\
        % \(\mhasty{H}{\Gamma}{p}{\mhole{a}}{A}{q}\) &
        % ... \\
        % \(\mhaslb{H}{\Gamma}{\mhole{t}}{\ms{L}}\) &
        % ... \\
        % \(\mlhaslb{H}{\ms{L}}{\mhole{L}}{\ms{K}}\) &
        % ... \\
        % \(\isrw{\mc{H}}{I}{H}\) &
        % ... \\
    \end{tabular}
    \endgroup
  \end{center}
  \caption{Metatheoretic typing judgements for \isotopessa}
  \Description{Metatheoretic typing judgements for isotope-SSA}
  \label{fig:ssa-meta-judgements}
\end{figure}

\begin{figure}
  \begin{gather*}
    \boxed{\gamma: \ms{Var} \to \ms{Expr}}
    \\
    \prftree[r]{\rle{subst-nil}}
      {\issubst{\gamma}{\cdot}{\cdot}}
      \qquad
    \prftree[r]{\rle{subst-cons}}
      {\issubst{\gamma}{\Theta_\Gamma}{\Gamma}}
      {\hasty{\Theta_x}{1}{\gamma(x)}{A}{q}}
      {\tlin{\Theta_x}{q}}
      {\csplits{\Theta}{\Theta_\Gamma}{\Theta_x}}
      {\issubst{\gamma}{\Theta}{\Gamma, \thyp{x}{A}{q}}}
    \\
    \boxed{\mc{L}: \ms{Label} \to \ms{Expr} \to \ms{Block} \text{ s.t. for fresh } x, [a/x]\mc{L}(\lbl{\ell}, x) = \mc{L}(\lbl{\ell}, a)}
    \\
    \prftree[r]{\rle{lb-subst-nil}}
    {\lbsubst{\mc{L}}{\cdot}{\ms{K}}}
      \qquad
    \prftree[r]{\rle{lb-subst-cons}}
      {\lbsubst{\mc{L}}{\ms{L}}{\ms{K}}}
      {\haslb{\Gamma, \thyp{x}{A}{}}{\mc{L}(\lbl{\ell}, x)}{\ms{K}}}
      {\lbsubst{\mc{L}}{\ms{L}, \llhyp{\ell}{\Gamma}{A}}{\ms{K}}}
  \end{gather*}
  \caption{ Typing rules for \isotopessa substitutions. We generally assume that
    for all but finitely many variables \(x\) and labels \(\lbl{\ell}\),
    \(\gamma(x) = x\) and \(\mc{L}(\lbl{\ell}, x) = \lbrb{\ell}{x}\). }
    \Description{Typing rules for isotope-SSA substitutions}
  \label{fig:ssa-subst-typing}
\end{figure}

Our main metatheoretic results will be \textit{substitution} and
\textit{label-substitution}, which will require introducing some substitution
judgements to state properly, which we do in Figure
\ref{fig:ssa-meta-judgements}. In particular, the judgement
\(\issubst{\gamma}{\Theta}{\Gamma}\) says that \(\gamma\), which is taken to be
a map from variable names to expressions, is a valid substitution taking
\(\Theta\) to \(\Gamma\). What this means is that, for each variable
\(\thyp{x}{A}{q}\) in \(\Gamma\), \(\gamma(x)\) can be interpreted as a
\textit{central} term of type \(A^q\) in a subcontext of \(\Theta\), such that
when this is done for all \(\thyp{x}{A}{q} \in \Gamma\) any non-affine variables
in \(\Theta\) are used up: this is described formally via the typing rules
\rle{subst-nil}, which says that any function \(\gamma\) takes the empty context
to itself, and \rle{subst-cons}, which says that if \(\gamma\) takes
\(\Theta_\Gamma\) to \(\Gamma\), \(\hasty{\Theta_x}{1}{\gamma(x)}{A}{q}\), \(q
\subseteq \ms{lin}(\Theta_x)\) (to avoid necessating the duplication/discarding
of substructural variables), and \(\Theta\) splits into \(\Theta_\Gamma\) and
\(\Theta_x\), then \(\issubst{\gamma}{\Theta}{\Gamma, \thyp{x}{A}{q}}\).

Given such a substitution \(\gamma\), there are two ways to use it. The first is
standard capture-avoiding substitution, which we denote \([\gamma]e\) for
expressions, \([\gamma]t\) for blocks, and \([\gamma]L\) for label-sets, as is
standard. The second is to use the substitution to generate a set of
let-bindings for a context \(\Gamma\), written
\(\exprletsubst{\substctx{\gamma}{\Gamma}}{e}\) for expressions and
\(\stmtletsubst{\substctx{\gamma}{\Gamma}}{t}\) for blocks, as defined in Figure
\ref{fig:ssa-subst-ops}. Note that this operation cannot be easily defined for
label-sets. In Section \ref{sec:semantics}, we will show that this is
semantically equivalent to regular capture-avoiding substitution, due to the
fact that all terms in a substitution are central \textit{and} that only affine
computations can be used in an expression substituting for an affine variable,
and only relevant computationbs can be used in an expression substituting for a
relevant variable.

We can now state our substitution theorem for terms in a straightforward way: if
\(\issubst{\gamma}{\Theta}{\Gamma}\) and \(\hasty{\Gamma}{p}{e}{A}{q}\), then
\(\hasty{\Theta}{p}{[\gamma]e}{A}{q}\) and
\(\hasty{\Theta}{p}{\exprletsubst{\substctx{\gamma}{\Gamma}}{e}}{A}{q}\).
Unfortunately, the substitution theorem for blocks is a little more complicated:
the naive choice of attempting to show \(\haslb{\Gamma}{t}{\ms{L}} \implies
\haslb{\Theta}{[\gamma]t}{\ms{L}}\) does not work, since \(\ms{L}\) may contain
captured variables from \(\Gamma\) that are no longer available if we are typing
\([\gamma]t\) in the context of \(\Theta\). 

\begin{figure}
  \begin{gather*}
    % \restrictsubst{\gamma}{\Gamma}(x) 
    %   =
    %   (\ms{if}\; x \in \Gamma
    %   \;\ms{then}\;\gamma(x)
    %   \;\ms{else}\;x)
    % \\
    \stmtletsubst{\substctx{\gamma}{\cdot}}{t} = t
    \qquad
    \stmtletsubst{\substctx{\gamma}{\Gamma, x: A^q}}{t} 
      = \letstmt{x}{\gamma(x)}{\stmtletsubst{\substctx{\gamma}{\Gamma}}{t}}
    \qquad
    \exprletsubst{\cdot}{e} = e
    \qquad
    \exprletsubst{\substctx{\gamma}{\Gamma, x: A^q}}{e} 
      = \letstmt{x}{\gamma(x)}{\substctx{\gamma}{\Gamma}}
    \\
    \substlbs{\gamma}{\cdot}(\lbl{\ell}, x) = \lbrb{\ell}{x}
    \qquad
    \substlbs{\gamma}{\ms{L}, \llhyp{\ell}{\Gamma}{A}}(\lbl{k}, x)
      = 
      (\ms{if}\; \lbl{k} = \lbl{\ell} 
      \;\ms{then}\;\stmtletsubst{\substctx{\gamma}{\Gamma}}{\lbrb{\ell}{x}}
      \;\ms{else}\;\substlbs{\gamma}{\ms{L}}(\lbl{\ell}, x))
    \\
    [\mc{L}](\lbrb{\ell}{a}) = \mc{L}(\lbl{\ell}, a)
    \qquad
    [\mc{L}]\ite{e}{s}{t} = \ite{e}{[\mc{L}]s}{[\mc{L}]t}
    \\
    [\mc{L}](\letstmt{x}{a}{t}) = (\letstmt{x}{a}{[\mc{L}]t})
    \qquad
    [\mc{L}](\letstmt{(x, y)}{a}{t}) = (\letstmt{(x, y)}{a}{[\mc{L}]t})
    \\
    [\mc{L}]\ewhere{t}{L} = \ewhere{[\mc{L}]t}{[\mc{L}]L}
    \\
    [\mc{L}]\cdot = \cdot
    \qquad
    [\mc{L}](\lwbranch{\ell}{x: A}{t})
    = (\lwbranch{\ell}{x: A}{[\mc{L}]t})
  \end{gather*}
  \caption{Operations on \isotopessa substitutions.}
  \Description{Operations on isotope-SSA substitutions}
  \label{fig:ssa-subst-ops}
\end{figure}

To solve this problem, we need to introduce the concept of a
\textit{label-substitution} \(\mc{L}\), which we define as a function that takes
labels \(\lbl{\ell}\) and variables \(x\) to blocks \(t\) such that, for any
fresh variable \(x\), \([a/x]\mc{L}(\lbl{\ell}, x) = \mc{L}(\lbl{\ell}, a)\). We
introduce the judgement \(\lbsubst{\mc{L}}{\ms{L}}{\ms{K}}\) to state that the
label-substitution \(\mc{L}\) takes the label-set \(\ms{L}\) to the label-set
\(\ms{K}\); that is, for each \(\lbl{\ell}[\Gamma](x: A) \in \ms{L}\), we have
that \(\mc{L}(\lbl{\ell}, x)\) is a block that branches to \(\ms{K}\). This is
formally defined via the rules \rle{lb-subst-nil}, which states that this is
vacuously true for \(\ms{L} = \cdot\), and \rle{lb-subst-cons}, which says that
if \(\lbsubst{\mc{L}}{\ms{L}}{\ms{K}}\) and \(\haslb{\Gamma,
\thyp{x}{A}{}}{\mc{L}(\lbl{\ell}, x)}{\ms{K}}\) then \(\lbsubst{\mc{L}}{\ms{L},
\lhyp{\lbl{\ell}}{\Gamma}{A}}{\ms{K}}\), in Figure \ref{fig:ssa-subst-typing}.
Given a context \(\gamma\) and a label-set \(\ms{L}\), we can convert it to a
label-substitution \(\substlbs{\gamma}{\ms{L}}\) using the definition in Figure
\ref{fig:ssa-subst-ops}: using this, we can state the full substitution lemma as
follows:
\begin{lemma}[Substitution] 
  Given \(\issubst{\gamma}{\Theta}{\Gamma}\),
  \begin{itemize}
    \item If \(\hasty{\Gamma}{p}{a}{A}{q}\), then
    \(\hasty{\Theta}{p}{[\gamma]a}{A}{q}\) and
    \(\hasty{\Theta}{p}{\exprletsubst{\substctx{\gamma}{\Gamma}}{a}}{A}{q}\)
    \item If \(\haslb{\Gamma}{t}{\ms{L}}\), then there exists \(\ms{K}\) such
    that \(\lbsubst{\substlbs{\gamma}{\ms{L}}}{\ms{K}}{\ms{L}}\) and
    \(\haslb{\Theta}{[\gamma]t}{\ms{K}}\),
    \(\haslb{\Theta}{\stmtletsubst{\substctx{\gamma}{\Gamma}}{t}}{\ms{K}}\)
    \item If \(\lhaslb{\ms{W}}{L}{\ms{L}}\), then there exists \(\ms{K},
    \ms{W}'\) such that 
    \begin{itemize}
      \item \(\lbsubst{\substlbs{\gamma}{\ms{L}}}{\ms{K}}{\ms{L}}\) 
      \item \(\lbsubst{\substlbs{\gamma}{\ms{W}'}}{\ms{W}}{\ms{W}'}\) 
      \item \(\lhaslb{\ms{W}'}{[\gamma]L}{\ms{K}}\)
    \end{itemize}
  \end{itemize}
\end{lemma}

% \begin{lemma}[Substitution Splitting] 
%   If \(\csplits{\Gamma}{\Delta}{\Xi}\) and \(\issubst{\gamma}{\Theta}{\Gamma}\),
%   then there exist \(\Theta_\Delta\), \(\Theta_\Xi\) such that:
%   \begin{itemize}
%     \item \(\csplits{\Theta}{\Theta_\Delta}{\Theta_\Xi}\)
%     \item \(\issubst{\gamma}{\Theta_\Delta}{\Delta}\)
%     \item \(\issubst{\gamma}{\Theta_\Xi}{\Xi}\)
%   \end{itemize}
% \end{lemma}

% \TODO{some text about substitution weakening}

We can also, using the rules in Figure \ref{fig:ssa-subst-ops}, define
capture-avoiding label-substitution for blocks \([\mc{L}]t\) and label-sets
\([\mc{L}]L\), allowing us to state the following lemma:
\begin{lemma}[Label Substitution] 
  Given \(\lbsubst{\mc{L}}{\ms{L}}{\ms{K}}\), 
  \begin{itemize}
    \item For all \(\haslb{\Gamma}{t}{\ms{L}}\),
    \(\haslb{\Gamma}{[\mc{L}]t}{\ms{K}}\)
    \item For all \(\lhaslb{\ms{W}}{L}{\ms{L}}\),
    \(\lhaslb{\ms{W}}{[\mc{L}]L}{\ms{K}}\)
  \end{itemize}
\end{lemma}

\TODO{small rewriting section here?}

\section{SSA Semantics}

\label{sec:semantics}

In this section, we will give a categorical semantics for \isotopessa. Starting
with Freyd categories, we will generalize to \textit{substructural effectful
categories} to support substructurality. We will then require our substructural
effectful category to have \textit{distributive coproducts} to implement
branching. Finally, we will require an \textit{Elgot structure} to implement
general, potentially nonterminating control flow.

\subsection{Substructural Categories}

We begin with some basic definitions:
\begin{definition}[Binoidal Category] 
  A \textbf{binoidal category} \(\mc{C}\) is a category equipped with a binary
  operation on objects \(\otimes: \mc{C} \times \mc{C} \to \mc{C}\) which is
  independently functorial in each argument (i.e., for each object \(A\), \(A
  \otimes -\) and \(- \otimes A\) are functors).
\end{definition}
Given \(f: \mc{C}(A, B)\), \(g: \mc{C}(A', B')\) in a binoidal category, we
write \(f \ltimes g = f \otimes A';B \otimes g\) and \(f \rtimes g = A \otimes
g;f \otimes B'\); we say \(f\) is \textbf{central} if, for all \(g\), \(f
\ltimes g = f \rtimes g\) and \(g \ltimes f = g \rtimes f\), in which case we
simply write \(f \otimes g\), \(g \otimes f\) respectively.
\begin{definition}[Premonoidal Category] 
  A \textbf{premonoidal category} is a binoidal category with an identity object
  \(I: \mc{C}\) such that
  \begin{itemize}
    \item There exist central natural isomorphisms \(\lambda_A: I \otimes A
    \simeq A\) (the \textbf{left unitor}), \(\rho_A: A \otimes I \simeq A\) (the
    \textbf{right unitor})
    \item There exists a central isomorphism \(\alpha_{A, B, C}: (A \otimes B)
    \otimes C \simeq A \otimes (B \otimes C)\) (the \textbf{associator})
    \item Satisfying the \textbf{pentagon law}: \todo{this}
    \item Satisfying the \textbf{triangle law}: \todo{this}
  \end{itemize}
  We say a premonoidal category is \textbf{symmetric} if it is equipped with a
  central natural isomorphism \(\sigma_{A, B}: A \otimes B \simeq B \otimes A\)
  (the \textbf{symmetry}) satisfying the \textbf{hexagon law} \todo{this}. We
  say a premonoidal category is \textbf{monoidal} if every morphism is central.
\end{definition}
One issue with the notion of a premonoidal category is that it is unclear what
exactly a premonoidal functor should be: if we want them to compose, we would
like to preserve central morphisms, but trying to preserve \textit{all} central
morphisms is too strict a notion. We want, in essence, to be able to distinguish
morphisms that are truly "pure" from those which simply happen to be central. To
do that, we introduce the notion of an \textit{effectful category} as follows:
\begin{definition}[Effectful Category] 
  An \textbf{effectful category} \(\mc{V} \to \mc{C}\) is an identity on objects
  functor \(\upg{\cdot}{}: \mc{V} \to \mc{C}\) from a monoidal category
  \(\mc{V}\) to a premonoidal category \(\mc{C}\) which preserves all
  premonoidal structure (i.e. sends associators to associators, unitors to
  unitors, and tensor products to tensor products). It is \textbf{symmetric} if
  both \(\mc{V}\) and \(\mc{C}\) are, and \(\upg{\cdot}{}\) preserves the
  symmetry. A \textbf{Freyd category} is an effectful category where \(\mc{V}\)
  is Cartesian.
\end{definition}

\TODO{confusion with notion of pure in truly "pure" above}

\TODO{text on Freyd categories}

\begin{definition}[(Symmetric) Premonoidal Subcategory]
  Given a (symmetric) premonoidal category \(\mc{C}\), we say a wide subcategory
  \(\mc{S} \subseteq \mc{C}\) is \textbf{(symmetric) premonoidal} if it is
  closed under whiskering and contains all associators, unitors, and symmetries.
  We say a subcategory is \textbf{monoidal} if, for all \(f: \mc{S}(A, B)\),
  \(g: \mc{S}(A', B')\), \(f \ltimes g = f \rtimes g\); we say it is
  \textbf{central} if this generalizes to the case where only one of \(f, g\) is
  in \(\mc{S}\) (i.e., if all morphisms in \(\mc{S}\) are central).
\end{definition} 
Note that the set of symmetric premonoidal subcategories, monoidal
subcategories, and central subcategories of a category \(\mc{C}\) are all closed
under intersections and therefore, in particular, form a complete
meet-semilattice with shared bottom element \(\ms{Skeleton}(\mc{C})\), where
\(\ms{Skeleton}(C)\) is the smallest possible symmetric premonodial subcategory.
The largest symmetric premonoidal subcategory is of course \(\mc{C}\), while the
largest central subcategory is the center of \(\mc{C}\), \(Z(\mb{C})\); the
existence of a top-element means that in these cases we may induce the structure
of a complete lattice in the usual manner.

% Should we go down this road?
% \begin{definition}[Strictly Effectful Category] \
%   \TODO{this}
% \end{definition}

\TODO{text}

\begin{definition}[Substructural (Pre)monoidal Category]
  A \textbf{substructural (pre)monoidal category} is a tuple \((\mc{C}, \mc{A},
  \mc{R})\) where:
  \begin{itemize}
    \item \(\mc{C}\) is a symmetric premonoidal category equipped with wide
    symmetric premonoidal subcategories \(\mc{C}^{\ms{a}}, \mc{C}^{\ms{r}}\).
    We define, for \(q \subseteq \{\ms{a}, \ms{r}\}\), \(\mc{C}^q = \bigcap_{\ell \in q}\mc{C}^\ell\), where \(\mc{C}^\varnothing = \mc{C}\)
    \item \(\mc{A} \subseteq |\mc{C}|\) is a set of \textbf{affine} objects
    closed under tensor products, such that
    \begin{itemize}
      \item \(\mb{1} \in \mc{A}\)
      \item Every \(A\) in \(\mc{A}\) is equipped with a central morphism
      \(\ms{drop}(A): \mc{C}^{\{\ms{a}, \ms{r}\}}(A, \mb{1})\)
      \item \(\forall A, B \in \mc{A}, \ms{drop}(A \otimes B) = \ms{drop}(A) \otimes \ms{drop}(B);\lambda_I\)
      \item \(\forall A, B \in \mc{A}, \forall f: \mc{C}^{\ms{a}}(A, B), f;\ms{drop}(B) = \ms{drop}(A)\)
    \end{itemize}
    \item \(\mc{R} \subseteq |\mc{C}|\) is a set of \textbf{relevant} objects
    closed under tensor products, such that
    \begin{itemize}
      \item \(\mb{1} \in \mc{R}\)
      \item Every \(A\) in \(\mc{R}\) is equipped with a central morphism
      \(\ms{split}(A): \mc{C}^{\{\ms{a}, \ms{r}\}}(A, A \otimes A)\)
      \item \(\forall A, B \in \mc{R}, \ms{split}(A \otimes B) = \ms{split}(A) \otimes \ms{split}(B);\alpha;A \otimes \sigma_{A, B} \otimes B;\alpha\)
      \item Associativity: \(\forall A \in \mc{R}, \ms{split}(A);\ms{split}(A) \otimes A;\alpha = \ms{split}(A);A \otimes \ms{split}(A)\)
      \item Commutativity: \(\forall A \in \mc{R}, \ms{split}(A);\sigma = \ms{split}(A)\)
      \item Comonoid: \(\forall A \in \mc{A} \cap \mc{R}, \ms{split}(A);A \otimes \ms{drop}(A) = \rho_A^{-1} \quad \ms{split}(A);\ms{drop}(A) \otimes A = \lambda_A^{-1}\)
      \item \(\forall A, B \in \mc{R}, \forall f: \mc{C}^{\ms{r}}(A, B), f;\ms{split}(B) = \ms{split}(A);f \ltimes f = \ms{split}(A);f\rtimes f\)
    \end{itemize}
  \end{itemize}
  We call the morphisms in \(\mc{C}^{\ms{a}}\) \textbf{affine}, the
  morphisms in \(\mc{C}^{\ms{r}}\) \textbf{relevant}, and the morphisms in
  \(\mc{C}^{\{\ms{a}, \ms{r}\}}\) \textbf{pure} or \textbf{intuitionistic}. We
  call the morphisms in \(\mc{C}^\varnothing\) but not necessarily in any of the
  other \(\mc{C}^q\) \textbf{linear}. We call a substructural \textit{monoidal}
  category a \textbf{substructural Cartesian category} if \(\mc{A} = \mc{R} =
  |\mc{C}|\), in which case \(\mc{C}^{\{\ms{a}, \ms{r}\}}\) is Cartesian.
\end{definition}
A simple example of a substructural Cartesian category is the category of
relations \(\ms{Rel}\), viewed as the Kleisli category of the power-set monad
with morphisms of the form \(\ms{Rel}(A, B) = A \to \mc{P}(B)\). In particular,
we can define:
\begin{equation}
  \begin{aligned}
    \ms{Rel}^{\ms{a}}(A, B) 
    &= \{R \in \ms{Rel}(A, B) \mid \forall a, |R(a)| \geq 1\} 
    &&= A \to \mc{P}^+(B) \\
    \ms{Rel}^{\ms{r}}(A, B) 
    &= \{R \in \ms{Rel}(A, B) \mid \forall a, |R(a)| \leq 1\} 
    &&= A \to \ms{Option}(B) \\
    \implies \ms{Rel}^{\{\ms{a}, \ms{r}\}}(A, B) 
    &= \{R \in \ms{Rel}(A, B) \mid \forall a, |R(a)| = 1\} 
    &&= A \to B
  \end{aligned}
\end{equation}
Here, \(\ms{split}(A)\) is just the copy operator \(a \mapsto (a, a)\) and
\(\ms{drop}(A)\) the deletion operator \(a \mapsto ()\); it is quite trivial to
prove that these lie in \(\ms{Rel}^{\{\ms{a}, \ms{r}\}}\), being functions, and
that they satisfy the axioms of a commutative comonoid, as desired. 

Note that we cannot simply have \(\ms{Rel}^{\ms{a}} = \ms{Rel}\), since, for
any morphism \(R: \ms{Rel}(A, B)\) such that \(\exists a \in A, R(a) =
\varnothing\), we have
\begin{equation}
  (R;\ms{drop}(B))(a) = \varnothing \neq \ms{drop}(A)(a) = \{()\}
\end{equation}
We will see that this corresponds to the fact that the following programs are
not semantically equal
\begin{equation}
  \ms{let}\;x = \ms{loop}(); e \not\simeq e \qquad \text{where} \qquad x \notin \ms{fv}(e)
\end{equation}
Similarly, we cannot simply have \(\ms{Rel}^{\ms{r}} = \ms{Rel}\), since,
for any morphism \(R: \ms{Rel}(A, B)\) such that \(\exists a \in A, |R(a)|
\supseteq \{b_0, b_1\}\) where \(b_0 \neq b_1\), we have that
\begin{equation}
  (R;\ms{split}(B))(a) = \{(b, b) \mid b \in R(a)\} \neq (\ms{split}(A);R)(a) = R(a) \times R(a) 
\end{equation}
since only the left-hand side contains the pair \((b_0, b_1)\), this
corresponding to the fact that the following programs are not semantically equal
\begin{equation}
  \ms{let}\;x = \ms{rand}(); (x, x) \not\simeq (\ms{rand}(), \ms{rand}())
\end{equation}
To get a substructural premonoidal category, we can apply the state monad
transformer to the power set monad to obtain the Kleisli category with morphisms
of the form \(\ms{Set}_{\ms{St}\;\mc{P}\;S}(A, B) = A \to S \to \mc{P}(B \times
S)\). Once again, it is clear that all objects have split and drop morphisms
inherited from \(\ms{Set}\), and therefore are affine and relevant

\begin{definition}[Susbtructural Effectful Category]
  A \textbf{substructural effectful category} \(\mc{C}\) consists of
  \begin{itemize}
    \item A substructural monoidal category \(\mc{C}_1\); we call the morphisms
    in this category \textbf{central} or \textbf{linear}
    \item A substructural premonoidal category \(\mc{C}_0\) sharing the same
    affine and relevant objects as \(\mc{C}_1\)
    \item An identity-on-objects functor \(\upg{\cdot}{}: \mc{C}_1^\varnothing
    \to \mc{C}_0^\varnothing\) preserving all substructural premonoidal
    structure, i.e., preserves all symmetries, associators, unitors, drops, and
    splits and sends \(\mc{C}_1^q\) to \(\mc{C}_0^q\). In particular,
    restricting \(\upg{\cdot}{}\) appropriately makes all \(\mc{C}_1^q \to
    \mc{C}_0^q\) into effectful categories.
  \end{itemize}
  We define \(\upg{f}{0}\) to be \(\upg{f}{}\) for morphisms in \(\mc{C}_1^q\)
  and simply \(f\) for morphisms in \(\mc{C}_0\), and \(\upg{f}{1}\) to simply
  be the identity functor on \(\mc{C}_1^q\).
 
  Morphisms in \(\mc{C}_p^{\ms{r}}\) are called \textbf{copyable}, morphisms
  in \(\mc{C}_p^{\ms{a}}\) are called \textbf{discardable}, while morphisms
  in \(\mc{C}_1^{\ms{a}}, \mc{C}_1^{\ms{r}}, \mc{C}_1^{\{\ms{a},
  \ms{r}\}}\) specifically are called \textbf{affine}, \textbf{relevant}, and
  \textbf{intuitionistic} or \textbf{pure} respectively.
\end{definition}

We can obtain an example of a substructural effectful category quite easily by,
considering the power-set monad above, applying the state transformer to yield
the Kliesli category with morphisms of the form \(\mc{C}_0(A, B) = A \to S \to
\mc{P}(A \times S)\). Letting \(\mc{C}_1^q = \ms{Rel}^q\) and defining
\(
  \upg{R}{} = \lambda a, s. R(a) \times \{s\}
\), we can verify that this satisfies the axioms of a substructural effectful category.
Note that here the \textit{central} morphisms are precisely those which do not
read from or write to the state, and therefore which commute with all other
morphisms.

In general, \textit{every} premonoidal category \(\mc{C}\) can be viewed a
substructural premonoidal category, with \(\mc{C}^\varnothing = \mc{C}\) and
\(\mc{C}^{\{\ms{a}, \ms{r}\}}, \mc{C}^{\{\ms{a}\}}, \mc{C}^{\{\ms{r}\}}\) the
subcategory consisting only of the associators, unitors, and symmetry, and only
the identity object considered affine and relevant. On the other extreme, every
\textit{Cartesian} category \(\mc{C}\) can be viewed as a substructural
Cartesian category with all morphisms both relevant and affine; in fact, this is
the case \textit{if and only if} a category is Cartesian, hence the term
"substructural Cartesian category."

We can also deduce the following:
\begin{proposition}
  If \(\mc{C}\) is a substructural monoidal category then the Kleisli category
  \(\mc{C}_{\mb{T}}\) is a substructural premonoidal category with:
  \begin{itemize}
    \item \(\ms{Aff}(\mc{C}_{\mb{T}}) = \ms{Aff}(\mc{C})\), \(\ms{Rel}(\mc{C}_{\mb{T}}) = \ms{Rel}(\mc{C})\)
    \item \(\mc{C}_{\mb{T}}^q(A, B) = \upg{\mc{C}^q(A, B)}{}\), where
    \(\upg{f}{} = f;\eta\)
    \item Drops \(\upg{\ms{drop}(A)}{}\) and splits \(\upg{\ms{split}(A)}{}\)
  \end{itemize} 
  In particular, \(\upg{\cdot}{}: \mc{C} \to \mc{C}_{\mb{T}}\) is a
  substructural effectful category.
\end{proposition}

\TODO{work into text?}

\subsection{Distributivity}

To add support for branching, we need to equip our category with a
\textit{distributive coproduct structure}, as follows:
\begin{definition}[Distributive (pre)monoidal category] 
  A \textbf{distributive (pre)monoidal category} is a (pre)monoidal category
  with all finite coproducts equipped with a natural family of central
  isomorphisms (the \textbf{distributor}) \(\delta_{A, B, C}: (A + B) \otimes C
  \simeq A \otimes C + B \otimes C\). This isomorphism is subject to rather
  complex coherence conditions, worked out in full for the more general case of
  \textit{rig categories} in \cite{laplaza-distributivity}, but for our purposes
  we will only make use of the following:
  \begin{equation*}
    \delta_{A, B, C}^{-1} = [\iota_0 \otimes C, \iota_1 \otimes C] 
    \qquad
    \sigma^+_{A, B} \otimes C;\delta_{B, A, C} = \delta_{A, B, C};\sigma^+_{A \otimes C, B \otimes C}
  \end{equation*}
  Note in particular that this implies that \(\iota_i \otimes C;\delta_{A, B, C}
  = \iota_i\). A substructural (pre)monodial category is \textbf{distributive}
  if it is distributive as a (pre)monoidal category, and its distributor is
  \textit{pure}. A substructural effectful category \(\mc{C}_1 \to \mc{C}_0\) is
  \textbf{distributive} if \(\mc{C}_1, \mc{C}_0\) are distributive (as a
  substructural (pre)monoidal category) and \(\upg{\cdot}{}\) strictly preserves
  all coproducts.
\end{definition}
Many (pre)monoidal categories are distributive, including of course
\(\ms{Set}\), but also categories which can be derived from monads on more
primitive categories such as \(\ms{PFun}\) and \(\ms{Rel}\), as can be deduced
from the following proposition:
\begin{proposition}
  If \(\mc{C}\) is a distributive substructural monoidal category and \(\mb{T}\)
  is a strong monad on \(\mc{C}\), then the Kleisli category \(\mc{C}_{\mb{T}}\)
  is a distributive substructural effectful category (with affine and relevant
  objects and morphisms inherited from \(\mc{C}\))
\end{proposition}
\begin{proof}
  Let \(\delta_{A, B, C}: \mc{C}((A + B) \otimes C, A \otimes C + B \otimes C)\) denote \(\mc{C}\)'s distributor. We will write composition in \(\mc{C}\) as "\(;\)" and composition in the Kleisli category \(\mb{T}\) as "\(\gg\)".
  We will show that
  \(\upg{\delta_{A, B, C}}{}: \mc{C}_{\mb{T}}((A + B) \otimes C, A \otimes C + B \otimes C)\) satisfies the desired properties of a distributor. In particular, it is:
  \begin{itemize}
    \item Central: this follows immediately from being the image of a pure morphism
    \item Affine: \(\upg{\delta}{} \gg \upg{\ms{drop}}{} = \upg{(\delta;\ms{drop})}{} = \upg{\ms{drop}}{}\)
    \item Relevant: \(\upg{\delta}{} \gg \upg{\ms{split}}{} = \upg{(\delta;\ms{split})}{} = \upg{(\ms{split};\delta \otimes \delta)}{} = \upg{\ms{split}}{} \gg \upg{\delta}{} \otimes \upg{\delta}{}\)
    \item Left inverse: \(\upg{\delta}{} \gg \upg{[\iota_0 \otimes C, \iota_1 \otimes C]}{} = \upg{\delta;[\iota_0 \otimes C, \iota_1 \otimes C]}{} = \ms{id}\)
    \item Right inverse: \(\upg{[\iota_0 \otimes C, \iota_1 \otimes C]}{} \gg \upg{\delta}{} = \upg{[\iota_0 \otimes C, \iota_1 \otimes C];\delta}{} = \ms{id}\)
    \item Symmetric: \(\upg{\sigma^+}{} \otimes C \gg \upg{\delta}{} = \upg{\sigma^+ \otimes C;\delta}{} = \upg{\delta;\sigma^+}{} = \upg{\delta}{} \gg \upg{\sigma^+}{}\)
  \end{itemize}
\end{proof}

\TODO{text}

\subsection{Elgot Categories}

The structure so far is enough to support non-looping control flow, but to allow
general, potentially nonterminating control flow, we need a way to reason about
iteration. To do so, we introduce the notion of an \textit{Elgot category} as
follows:
\begin{definition}[Elgot category] 
  An \textbf{Elgot category} is a category \(\mc{C}\) with coproducts equipped
  with an operator \(\cdot^\dagger: \mc{C}(A, B + A) \to \mc{C}(A, B)\)
  satisfying the following properties:
  \begin{itemize}
    \item \textbf{Fixpoint:} \(f;[\ms{id}, f^\dagger] = f^\dagger\)
    \item \textbf{Naturality:} \((f;g + \ms{id})^\dagger = f^\dagger;g\)
    \item \textbf{Codiagonal:} \(\forall f \in \mc{C}(A, (B + A) + A), (f^\dagger)^\dagger = (f;[\ms{id}, \iota_1])^\dagger\)
    \item \textbf{Uniformity:}
    \(
      h;f = g;\ms{id} + h \implies h;f^\dagger = g^\dagger
    \)
  \end{itemize}
  A monad \(\mb{T}\) on \(\mc{C}\) such that its Kleisli cateogry
  \(\mc{C}_{\mb{T}}\) is Elgot is called an \textbf{Elgot monad}.
\end{definition}
Some important properties which hold for all Elgot categories include:
\begin{itemize}
  \item \textbf{Dinaturality:} \((g;[\iota_0, h])^\dagger = g;[\ms{id},
  (h;[\iota_0, g])^\dagger]\)
  \item \textbf{Squaring:} \((f;[\iota_0, f])^\dagger = f^\dagger\)
\end{itemize}
Proofs can be found in Lemma 31 of \cite{goncharov-squaring}.

One important way of generating Elgot categories is through monads: a monad
whose Kliesli category is Elgot is called an \textbf{Elgot monad}. Examples of
Elgot monads include \(\ms{Option}\) (where \(\ms{None}\) is interpreted as
nontermination), the powerset monad (where the empty set is interpreted as
nontermination), and the trace monads we will explore in Section
\ref{ssec:trace-monads}.

It turns out that various monad transformers preserve Elgot structure, allowing
us to build up Elgot monads in a composable fashion. In particular,
\begin{proposition}
  If \(\mb{T}\) is an Elgot monad on \(\ms{Set}\), then \(\ms{StateT}\;S\;\mb{T}
  = \lambda A. S \to \ms{T}(A \times S)\), \(\ms{ReaderT}\;\mb{T}\;S = \lambda
  A. S \to \ms{T}(A)\), and \(\ms{WriterT}\;\mb{T}\;S = \lambda A. \ms{T}(A
  \times S)\) are all Elgot, with iteration operators lifted from \(\mb{T}\) in
  the obvious manner.
\end{proposition}

\subsection{Denotational Semantics}

We're now ready to give a semantics for an \isotopessa dialect with a set of
types \(\mc{T}\) and instructions \(\mc{I}\). We'll need:
\begin{itemize}
  \item A \textit{substructural effectful category} \(\mc{C} = \mc{C}_1 \to
  \mc{C}_0\) such that \(\mc{C}_0\) is \textit{distributive} and \textit{Elgot},
  equipped with
  \item A function \(\ms{base}(X)\) from base types \(X \in \mc{T}\) to objects
  in \(|\mc{C}|\), such that \(\ms{a} \in \ms{lin}(X) \implies \ms{base}(X) \in
  \ms{Aff}(\mc{C})\) and \(\ms{r} \in \ms{lin}(X) \implies \ms{base}(X) \in
  \ms{Rel}(\mc{C})\)
  \item A function \(\ms{inst}(f)\) from instructions \(f \in \mc{I}_p^q(A, B)\)
  to morphisms in \(\mc{C}_p^q(\dnt{A}, \dnt{B})\)
\end{itemize}

\TODO{We know \textit{what} these semantics are, unlike below, but maybe clarify
\textit{why}} 

We begin by giving a semantics for types and contexts in terms of objects in
\(|\mc{C}|\) in Figure \ref{fig:ssa-type-semantics}: we map tensor products to
tensor products, the unit type to the monoidal unit, and booleans to \(\mb{1} +
\mb{1}\). Contexts are simply interpreted as the tensor products of the variable
types which appear in them; in particular, the quantity annotations have
\textit{no} effect on a context's semantics. We then give a semantics for
structural judgements in Figure \ref{fig:ssa-structural-semantics}. In
particular, we begin by interpreting the judgements that a type is affine (i.e.
\(\tlin{A}{\taff}\)) or relevant (i.e. \(\tlin{A}{\trel}\)) as the drop morphism
\(\ms{drop}(\dnt{A}): \mc{C}(\dnt{A}, I)\) and split morphism
\(\ms{drop}(\dnt{A}): \mc{C}(\dnt{A}, \dnt{A} \otimes \dnt{A})\) respectively.
We can then interpret a context split \(\csplits{\Gamma}{\Delta}{\Xi}\) as a
pure morphism \(\mc{C}^\tint_1(\dnt{\Gamma}, \dnt{\Delta} \otimes \dnt{\Xi})\)
which partitions the variables as appropriate, splitting and dropping as
necessary. We interpret a context weakening \(\cwk{\Gamma}{\Delta}\) as the
associated split \(\csplits{\Gamma}{\Delta}{\cdot}\) postcomposed with a unitor.
Finally, we interpret a label-context weakening \(\lwk{\ms{L}}{\ms{K}}\) as a
pure morphism \(\mc{C}^\tint_1(\dnt{\ms{L}}, \dnt{\ms{K}})\), where
\rle{join-nil} is the identity, \rle{join-cons} is the sum of
\(\dnt{\lwk{\ms{L}}{\ms{K}}}\) and \(\dnt{\cwk{\Gamma}{\Delta}}\), and
\rle{join-zero} is \(\dnt{\lwk{\ms{L}}{\ms{K}}}\) postcomposed with the left
injection.

\begin{figure}
  \begin{gather*}
    \boxed{\dnt{A}}: |\mc{C}| \\
    \dnt{X} = \ms{base}(X) 
    \qquad
    \dnt{\mb{1}} = \mb{1}
    \qquad
    \dnt{\mb{2}} = \mb{2} = \mb{1} + \mb{1}
    \qquad \dnt{A \otimes B} = \dnt{A} \otimes \dnt{B} \\
    \boxed{\dnt{\Gamma}}: |\mc{C}| \\
    \dnt{\cdot} = \mb{1} \qquad 
    \dnt{\Gamma, \thyp{x}{A}{q}} = \dnt{\Gamma} \otimes \dnt{A} \\
    \boxed{\dnt{\ms{L}}}: |\mc{C}| \\
    \dnt{\cdot} = \mb{0} \qquad
    \dnt{\ms{L}, \llhyp{\ell}{\Gamma}{A}} = 
      \dnt{\ms{L}} + \dnt{\Gamma} \otimes \dnt{A}
  \end{gather*}
  \caption{Semantics for \isotopessa types and contexts}
  \Description{Semantics for isotope-SSA types and contexts}
  \label{fig:ssa-type-semantics}
\end{figure}

\begin{figure}
  \begin{gather*}
    \boxed{\dnt{\tlin{A}{\taff}: \mc{C}_1^\tint(\dnt{A}, \mb{1})}}
      \\
      \dnt{\tlin{A}{\taff}} = \ms{drop}(\dnt{A})
      % \dnt{\tlin{X}{\taff}} = \ms{drop}(X) \qquad
      % \dnt{\tlin{A \otimes B}{\taff}} 
      %   = \dnt{\tlin{A}{\taff}} \otimes \dnt{\tlin{B}{\taff}}
      %   ; \lambda 
      \\
    \boxed{\dnt{\tlin{A}{\trel}}
      : \mc{C}_1^\tint(\dnt{A}, \dnt{A} \otimes \dnt{A})}
      \\
      \dnt{\tlin{A}{\trel}} = \ms{split}(\dnt{A})
      % \dnt{\tlin{X}{\trel}} = \ms{split}(X) \qquad
      % \dnt{\tlin{A \otimes B}{\trel}} = 
      %   \dnt{\tlin{A}{\trel}} \otimes \dnt{\tlin{B}{\trel}}
      %   ;\alpha;\dnt{A} \otimes \sigma \otimes \dnt{B};\alpha 
      \\
    \boxed{\dnt{\csplits{\Gamma}{\Delta}{\Xi}}
      : \mc{C}_1^\tint(\dnt{\Gamma}, \dnt{\Delta} \otimes \dnt{\Xi})}
      \\
    \dnt{\csplits{\cdot}{\cdot}{\cdot}} = \lambda^{-1} \qquad
    \dnt{\csplits
      {\Gamma, \thyp{x}{A}{q}}
      {\Delta, \thyp{x}{A}{q}}
      {\Xi}} 
      = \dnt{\csplits{\Gamma}{\Delta}{\Xi}} \otimes \dnt{A};\alpha;\dnt{\Delta} \otimes \sigma;\alpha 
      \\
    \dnt{\csplits
      {\Gamma, \thyp{x}{A}{q}}
      {\Delta}
      {\Xi, \thyp{x}{A}{q}}} 
      = \dnt{\csplits{\Gamma}{\Delta}{\Xi}} \otimes \dnt{A};\alpha 
      \\
    \dnt{\csplits
      {\Gamma, \thyp{x}{A}{q}}
      {\Delta}
      {\Xi, \thyp{x}{A}{q}}} \
      = \dnt{\Gamma} \otimes \dnt{\tlin{A}{\trel}};
        \alpha;
        \dnt{\Delta} \otimes \sigma \otimes \dnt{A};
        \alpha 
        \\
    \dnt{\csplits
      {\Gamma, \thyp{x}{A}{q}}
      {\Delta}
      {\Xi}}
      = \dnt{\csplits{\Gamma}{\Delta}{\Xi}}
        \otimes \dnt{\tlin{A}{\taff}}
      ; \rho 
      \\
    \boxed{\dnt{\cwk{\Gamma}{\Delta}}
      : \mc{C}_1^\tint(\dnt{\Gamma}, \dnt{\Delta})} 
      \\
      \dnt{\cwk{\Gamma}{\Delta}} 
      = \dnt{\csplits{\Gamma}{\Delta}{\cdot}};\rho 
      \\
    \boxed{\dnt{\lwk{\ms{L}}{\ms{K}}}
      : \mc{C}_1^\tint(\dnt{\ms{L}}, \dnt{\ms{K}})} 
      \\
    \dnt{\lwk{\cdot}{\cdot}} = \ms{id} \qquad
    \dnt{\lwk
      {\ms{L}, \llhyp{\ell}{\Gamma}{A}}
      {\ms{K}, \llhyp{\ell}{\Delta}{A}}}
      = \dnt{\lwk{\ms{L}}{\ms{K}}} + \upg{(\dnt{\cwk{\Gamma}{\Delta}} \otimes \dnt{A})}{} 
      \\
    \dnt{\lwk{\ms{L}}{\ms{K}, \llhyp{\ell}{\Gamma}{A}}}
      = \dnt{\lwk{\ms{L}}{\ms{K}}};\iota_0
  \end{gather*}
  \caption{Semantics for \isotopessa structural judgements}
  \Description{Semantics for isotope-SSA structural judgements}
  \label{fig:ssa-structural-semantics}
\end{figure}

\begin{figure}
  \begin{gather*}
    \boxed{\dnt{\hasty{\Gamma}{p}{a}{A}{q}}
      : \mc{C}_p^q(\dnt{\Gamma}, \dnt{A})} 
      \\
    \dnt{\hasty{\Gamma}{p}{x}{A}{q}} 
      = \dnt{\cwk{\Gamma}{\thyp{x}{A}{q}}}
      \qquad
    \dnt{\hasty{\Gamma}{p}{f\;a}{B}{q}}
      = \upg{\dnt{\hasty{\Gamma}{1}{a}{A}{q}}}{p}
      ; \ms{inst}_p(f) 
      \\
    \dnt{\hasty{\Gamma}{p}{(a, b)}{A \otimes B}{q}}
      = \upg{(
        \dnt{\csplits{\Gamma}{\Delta}{\Xi}};
        \dnt{\hasty{\Delta}{1}{a}{A}{q}} \otimes
        \dnt{\hasty{\Xi}{1}{b}{B}{q}}
      )}{p} 
      \\
    \dnt{\hasty{\Gamma}{p}{()}{\mb{1}}{q}}
      = \upg{\dnt{\cwk{\Gamma}{\cdot}}}{p} 
      \\
    \dnt{\hasty{\Gamma}{p}{\ctt}{\mb{2}}{q}}
      = \upg{(\dnt{\cwk{\Gamma}{\cdot}};\ctt)}{p}
      \qquad
    \dnt{\hasty{\Gamma}{p}{\cff}{\mb{2}}{q}}
      = \upg{(\dnt{\cwk{\Gamma}{\cdot}};\cff)}{p} 
      \\
    \dnt{\hasty{\Gamma}{p}{\letexpr{x}{a}{e}}{B}{q}}
      = \upg{(
        \dnt{\csplits{\Gamma}{\Delta}{\Xi}}
        ; \dnt{\Delta} \otimes \dnt{\hasty{\Xi}{1}{a}{A}{q}}
      )}{p};\dnt{\hasty{\Delta, \thyp{x}{A}{}}{p}{e}{B}{q}} 
      \\
    \dnt{\hasty{\Gamma}{p}{\letexpr{(x, y)}{a}{e}}{C}{q}}
      = \upg{(
        \dnt{\csplits{\Gamma}{\Delta}{\Xi}}
        ; \dnt{\Delta} \otimes \dnt{\hasty{\Xi}{1}{a}{A \otimes B}{q}}
      )}{p}
      ; 
      \\ \qquad \qquad \qquad \qquad \qquad \alpha
      ; \dnt{\hasty{\Delta, \thyp{x}{A}{}, \thyp{y}{B}{}}{p}{e}{C}{q}}
    %   \\
    % \dnt{\hasty{\Gamma}{0}{\lbsplice{\ell}{A}{t}}{A}{\varnothing}}
    %   = \dnt{\haslb{\Gamma}{t}{\llhyp{\ell}{\cdot}{A}}};\alpha_+
  \end{gather*}
  \caption{Semantics for \isotopessa expressions}
  \Description{Semantics for isotope-SSA expressions}
  \label{fig:ssa-term-semantics}
\end{figure}

\TODO{\textit{what} is the semantics of a term}

We can now give a semantics for terms in Figure \ref{fig:ssa-term-semantics}.
Our treatment of \rle{var} is quite standard: the expression \(x\) types in
\(\Gamma\) if all other variables in \(\Gamma\) and can be discarded, in which
case the semantics is simply doing so, i.e.
\(\dnt{\cwk{\Gamma}{\thyp{x}{A}{q}}}\), lifting to \(\mc{C}_0\) as necessary.
Similarly, to type a function application \rle{var}, we first take
\(\dnt{\hasty{\Gamma}{1}{a}{A}{q}}\), which is guarnateed to be central, lift it
to the appropriate category \(\mc{C}_p\), and then simply postcompose it with
the denotation of \(f\). \rle{pair} is handled by composing the context split
\(\csplits{\Gamma}{\Delta}{\Xi}\) taking the tensor product of the components
\(\dnt{\hasty{\Delta}{1}{a}{A}{q}}\), \(\dnt{\hasty{\Xi}{1}{b}{B}{q}}\); this is
well-defined since both components are central and \(\mc{C}_1\) is monoidal. We
then lift the result to the appropriate \(\mc{C}_p\). The semantics of
\rle{unit} is simply dropping the entire context; \(\dnt{\cwk{\Gamma}{\cdot}}\),
while \rle{true}, and \rle{false} are each handled by first dropping the context
and then postcomposing with the appropriate constant; in all cases, this is
followed by lifting to \(\mc{C}_p\). Finally, \rle{let} and \rle{let2} are
handled by first splitting the context, then passing the right hand component to
the semantics of the bound expression \(\dnt{\hasty{\Xi}{1}{a}{A}{q}}\) or
\(\dnt{\hasty{\Xi}{1}{a}{A \otimes B}{q}}\), and then, lifting to \(\mc{C}_p\)
as necessary, after reassociating with \(\alpha\), passing everything to the
inner expression \(\dnt{\hasty{\Delta, \thyp{x}{A}{q}}{p}{e}{B}{q}}\) or
\(\dnt{\hasty{\Delta, \thyp{x}{A}{q}, \thyp{y}{B}{q}}{p}{e}{C}{q}}\). Note that
this inner expression is not necessarily central, unlike the bound expression!

\begin{figure}
  \begin{gather*}
    \boxed{\dnt{\haslb{\Gamma}{t}{\ms{L}}}
      : \mc{C}_0^\varnothing(\dnt{\Gamma}, \dnt{\ms{L}})} \\
    \dnt{\haslb{\Gamma}{\lbrb{\ell}{a}}{\ms{L}}}
      = \upg{(\dnt{\csplits{\Gamma}{\Delta}{\Xi}}
      ; \dnt{\Delta} \otimes \dnt{\hasty{\Xi}{1}{a}{A}{\tint}})}
      ; \dnt{\lwk{\llhyp{\ell}{\Delta}{A}}{\ms{L}}} \\
    \dnt{\haslb{\Gamma}{\ite{e}{s}{t}}{\ms{L}}}
      = \upg{(\dnt{\csplits{\Gamma}{\Delta}{\Xi}}
      ; \dnt{\hasty{\Delta}{1}{e}{\mb{2}}{\tint}} \otimes \dnt{\Xi})}
      ; \ms{ite}_{\dnt{\Xi}} ;
      \\ \qquad \qquad \qquad \qquad
      [
        \dnt{\haslb{\Xi}{s}{\ms{L}}}, 
        \dnt{\haslb{\Xi}{t}{\ms{L}}} 
      ]
      \\
    \dnt{\haslb{\Gamma}{\letstmt{x}{a}{t}}{\ms{L}}}
      = \upg{\dnt{\csplits{\Gamma}{\Delta}{\Xi}}}
      ; \dnt{\Delta} \otimes \dnt{\hasty{\Gamma}{0}{a}{A}{q}}
      ; \dnt{\haslb{\Delta, \thyp{x}{A}{}}{t}{\ms{L}}}
      \\
    \dnt{\haslb{\Gamma}{\letstmt{(x, y)}{a}{t}}{\ms{L}}}
      = \upg{\dnt{\csplits{\Gamma}{\Delta}{\Xi}}}
      ; \dnt{\Delta} \otimes \dnt{\hasty{\Gamma}{0}{a}{A \otimes B}{q}}
      ; \alpha
      ; \dnt{\haslb{\Delta, \thyp{x}{A}{}, \thyp{y}{B}{}}{t}{\ms{L}}}
      \\
    \dnt{\haslb{\Gamma}{\ewhere{t}{L}}{\ms{K}}}
      = \dnt{\haslb{\Gamma}{t}{\ms{L}}}
      ; \dnt{\lhaslb{\ms{L}}{L}{\ms{K}}}^\dagger
      \\
    \boxed{\dnt{\lhaslb{\ms{L}}{L}{\ms{K}}}
      : \mc{C}_0^\varnothing(\dnt{\ms{L}}, \dnt{\ms{K}} + \dnt{\ms{L}})}
      \\
    \dnt{\lhaslb{\ms{L}}{L, \lwbranch{\ell}{x: A}{t}}{\ms{K}}}
      = \dnt{\lhaslb{\ms{L}}{L}{\ms{K}, \llhyp{\ell}{\Gamma}{A}}}
      ; [[\iota_0, \dnt{\haslb{\Gamma, \thyp{x}{A}{}}{t}{\ms{L}}}], \iota_1]
      \\
    \dnt{\lhaslb{\ms{L}}{\cdot}{\ms{L}}}
      = \iota_0
  \end{gather*}
  \caption{Semantics for \isotopessa blocks}
  \Description{Semantics for isotope-SSA blocks}
  \label{fig:ssa-block-semantics}
\end{figure}

\TODO{\textit{what} is the semantics of a block}

We can now give a semantics for blocks in Figure \ref{fig:ssa-block-semantics}.
Our semantics for break statements \(\lbrb{\ell}{a}\) is given by the semantics
for splitting the context into variables which pass into dominated blocks and
variables used to compute the argument, \(\dnt{\csplits{\Gamma}{\Delta}{\Xi}}\),
with the latter passed into the semantics of the argument (which is pure, and
therefore must be lifted to \(\mc{C}_0\))
\(\dnt{\hasty{\Gamma}{1}{a}{A}{\tint}}\), with the result along with the
preserved variables passed into \(\lwk{\llhyp{\ell}{\Delta}{A}}{\ms{L}}\) to
obtain the desired target. The semantics of if-statements are given by first
splitting the context via \(\dnt{\csplits{\Gamma}{\Delta}{\Xi}}\) and then
calculating the discriminant \(\dnt{\hasty{\Delta}{1}{b}{\mb{2}}{\tint}}\),
which must be pure and therefore needs to be lifted. This is then passed through
the distributor \(\delta\), with each branch passed to the semantics of its
corresponding block \(\dnt{\haslb{\Xi}{s}{\ms{L}}}\),
\(\dnt{\haslb{\Xi}{t}{\ms{L}}}\). The semantics of \rle{let-blk} and
\rle{let2-blk} are analogous to those of \rle{let} and \rle{let2} in the
expression case, except that now the bound expression \(a\) is allowed to be
impure (and therefore in particular does not need to be upgraded).

Finally, we must give semantics for \rle{where}. To do so, we begin by giving a
semantics for control-flow graphs: \rle{nil-br} simply passes the input context
\(\ms{L}\) through unchanced, while \rle{cons-br} takes any calls to
\(\lbl{\ell}\) and passes them to the denotation
\(\dnt{\haslb{\Gamma}{t}{\ms{L}}}\), passing the rest of \(\ms{L}\) to
\(\dnt{\lhaslb{\ms{L}}{L}{\ms{K}}}\). We can now quite easily state the
semantics of \rle{where}: we simply precompose the semantics of the inner block
\(\dnt{\haslb{\Gamma}{t}{\ms{L}}}\) with the semantics of the associated
control-flow graph \(\dnt{\lhaslb{\ms{L}}{L}{\ms{K}}}\).

\TODO{explain what the above actually means, \textit{what} is the semantics of a
CFG}


\subsection{Semantic Metatheory}

\TODO{some text about semantic metatheory}

\begin{lemma}[Coherence] \
  \begin{itemize}
    \item Given any two derivations \(D_1: \hasty{\Gamma}{p}{a}{A}{q}\), \(D_2: \hasty{\Gamma}{p}{a}{A}{r}\), \(\dnt{D_1} = \dnt{D_2}\)
    \item Given any two derivations \(D_1, D_2: \haslb{\Gamma}{t}{\ms{L}}\), \(\dnt{D_1} = \dnt{D_2}\)
    \item Given any two derivations \(D_1, D_2: \csplits{\Gamma}{\Delta}{\Xi}\), \(\dnt{D_1} = \dnt{D_2}\)
    \item Given any two derivations \(D_1, D_2: \cwk{\Gamma}{\Delta}\), \(\dnt{D_1} = \dnt{D_2}\)
    \item Given any two derivations \(D_1, D_2: \lwk{\ms{L}}{\ms{K}}\), \(\dnt{D_1} = \dnt{D_2}\)
  \end{itemize}
\end{lemma}

\TODO{some text about coherence, implicit use above}

\begin{lemma}[Semantic Upgrade]
  For all \(\Gamma, a, A\), if \(\hasty{\Gamma}{1}{a}{A}{q}\), then
  \[\dnt{\hasty{\Gamma}{0}{a}{A}{q}} = \upg{\dnt{\hasty{\Gamma}{1}{a}{A}{q}}}{}\]
\end{lemma}

\TODO{some text about upgrade}

\begin{lemma}[Semantic Weakening] 
  If \(\cwk{\Gamma}{\Delta}\), then 
  \begin{itemize}
    \item For all \(\hasty{\Delta}{p}{a}{A}{q}\),
    \[
      \upg{\dnt{\cwk{\Gamma}{\Delta}}}{p}
      ; \dnt{\hasty{\Delta}{p}{a}{A}{q}}
      = \dnt{\hasty{\Gamma}{p}{a}{A}{q}} 
    \]
    \item For all \(\haslb{\Delta}{t}{\ms{L}}\),
    \[
      \upg{\dnt{\cwk{\Gamma}{\Delta}}}{p}
      ; \dnt{\haslb{\Delta}{t}{\ms{L}}}
      = \dnt{\haslb{\Gamma}{t}{\ms{L}}}  
    \]
  \end{itemize}

  Similarly, if \(\lwk{\ms{L}}{\ms{K}}\) and \(\haslb{\Gamma}{t}{\ms{L}}\), then
  \[
    \dnt{\haslb{\Gamma}{t}{\ms{L}}}
    ; \dnt{\lwk{\ms{L}}{\ms{K}}}
    = \dnt{\haslb{\Gamma}{t}{\ms{K}}}
  \]
\end{lemma}

\TODO{some text about semantic weakening}

\begin{figure}
  \begin{gather*}
    \boxed{\dnt{\issubst{\gamma}{\Theta}{\Gamma}}: \mc{C}_1^\varnothing(\dnt{\Theta}, \dnt{\Gamma})} 
    \\
    \dnt{\issubst{\gamma}{\Theta}{\Gamma, \thyp{x}{A}{q}}}
      = \dnt{\csplits{\Theta}{\Theta_\Gamma}{\Theta_x}};\dnt{\issubst{\gamma}{\Theta_\Gamma}{\Gamma}} \otimes \dnt{\hasty{\Theta_x}{1}{a}{A}{q}} \\
    \dnt{\issubst{\cdot}{\cdot}{\cdot}} = \ms{id}
    \\
    \boxed{\dnt{\lbsubst{\mc{L}}{\ms{L}}{\ms{K}}}: \mc{C}_0^\varnothing(\dnt{\ms{L}}, \dnt{\ms{K}})} 
  \end{gather*}
  \caption{Semantics for \isotopessa substitutions}
  \Description{Semantics for isotope-SSA substitutions}
  \label{fig:ssa-subst-semantics}
\end{figure}

\TODO{some text about substitution semantics}

\begin{theorem}[Semantic Substitution] 
  Given \(\issubst{\gamma}{\Theta}{\Gamma}\),
  \begin{itemize}
    \item For all \(\hasty{\Gamma}{p}{a}{A}{q}\), 
    \[
      \upg{\dnt{\issubst{\gamma}{\Theta}{\Gamma}}}{p}
      ;\dnt{\hasty{\Gamma}{p}{a}{A}{q}} 
      = \dnt{\hasty{\Theta}{p}{[\gamma]a}{A}{q}}
    \]
    \item For all 
      \(\haslb{\Gamma}{t}{\ms{L}}\), 
      \(\lbsubst{\substlbs{\gamma}{\ms{L}}}{\ms{K}}{\ms{L}}\), 
      \(\haslb{\Theta}{[\gamma]t}{\ms{K}}\),
    \[
      \upg{\dnt{\issubst{\gamma}{\Theta}{\Gamma}}}{}
      ; \dnt{\haslb{\Gamma}{t}{\ms{L}}}
      = \dnt{\haslb{\Theta}{[\gamma]t}{\ms{K}}} 
      ; \dnt{\lbsubst{\substlbs{\gamma}{\ms{K}}}{\ms{K}}{\ms{L}}}
    \]
    \item For all 
      \(\lhaslb{\ms{W}}{L}{\ms{L}}\), 
      \(\lbsubst{\substlbs{\gamma}{\ms{L}}}{\ms{K}}{\ms{L}}\),
      \(\lbsubst{\substlbs{\gamma}{\ms{W'}}}{\ms{W}}{\ms{W'}}\),
      \[
        \dnt{\lbsubst{\substlbs{\gamma}{\ms{W'}}}{\ms{W}}{\ms{W'}}}
        ; \dnt{\lhaslb{\ms{W'}}{[\gamma]L}{\ms{K}}}
        = \dnt{\lhaslb{\ms{W}}{L}{\ms{L}}}
        ; \dnt{\lbsubst{\substlbs{\gamma}{\ms{L}}}{\ms{K}}{\ms{L}}}
        + \dnt{\lbsubst{\substlbs{\gamma}{\ms{W'}}}{\ms{W}}{\ms{W'}}}
      \]
  \end{itemize}
  Similarly, given \(\lbsubst{\mc{L}}{\ms{L}}{\ms{K}}\),
  \begin{itemize} 
    \item For all \(\haslb{\Gamma}{t}{\ms{L}}\), we have
    \[
      \dnt{\haslb{\Gamma}{t}{\ms{L}}};\dnt{\lbsubst{\mc{L}}{\ms{L}}{\ms{K}}}
      = \dnt{\haslb{\Gamma}{[\mc{L}]t}{\ms{K}}}
    \]
    \item For all \(\lhaslb{\ms{W}}{L}{\ms{L}}\), we have
    \[
      \dnt{\lhaslb{\ms{W}}{L}{\ms{L}}};\dnt{\lbsubst{\mc{L}}{\ms{L}}{\ms{K}}}
      = \dnt{\lhaslb{\ms{W}}{[\mc{L}]L}{\ms{L}}}
    \]
  \end{itemize}
\end{theorem}

\TODO{some text about semantic substitution}

% \subsection{Normalization}

% \label{ssec:ssa-norm}

% \TODO{this}

\subsection{Optimizations}

\TODO{this:}

\TODO{substitution \(\implies\): loop hoisting, global value numbering, some dead code elimination}

\TODO{rewriting \(\implies\) strength reduction, E-graph optimization, see \cite{cranelift}}

\TODO{inlining}

\TODO{loop fusion}

\TODO{unreachable code elimination}

\TODO{more dead code elimination}

\section{Concrete Models}

In this section, we will build on the bare-bones models given in Section
\ref{sec:semantics} to develop more sophisticated models capable of underpinning
\isotopessa dialects supporting features such as weak memory and separation
logic.

\subsection{Trace Models}

\label{ssec:trace-monads}

We will begin by giving a useful foundation for a wide class of \isotopessa
models: the "trace models" over \(\mb{Set}\), which accumulate a (potentially
infinite) nondeterministic trace of monoidal "events". These will later come in
handy for reasoning about output behaviour and weak memory.

We begin with some definitions:
\begin{definition}[Stream Action]
  A \textbf{stream action} of a monoid \(M\) on a set \(I\) equipped with a
  monoid action \(\cdot: M \times I \to I\) is a function \(\Theta: M^\omega \to
  I\) mapping (infinite) streams of \(M\) to elements of \(I\) satisfying
  \(
    m \cdot \Theta(\ell) = \Theta(m \colon \ell)
  \),
  where \(m \colon \ell\) denotes the stream formed by prefixing \(m\) to the
  stream \(\ell\).
\end{definition}
Some basic examples include:
\begin{enumerate}
  \item For any monoid \(M\), the unique map \(\Theta: M^\omega \to \mathbf{1}\)
  forms a stream action on the one element set \(\mathbf{1}\).
  \item Given a monoid \(M\) acting on sets \(I_i\) with action \(\cdot_i\) and
  stream actions \(\Theta_i: M^\omega \to I_i\), \(\Theta =
  \langle\Theta_i\rangle_i: M^\omega \to \Pi_iI_i\) is a stream action
  compatible with the action \(m \cdot l = (m \cdot_i l_i)_i\).
  \item In particular, given monoids \(M_i\) acting on sets \(I_i\) with action
  \(\cdot_i\) and stream actions \(\Theta_i: M_i^\omega \to I_i\), \(\Theta =
  \langle \Theta_i \circ \ms{fmap}\;\pi_i\rangle_i: (\Pi_iM_i)^\omega \to
  \Pi_iI_i\) is a stream action compatible with the action \(m \cdot l = (m_i
  \cdot_i l_i)_i\).
  \item The concatenation map \(\Theta: (A^*)^\omega \to A^\omega\) is a stream
  action of \(A^*\), the set of (finite) lists of \(A\), on \(A^{\leq \omega}\),
  the set of (\textit{potentially} infinite) sequences of \(A\), where \(\ell
  \cdot s\) prepends the (finite) list \(\ell\) to the (potentially infinite)
  sequence \(s\).
  \item The sum map \(\Sigma: \nats^\omega \to \nats \cup \{\infty\}\) is a
  stream action of \(\nats\) on \(\nats \cup \{\infty\}\), where \(\nats\) acts
  on \(\nats \cup \{\infty\}\) by addition.
\end{enumerate}
We define the \textbf{stream center} of \(M\) with respect to \(I\), denoted
\(\mc{SZ}(M, I)\), as the subset of the center of \(M\), \(Z(M)\), consisting of
those elements \(m \in Z(M)\) such that for all \(i \in I\), \(m \cdot i = i\).
For instance, \(\mc{SZ}(M, \mathbf{1}) = Z(M)\), and \(\mc{SZ}(A^*, A^{\leq
\omega}) = Z(A^*) = \{[\,]\}\), while \(\mc{SZ}(\nats, \nats \cup \{\infty\}) =
\{0\} \subseteq Z(\nats) = \nats\). Generally, \(\mc{SZ}(M, I)\) always contains
the unit of \(M\).

We can now define the \textit{trace monad} of a stream action as follows:
\begin{definition}[Trace Monad]
  Given a stream action \(\Theta\) of \(M\) on \(I\), we can define the
  \textbf{nondeterministic trace monad} \(\ms{Traces}\;\Theta\;A = \mc{P}^+(A
  \times M + I)\), with \(\eta_A = \lambda a.\iota_0 (a, 1)\) and 
  \[
    \mu_A\;T = \{[\lambda (a, m). m \cdot a, \ms{id}](t) \mid t \in T\}
  \]
  where the action of \(M\) on traces is given by \(m \cdot \iota_0 (a, m') =
  \iota_0 (a, mm')\) and \(m \cdot \iota_1 t = \iota_1 (m \cdot t)\). This is an
  Elgot monad, with iteration operator \(f^\dagger(a) = f^\infty(a) \cup
  \bigcup_{n \in \nats}\{\iota_0 (\iota_0\;b, m) \in f_n(a)\} \cup \{\iota_1 t
  \in f_n(a)\}\), where
  \[
    f_0 = f, \qquad f_{n + 1} = f \gg [\upg{\iota_0}{}, f_n]
  \]
  \[
    f^\infty(a) = \{\Theta (\lambda n. m_n) \mid a_0 = a \land \forall n, \exists a_{n + 1}. \iota_0 (\iota_1\;a_{n + 1}, m_n) \in f(a_n)\}
  \]
  \TODO{clean this up}
\end{definition}

\TODO{text}

\begin{definition}[Relevant, Affine Monad]
  A monad \(\mb{T}\) on \(\ms{Set}\) is \textbf{relevant} if its Kleisli
  category satisfies \(f;\ms{split} = \ms{split};f \ltimes f = \ms{split};f
  \rtimes f\), where \(\ms{split} = \upg{\lambda a. (a, a)}{}\). \(\mb{T}\) is
  \textbf{affine} if its Kleisli category satisfies \(f;\ms{drop} = \ms{drop}\),
  where \(\ms{drop} = \upg{\lambda a.()}{}\). A monad is \textbf{intuitionistic}
  if it is both relevant and affine.
\end{definition}

\begin{definition}[Submonad] 
  A \textbf{submonad} \(\mb{S}\) of a monad \(\mb{T}\) on \(\ms{Set}\) is a
  collection of sets such that, for all \(A\), \(\mb{S}\;A \subseteq
  \mb{T}\;A\), \(\forall a \in A, \eta\;a \in \mb{S}\;A\), and, for all \(s \in
  \mb{S}\;A\), \(f: A \to \mb{S}\;B\), \(\mb{bind}\;s\;f \in \mb{S}\;A\). We say
  that a submonad of \(\mb{T}\) is \textbf{Elgot} if \(\mb{T}\) is Elgot and,
  for all \(f: A \to \mb{S}(B + A)\), \(f^\dagger: A \to \mb{S}\;B\) where
  \(f^\dagger\) is computed as in \(\mb{T}\). Note that \(\mb{S}\) is also an
  (Elgot) monad in its own right, inheriting \(\mu, \eta\) and \(\cdot^\dagger\)
  from \(\mb{T}\).
\end{definition}
Some important examples of submonads include:
\begin{enumerate}
  \item The \textbf{unit submonad} \(\ms{pure}(\mb{T})\) of every monad
  \(\mb{T}\) with \(\ms{pure}(\mb{T})\;A = \{\eta_A\;a \mid a \in A\}\);
  similarly, \(\mb{T}\) is itself a submonad of \(\mb{T}\).
  \item Given submonads \(\mb{S}_i\) of \(\mb{T}\), their elementwise
  intersections \(\bigwedge_i\mb{S}_i = \lambda A. \bigcap_i \mb{S}_i\;A\) are
  submonads, forming a complete meet-semilattice with bottom element
  \(\ms{pure}(\mb{T})\) and top element \(\mb{T}\); this induces a complete
  lattice in the standard manner.
  \item In particular, the intersection of a relevant/affine/commutative
  submonad and another submonad is relevant/affine/commutative respectively.
  Since every commutative submonad induces a central subcategory of the Kleisli
  category of the parent monad, it follows that a triple of a relevant, affine,
  and commutative submonad make the Kleisli category of a monad \(\mb{T}\) into
  a substructural effectful category with all objects both affine and relevant.
  \item If \(C \subseteq Z(M)\) is submonoid of \(M\), then \(A \mapsto
  \mc{P}^+(A \times C + \mb{0}), \mc{P}^+(A \times C \times I) \subseteq
  \ms{Traces}\;\Theta\;A\) are submonads
  \item In particular, if \(M_C \subseteq Z(M)\) is a central submonoid of \(M\)
  such that \(\forall m \in M_C, \forall i \in I, m \cdot i = i\), then \(A
  \mapsto \mc{P}^+(A \times M_C + \mb{0}) \subseteq \ms{Traces}\;\Theta\;A\) is
  a submonad whose Kleisli category is a central subcategory of the Kleisli
  category of \(\ms{Traces}\;\Theta\;A\). Note \(M_C = \{1\}\) is always such a
  submonoid.
  \label{item:central-submonad}
  \item Similarly, if \(M_R \subseteq M_C\) is a submonoid such that \(\forall m
  \in M_R, m \cdot m = m\), then \(A \mapsto \{S \in \mc{P}^+(A \times M_R + I)
  \mid |S| = 1\} \subseteq \ms{Traces}\;\Theta\;A\) is a \textit{relevant}
  submonad. Note \(M_R = \{1\}\) is always such a submonoid.
  \label{item:relevant-submonad}
  \item Finally, \(A \mapsto \mc{P}^+(A \times \{1\} + \mb{0}) \subseteq
  \ms{Traces}\;\Theta\;A\) is an \textit{affine} submonad.
  \label{item:affine-submonad}
\end{enumerate}
In particular, \ref{item:central-submonad}, \ref{item:relevant-submonad},
\ref{item:affine-submonad} equip the Kleisli category of
\(\ms{Traces}\;\Theta\;A\) with the structure of a substructural effectful
category given choices of \(M_C\), \(M_R\).

\subsection{Heaps, Printing, and Separation Logic}

\label{ssec:separation}

As a concrete example, we can now introduce the \textbf{printing monad}, which
is simply defined as \(\ms{Print}\;A \equiv \ms{Traces}\;\Sigma\), where
\(\Sigma: (\ms{Bytes}^*)^\omega \to \ms{Bytes}^{\leq \omega}\) denotes
concatenation of lists of bytestrings. This can be equipped with the structure
of a substructural effectful category by choosing \(M_C = M_R = \{[]\}\), giving
us an \isotopessa model supporting the instructions \(\ms{print} \in
\mc{I}^\varnothing_0(\ms{Bytes}, \mb{1})\) and \(\ms{rand} \in
\mc{I}^\taff_1(\mb{1}, A)\) for \(A\) nonempty, with standard semantics
\(\dnt{\ms{print}} = \lambda b.\{\iota_0 ((), [b])\}\), \(\dnt{\ms{rand}} =
\lambda (). \{\iota_0 (a, p[]) \mid a \in A\}\).

We can make things a little more interesting by adding a heap to the mix,
defining our monad \(\ms{Comp} = \ms{StateT}\;\ms{Heap}\;\ms{Print}\), where
\(\ms{Heap} = \nats \rightharpoonup \nats\) is simply a partial function. The
Kleisli category of this monad can be equipped with the structure of a
substructural effectful category by simply taking the affine/relevant morphisms
to be the lifts of the affine/relevant morphisms in \(\ms{Set}_{\ms{Print}}\);
as it inherits an Elgot structure from that on \(\ms{Set}_{\ms{Print}}\), we
therefore have an \isotopessa model supporting the instructions \(\ms{set}:
\mc{I}^\varnothing_0(\nats \times \nats, \mb{1})\), \(\ms{get}:
\mc{I}^\varnothing_0(\nats, \nats)\), \(\ms{alloc}: \mc{I}^\varnothing_0(\nats,
\nats)\), and \(\ms{free}: \mc{I}^\varnothing_0(\nats, \mb{1})\). We can assign
semantics to \(\ms{set}\) in the standard fashion, with \(\dnt{\ms{set}} =
\lambda (p, v)\;h. \{\iota_0((), [p \mapsto v]h, [])\}\). \(\ms{get}\) is a bit
more tricky, since it is unclear what to do when we try to access uninitialized
memory: one option is simply to return an arbitrary value, with \(\dnt{\ms{get}}
= \lambda p\;h. \{\iota_0(v, h, []) \mid v = h\;p \lor p \notin h\}\); for now,
this will do. Finally, \(\dnt{\ms{alloc}} = \lambda v\;h.\{\iota_0\;(p, h', [])
\mid h' = h \sqcup p \mapsto v\}\) simply fills a random empty heap cell with
the provided value, returning a pointer to the cell, while \(\dnt{\ms{free}} =
\lambda p\;h. \{\iota_0\;((), h \setminus p, [])\}\)

We can use this construction to, building on the ideas in \cite{mellies-ftrs},
show how a simple separation logic based on \cite{reynolds-separation-2002}
slightly adapted to the categorical setting gives us a very rich \isotopessa
model supporting a basic notion of refinement types, while at the same time
giving us a nontrivial notion of nonlinear types.

Given a set \(A\), we can define a heap-predicate \(\Phi\) over \(A\) to be of
type \(A \to \ms{Heap} \to \ms{Prop}\). Taking as objects pairs \((A, \Phi)\),
we can define the category \(\ms{Sep}\) to have morphisms
\begin{equation}
  \ms{Sep}((A, \Phi), (B, \Psi)) = \{f: A \to \ms{Comp}\;B 
    \mid \forall \phi: \ms{Heap} \to \ms{Prop}, \ms{hoare}\;f\;(\Phi \ast \phi)\;(\Psi \ast \phi)\}
\end{equation}
where
\begin{equation}
  \ms{hoare}\;f\;\Phi\;\Psi = \forall a\;h. \Phi\;a\;h \implies \forall \iota_0\;(b, h') \in f\;a\;h. \Psi\;b\;h'
\end{equation}
is the usual partial correctness triple definition and the separating
conjunction of heap-predicates is defined as usual to be 
\begin{equation}
  \begin{gathered}
  \varphi \ast \psi = \lambda h. \exists h_1, h_2. h = h_1 \sqcup h_2 \land \varphi h_1 \land \psi h_2 
  \qquad
  \varphi \ast \Phi = \lambda a. \varphi \ast \Phi\;a
  \qquad
  \Phi \ast \varphi = \lambda a. \Phi\;a \ast \varphi \\
  \Phi \ast \Psi = \lambda (a, b). \Phi\;a \ast \Psi\;b: A \times B \to \ms{Heap} \to \ms{Prop}
  \end{gathered}
\end{equation}
We can naturally define a tensor product on objects \((A, \Phi) \otimes (B,
\Psi) = (A \times B, \Phi \ast \Psi)\), verifying that, taking
\begin{itemize}
  \item Monoidal unit \((\mb{1}, \ms{emp})\), where \(\ms{emp}\) is the
  predicate which holds only for the empty heap
  \item Associators and inverse associators given by \(\alpha_{(A, \Phi), (B,
  \Psi), (C, \Theta)} = \alpha_{A, B, C}\), which are valid since \(\ast\) is
  associative
  \item Unitors and inverse unitors given by \(\lambda_{(A, \Phi)} =
  \lambda_A\), \(\rho_{(A, \Phi)} = \rho_A\), which are valid since \(\ms{emp}\)
  is the unit of \(\ast\)
  \item \((C, \Theta) \otimes f = C \times f\) and \(f \otimes (C, \Theta) = f
  \times C\), which are valid definitions by the frame rule
\end{itemize}
equips \(\ms{Sep}\) with the structure of a premonoidal category. We can give \(\ms{Sep}\) the structure of a substructural effectful category by taking:
\begin{itemize}
  \item Affine objects to be precisely those for which \(\ms{drop} =
  \upg{\lambda a. ()}{} \in \ms{Sep}((A, \Phi), (\mb{1}, \ms{emp}))\), which for
  this logic corresponds to pairs \((A, \Phi)\) where \(\Phi\) does not depend
  on the heap.
  \item Relevant objects to be precisely those for which \(\ms{split} =
  \upg{\lambda a. (a, a)} \in \ms{Sep}((A, \Phi), (A, \Phi * \Phi))\), again in
  this case those for which \(\Phi\) does not depend on the heap.
  \item Central morphisms to be those which do not print, diverge, \textit{or} modify the
  heap. We will consider how to relax this condition later.
  \item Relevant and affine morphisms to be the lifts of relevant/affine morphisms in \(\ms{Print}\)
\end{itemize}
\(\ms{Sep}\) also has coproducts given by \((A, \Phi) + (B, \Psi) = (A + B, [\Phi, \Psi])\), which allow us to equip it with an Elgot structure by taking advantage of the fact that
\begin{equation}
  f \in \ms{Sep}((A, \Phi), (B + A, [\Psi, \Phi])) \implies f^\dagger \in \ms{Sep}((A, \Phi), (B, \Psi))
\end{equation}
This implies that \(\ms{Sep}\) is an \isotopessa model supporting the following interesting operations
\begin{itemize}
  \item \(\ms{coe}: \mc{I}^\tint_1((A, \Phi), (A, \Psi))\) where \(\forall a,
  \Phi\;a \implies \Psi\;a\), implemented by the identity morphism
  \item \(\ms{print}: \mc{I}^\varnothing_0((\ms{Bytes}, \ms{emp}), (\mb{1},
  \ms{emp}))\)
  \item \(\ms{rand}: \mc{I}^\taff_1((\mb{1}, \ms{emp}), (A, \ms{emp}))\) for \(A\) nonempty
  \item \(\ms{set}: \mc{I}^\varnothing_0((\nats \times \nats, \lambda (\ell,
  \cdot). \exists v. \ell \mapsto v), (\nats, \lambda \ell. \exists v. \ell
  \mapsto v))\), where \(\ms{set}\) is modified slightly to return the address
  at which the value passed in was stored to overcome a lack of dependency
  between the source and target objects.
  \item \(\ms{get}: \mc{I}^\varnothing_0((\nats, \lambda \ell. \exists v. \ell \mapsto v), (\nats, \lambda \ell. \exists v. \ell \mapsto v))\)
  \item \(\ms{alloc}: \mc{I}^\varnothing_0((\nats, \ms{emp}), (\nats, \lambda
  \ell. \exists v. \ell \mapsto v))\)
  \item \(\ms{free}: \mc{I}^\varnothing_0((\nats, \lambda \ell. \exists v. \ell \mapsto v), (\mb{1}, \ms{emp}))\)
\end{itemize}
We can clearly see that this is a refinement type system in the sence of
\cite{mellies-ftrs}: in particular, there is a forgetful functor \((A, \Phi)
\mapsto A\) which is simply the identity on morphisms. Unfortunately, this means
that our notion of equality is a little too stringent, since we need to consider
equality on inputs which do not match our preconditions. Thankfully, we can fix
this problem quite easily by quotienting morphisms over the equivalence relation
\begin{equation}
  \forall f, g: \ms{Sep}((A, \Phi), (B, \Psi)).
  f \simeq g \iff \forall a, h. \Phi\;a\;h \implies f\;a\;h = g\;a\;h
\end{equation}
yielding the new category \(\ms{SComp}\). In particular, this allows us to widen
our definition of central morphisms to include all morphisms which do not print
or diverge, including in particular \(\ms{set}, \ms{get}, \ms{alloc}\) and
\(\ms{free}\), which we can in fact mark relevant and affine as well, allowing
us, for example, to make use of intermediate allocations in an otherwise pure
computation.

\subsection{Weak Memory}

Another interesting use of the trace monad to build a sophisticated \isotopessa model is to build a model of TSO-style weak memory based on \citet{sparky}. For simplicity, we will consider an infinite set of named, fixed locations \(x, y, z \in \ms{Loc}\) which are subject to concurrent modification by all processors via TSO reads, writes, and fences.

We begin with the following definitions:
\begin{definition}[Pomset] 
  A \textbf{pomset} \(\alpha\) over a set of actions \(\mc{A}\) with a
  distinguished null action "\textbf{tick}" \(\delta \in \mc{A}\) is a
  \textit{nonempty} partially-ordered \textbf{carrier set} \(P\) such that every
  \(p \in P\) has finitely many predecessors equipped with a mapping \(\alpha: P
  \to \mc{A}\) quotiented under the equivalence relation \(\alpha: P \to \mc{A}
  \simeq \alpha': Q \to \mc{A}\) if, after removing finitely many \(\delta\)
  from \(P\) and \(Q\), there exists a partial order isomorphism between them
  which respects the values of \(\alpha, \alpha'\). A pomset is \textbf{finite}
  if its carrier set is. We will write unordered pomsets using multiset
  notation, e.g. \(\{a, a, b\}\), and linearly ordered pomsets using list
  notation, e.g. \([a, a, b]\).
\end{definition}
(Finite) pomsets form a monoid under sequential composition \(\alpha;\beta\),
which is defined by the function \([\alpha, \beta]: \ms{trim}(P + Q) \to
\mc{A}\), where \(P + Q\) is given the lexicographic ordering and
\(\ms{trim}(P)\) removes all elements of \(P\) with infinitely many
predecessors. They also form a monoid under parallel composition \(\alpha ||
\beta\), which is defined by the function \([\alpha, \beta]: P + Q \to \mc{A}\)
where \(P + Q\) is given the standard partial ordering (with elements of \(P\)
and \(Q\) incomparable). Note in both cases the monoidal unit is \(\{\delta\}\),
since we only consider nonempty pomsets but allow the removal of finitely many
ticks.

Given a partially ordered set \(N\) and a family of pomsets \(\alpha_n\) for \(n \in N\), we can define their \textit{sum} \(\Sigma_n\alpha_n\) to be given by the function \((n, a) \mapsto \alpha_n\;a: \ms{trim}(\Sigma_nP_n) \to \mc{A}\), where the dependent product \(\Sigma_nP_n\) is given the lexicographic order. In particular, choosing \(N = \nats\) makes \(\Sigma: \ms{Pom}_{\ms{fin}}^\omega \to \ms{Pom}\) into a stream action w.r.t the sequential composition monoid on finite pomsets. 

We define a \textbf{program order pomset} to be a pomset with \(\mc{A}_{\ms{PO}} =
\mc{A}_w \cup \mc{A}_r \cup \{\delta\}\), where \(\mc{A}_r\) consists of
\textit{reads} of the form \(x = v\) and \(\mc{A}_w\) consists of
\textit{writes} of the form \(x := v\) for locations \(x\) and values \(v \in
\ms{Word}\). We define the \textbf{program order monad} \(\ms{PO}\;A =
\ms{Traces}\;\Sigma\;A\), where \(\Sigma\) is the stream action on finite
program order pomsets given by sequential composition for \(\mc{A}\). This
yields an \isotopessa model with support for concurrent read and write
operations \(\ms{read}_x: \mc{I}^\varnothing_0(\mb{1}, \ms{Word})\),
\(\ms{write}_x: \mc{I}^\varnothing_0(\ms{Word}, \mb{1})\) with
semantics
\begin{equation}
  \dnt{\ms{read}_x} = R_x^{\ms{PO}} = \lambda (). \{(v, \{x = v\}) | v \in \ms{Word}\}
  \qquad
  \dnt{\ms{write}_x} = W_x^{\ms{PO}} =  \lambda v. \{((), \{x := v\})\}
\end{equation}
The semantics of \(\ms{read}\) in particular give a hint as to how this model
works: rather than tracking the state of the heap, we simply \textit{emit} a
pomset of the events that would have to happen for a given execution (in the
case of a read, a read event \(x = v\) whenever the read returns \(v\), and in
the case of a write, the single write event \(x := v\) for a write of \(v\)),
and later post-filter on the pomset of an entire program to get a sequentially
consistent execution. Hence, our semantics remains compositional, allowing us to
reason about each program fragment individually, while still allowing us to add
arbitrary external state (e.g. another program concurrently modifying the heap)
before finally considering the execution of an entire program.

On that note, we support the parallel execution of multiple (potentially
nonterminating) morphisms as follows:
\begin{equation}
  \begin{aligned}
  f_0 || f_1 = \lambda (a_0, a_1). 
  & \{\iota_0 ((b_0, b_1), \alpha_0 || \alpha_1) 
    \mid \iota_0 (b_i, \alpha_i) \in f_i\;a_i\} 
  \\ & \cup \{\iota_1 (\alpha_0 || \alpha_1) \mid (\iota_0 (b_0, \alpha_0) \in f_0\;a \lor \iota_0\;\alpha_0 \in f_0\;a_0) \land \iota_1 \alpha_1 \in f_1\;a_1\} 
  \\ & \cup \{\iota_1 (\alpha_0 || \alpha_1) \mid \iota_1\;\alpha_0 \in f_0\;a_0 \land \iota_0 (a_1, \alpha_1) \in f_1\;a_1\}: \ms{Set}_{\ms{PO}}(A \otimes A', B \otimes B')
  \end{aligned}
\end{equation}
It's tempting to try to use this as a tensor product of morphisms and so
graduate to a bona fide monoidal category rather than our current premonoidal
setting, since spatial concurrency is one of the headline use-cases of the
former, but unfortunately, sliding still fails:
\(
  (W_x^{\ms{PO}} || \ms{id}) ; (\ms{id} || W_y^{\ms{PO}})
  \neq (\ms{id} || W_y^{\ms{PO}}) ; (W_x^{\ms{PO}} || \ms{id})
\),
since the left-hand side will emit the pomset \([x := a, y := b]\) while the
right-hand side will emit the pomset \([y := b, x := a]\). If we did want this
behaviour, we'd need to have a significantly more sophisticated model of
composition which effectively takes nontermination into account, which we will
leave to future work.

To graduate from SC concurrency to a genuine, if maximally simple, weak memory
model, we will to introduce a \textit{buffer} of write actions. We will
implement TSO ordering by buffering all our writes, in which case they are only
visible to the local thread. On the other hand, reads events will first attempt
to read from the buffer, and, if there is no corresponding write in the buffer,
will read an arbitrary value, in both cases pushing an event to the global
pomset. At any point, we may choose to \textit{flush} some of the writes in
buffer to the global pomset of events. In particular, we will introduce the set
\(\mc{A}_b = \{(\bufloc{x} := v)\}\) of \textit{buffer write actions}, where an
action \(\bufloc{x} := v\) by a thread denotes adding a write \(x := v\) to the
thread's write buffer. We may then define the set of TSO actions
\(\mc{A}_{\ms{TSO}} = \mc{A}_{\ms{PO}} \cup \mc{A}_b\). A buffer will be defined
to be a list of write actions \(\ms{Buf} = \mc{A}_b^*\), which we will interpret
as linear pomsets over \(\mc{A}_{\ms{TSO}}\) ordered by index (with the empty list corresponding to \(\{\delta\}\)). In particular, we
define the monad \(\ms{TSO} = \ms{StateT}\;\ms{Buf}\;(\ms{Trace}\;\Sigma)\),
where \(\Sigma\) is the stream action of finite pomsets over
\(\mc{A}_{\ms{TSO}}\) on pomsets over \(\mc{A}_{\ms{TSO}}\); we can view
\(\ms{PO}\) as a submonad of this monad in the standard manner.

Given a buffer \(\ms{Buf}\), we can define the result of reads \([\cdot]_x:
\ms{Buf} \to \ms{Word} \sqcup \{\bot\}\) from the buffer as follows:
\begin{equation}
  (L;\{\bufloc{x} := v\})[x] = v
  \qquad
  (L;\{\bufloc{y} := v\})[x] = L[x]
  \qquad
  [][x] = \bot
\end{equation}
Note that writes at the \textit{end} of the buffer are prioritized, since later
writes overwrite earlier ones! The semantics of reads and writes are then given
by
\begin{equation}
  \begin{aligned}
  \dnt{\ms{read}_x} = R_x^{\ms{TSO}} 
    &= \ms{pflush};(\lambda ()\;L. \{(v, L, \{x = v\}) \mid L[x] = v \lor L[x] = \bot\});\ms{pflush} \\
  \dnt{\ms{write}_x} = W_x^{\ms{TSO}}
    &= \ms{pflush};(\lambda v\;L. \{(v, (L;\{\bufloc{x} := v\}), \{x := v\})\});\ms{pflush}
  \end{aligned}
\end{equation}
where buffer flushing is implemented via the morphism (called \(\ms{split}\) in
\cite{sparky})
\begin{equation}
  \ms{pflush} = \lambda ()\;L. \{((), R, \alpha) | L = \alpha;R\}
  : \ms{Set}_{\ms{TSO}}(\mb{1}, \mb{1})
\end{equation}
To be able to perform synchronization, we will also need to introduce a \(\ms{fence}: \mc{I}^\varnothing_0(\mb{1}, \mb{1})\) instruction, which simply causes all actions before the fence to be observed before any actions after the fence. For \(\ms{TSO}\), implementing this is as simple as flushing the buffer, giving
\begin{equation}
  \dnt{\ms{fence}} = \lambda ()\;L. \{((), [], L;\{\delta\})\}
\end{equation}
We can now define the parallel composition of morphisms in a "fork-join" style
as follows: we first flush the buffer completely (i.e., taking it and sticking
it at the beginning of our pomset), then execute \(f\) and \(g\) in parallel
with separate buffers, \textit{filtering out executions which completely flush
the buffer}. Since both threads end up with an empty buffer, the resulting
joined buffer is also empty, giving us the following:
\begin{equation}
  \begin{aligned}
    f_0 || f_1 = \lambda (a_0, a_1)\;L. 
    & \{\iota_0 ((b_0, b_1), [], L;(\alpha_0 || \alpha_1)) 
      \mid \iota_0 (b_i, [], \alpha_i) \in f_i\;a_i\;[]\} 
    \\ & \cup \{\iota_1 (\alpha_0 || \alpha_1) \mid (\iota_0 (b_0, [], \alpha_0) \in f_0\;a \lor \iota_0\;\alpha_0 \in f_0\;a_0\;[]) \land \iota_1 \alpha_1 \in f_1\;a_1\;[]\} 
    \\ & \cup \{\iota_1 (\alpha_0 || \alpha_1) \mid \iota_1\;\alpha_0 \in f_0\;a_0\;[] \land \iota_0 (a_1, [], \alpha_1) \in f_1\;a_1\;[]\}
    \end{aligned}
\end{equation}

\section{Related Work and Discussion}

There is a vast literature on SSA and related compiler IRs, but for
space reasons, we are forced to restrict our discussion to the most
closely related work.

Originally, \citet{kelsey-ssa-cps} showed how to embed SSA into a CPS
representation, and \citet{appel-ssa} showed informally how SSA can be
seen as a collection of mutually-tail-recursive
procedures. \citet{chakravarty-functional-ssa-2003} made Appel's
observation formal, by showing how to translate SSA into A-normal
form~\cite{anf}, and showed how to use the translation to prove the
correctness of a constant propagation analysis. These translations
were for an untyped source and target language, and
\citet{typed-effect-ssa-rigon-torrens-vasconcellos-20} give a typed
translation from SSA into the lambda calculus, which also tracks the
effects of expressions. The effect tracking is handled Koka-style,
where the effect system tracks a list of effect names. (In contrast,
we track the centrality and duplicability of terms, without reference
to the specific effect.)

All of these papers give SSA semantics by translation -- that is, they
defined the semantics of an SSA program to be the semantics of the
image of the translation, with the idea that the simpler semantics of
the functional language can make proofs easier. The Grail language of
\citet{beringer-imp-fun} makes this idea explicit by giving both a
functional and imperative semantics to a low-level language, and
showing the conditions under which they coincide. \citet{schneider-imp-fun} prove
a similar result, and further mechanise the proof in Coq.

There have also been a number of papers which have studied the
semantics of SSA directly.

\citet{ssa-types-matsuno-ohori-06} give a type theory for what appear
to be ordinary three-address code programs, where the types of the
variables at each program point represent the control flows from the
assignment statements reaching that program point. However, the type
information means that every well-typed program can be placed into
SSA-form by inserting $\phi$-nodes in a completely type-directed
way. This lets them avoid having to give an operational reading of
$\phi$-nodes, since they can use the standard semantics for
three-address code.

\citet{hua-explicit-ssa-2010} give another type system for SSA
programs, as well as a direct operational semantics for it, using the
standard sequential heap semantics, as well as proving progress and
preservation for their type system.

There have also been many operational semantics for SSA arising from
compiler verification efforts. \citet{barthe-compcert-ssa-2014} give
an operational semantics of SSA as part of the CompCertSSA project,
and prove that the translation from three-address code into SSA is
semantics preserving. \citet{herklotz-gsa-2023} also formalise an
extension to SSA, "gated SSA", and give it a semantics, and give
semantics-preserving translations between GSA and ordinary SSA. Going
beyond CompCertSSA, \citet{vellum} have studied the semantics of the LLVM
IR itself.

There has been much less work on denotational semantics for SSA, or
directly on its equational theory. \citet{pop-ssa-inout-2009} give an
unusual denotational model of SSA in terms of the iteration structure
of a program, which they use to better understand the loop-closing
$\phi$-nodes found both in the gated SSA representation as well as
practical compilers such as GCC.  \citet{garbuzov-structural-cfg-2018}
exhibit a correspondence between an operational semantics for SSA, and
an operational semantics for the call-by-push-value calculus of
\citet{cbpv}. Despite working over an operational semantics, they use
the normal-form bisimulations of \citet{lassen-bisim} to derive an
equational theory for use in justifying optimisations. Their
operational semantics does contain not contain any effects, which
means that it cannot justify effect-dependent optimisations such as
eliminating redundant writes.

Our work is the first to validate the equational theory of SSA against
a weak memory semantics. We focused on denotational models of weak
memory, since they make it easier to establish equational properties
than the operational and axiomatic models. However, for our proof to
go through, we needed a number of features of the semantics which are
sometimes difficult to establish in the weak memory world. First,
sequencing must be associative (i.e.,
$(t_1; t_2); t_3 \equiv t_1; (t_2; t_3)$).  Second, the semantics must
support loops, including the equation that a while-loop is equal to
its unrolling. Finally, sums need to be distributive (e.g., in the
boolean case, $(A \otimes \mb{2}) \simeq (A + A)$).

As a result, we focused on the TSO semantics of \citet{sparky}, which
does have all the properties we need. For the weaker ARM-style models
(which admit load buffering), there seem to be a number of models
which are "near misses", in that they have some, but not all, the
properties we need. The event structure model of
\citet{paviotti-modular-relaxed-dep-20} is formulated in terms of
step-indexing, so satisfies all of our needed properties except the
loop unrolling property -- step-indexing means it is a refinement
rather than an equation. The pomsets with preconditions model of
\citet{jagadeesan-pwp-20} does not exactly satisfy the associativity
of semicolon, and while the pomsets with predicate transformers model
of \citet{leaky-semicolon} does satisfy associativity, it does not yet
support loops.

We do not think it would be prohibitively difficult to extend these
models to support the required features, and we believe our results
suggest that it would be have a high payoff: those features would
suffice to build a model of our IR, giving us tight results
validating the transformations compilers want against the low-level
specification of the actual machine. 

\bibliographystyle{ACM-Reference-Format}
\bibliography{references}

\clearpage 

\appendix

\section{Proofs}

\subsection{Weakening and Substitution}

\begin{theorem}[Weakening] \
  \begin{itemize}
    \item If \(\cwk{\Gamma}{\Delta}\) and \(q \subseteq r\),
    \(\hasty{\Delta}{p}{a}{A}{r}\), then \(\hasty{\Gamma}{p}{a}{A}{q}\)
    \item If \(\cwk{\Gamma}{\Delta}\), \(\haslb{\Delta}{t}{\ms{L}}\), then
    \(\haslb{\Gamma}{t}{\ms{L}}\)
    \item If \(\lwk{\ms{L}}{\ms{K}}\), \(\haslb{\Gamma}{t}{\ms{L}}\) then
    \(\haslb{\Gamma}{t}{\ms{K}}\)
  \end{itemize}
\end{theorem}

\begin{proof} \
  \TODO{this}
\end{proof}

\begin{lemma}[Substitution Splitting] 
  If \(\csplits{\Gamma}{\Delta}{\Xi}\) and \(\issubst{\gamma}{\Theta}{\Gamma}\),
  then there exist \(\Theta_\Delta\), \(\Theta_\Xi\) such that:
  \begin{itemize}
    \item \(\csplits{\Theta}{\Theta_\Delta}{\Theta_\Xi}\)
    \item \(\issubst{\gamma}{\Theta_\Delta}{\Delta}\)
    \item \(\issubst{\gamma}{\Theta_\Xi}{\Xi}\)
  \end{itemize}
\end{lemma}

\begin{proof} 
  We proceed by induction on derivations \(\csplits{\Gamma}{\Delta}{\Xi}\). In
  the \rle{split-nil} case, we have that \(\issubst{\gamma}{\cdot}{\cdot}\) by
  \rle{subst-nil}. For all other cases, since \(\issubst{\gamma}{\Theta}{\Gamma,
  \thyp{x}{A}{q}}\), we have by \rle{subst-cons} that
  \(\csplits{\Theta}{\Theta_\Gamma}{\Theta_x}\),
  \(\issubst{\gamma}{\Theta_\Gamma}{\Gamma}\),
  \(\hasty{\Theta_\Gamma}{1}{\gamma(x)}{A}{q}\). Since
  \(\csplits{\Gamma}{\Delta}{\Xi}\), by induction, we have that
  \(\csplits{\Theta_\Gamma}{\Theta_\Delta}{\Theta_\Xi}\) where \(\Theta_\Delta\)
  and \(\Theta_\Xi\) have the desired properties above.
  \begin{itemize}
    \item \rle{split-left}: by left split association, we have that there exists
    \(\Theta_{\Delta, \thyp{x}{A}{q}}\), \(\csplits{\Theta}{\Theta_{\Delta,
    \thyp{x}{A}{q}}}{\Theta_\Xi}\), \(\csplits{\Theta_{\Delta,
    \thyp{x}{A}{q}}}{\Theta_\Delta}{\Theta_x}\). Using the fact that
    \(\issubst{\gamma}{\Theta_\Delta}{\Delta}\), we can conclude by
    \rle{subst-cons} that \(\issubst{\gamma}{\Theta_{\Delta,
    \thyp{x}{A}{q}}}{\Delta, \thyp{x}{A}{q}}\); the desired result hence
    follows.
    \item \rle{split-right}: by right split association, we have that there
    exists \(\Theta_{\Xi, \thyp{x}{A}{q}}\),
    \(\csplits{\Theta}{\Theta_\Delta}{\Theta_{\Xi, \thyp{x}{A}{q}}}\),
    \(\csplits{\Theta_{\Xi, \thyp{x}{A}{q}}}{\Theta_\Xi}{\Theta_x}\). Using the
    fact that \(\issubst{\gamma}{\Theta_\Xi}{\Xi}\), we can conclude by
    \rle{subst-cons} that \(\issubst{\gamma}{\Theta_{\Xi, \thyp{x}{A}{q}}}{\Xi,
    \thyp{x}{A}{q}}\); the desired result hence follows.
    \item \rle{split-dup}: since \(\tlin{A}{\trel}\), it follows by
    \rle{subst-cons} that \(\tlin{\Theta_x}{\trel}\), and hence that
    \(\csplits{\Theta_x}{\Theta_x}{\Theta_x}\). Therefore, by split association,
    we have that there exist \(\Theta_{\Delta, \thyp{x}{A}{q}}\), \(\Theta_{\Xi,
    \thyp{x}{A}{q}}\) such that \(\csplits{\Theta}{\Theta_{\Delta,
    \thyp{x}{A}{q}}}{\Theta_{\Xi, \thyp{x}{A}{q}}}\). Using the facts that
    \(\issubst{\gamma}{\Theta_\Delta}{\Delta}\),
    \(\issubst{\gamma}{\Theta_\Xi}{\Xi}\), we can conclude by \rle{subst-cons}
    that \(\issubst{\gamma}{\Theta_{\Delta, \thyp{x}{A}{q}}}{\Delta,
    \thyp{x}{A}{q}}\), \(\issubst{\gamma}{\Theta_{\Xi, \thyp{x}{A}{q}}}{\Xi,
    \thyp{x}{A}{q}}\); the desired result hence follows.
    \item \rle{split-drop}: since \(\tlin{A}{\taff}\), it follows by
    \rle{subst-cons} that \(\tlin{\Theta_x}{\taff}\), and hence that
    \(\cwk{\Theta_x}{\cdot}\). Therefore, by split association,
    \(\csplits{\Theta}{\Theta_\Delta}{\Theta_\Xi}\), giving the desired result.
  \end{itemize}
\end{proof}

\begin{lemma}[Substitution] 
  Given \(\issubst{\gamma}{\Theta}{\Gamma}\),
  \begin{itemize}
    \item If \(\hasty{\Gamma}{p}{a}{A}{q}\), then \(\hasty{\Theta}{p}{[\gamma]a}{A}{q}\)
    \item If \(\haslb{\Gamma}{t}{\ms{L}}\), then there exists \(\ms{K}\) such that \(\lbsubst{\substlbs{\gamma}{\ms{L}}}{\ms{K}}{\ms{L}}\) and \(\haslb{\Theta}{[\gamma]t}{\ms{K}}\)
    \item If \(\lhaslb{\ms{W}}{L}{\ms{L}}\), then there exists \(\ms{K}, \ms{W}'\) such that 
    \begin{itemize}
      \item \(\lbsubst{\substlbs{\gamma}{\ms{L}}}{\ms{K}}{\ms{L}}\) 
      \item \(\lbsubst{\substlbs{\gamma}{\ms{W}'}}{\ms{W}}{\ms{W}'}\) 
      \item \(\lhaslb{\ms{W}'}{[\gamma]L}{\ms{K}}\)
    \end{itemize}
  \end{itemize}
\end{lemma}

\begin{proof} \
  \TODO{this}
\end{proof}

\begin{lemma}[Label Substitution] 
  Given \(\lbsubst{\mc{L}}{\ms{L}}{\ms{K}}\), 
  \begin{itemize}
    \item For all \(\haslb{\Gamma}{t}{\ms{L}}\), \(\haslb{\Gamma}{[\mc{L}]t}{\ms{K}}\)
    \item For all \(\lhaslb{\ms{W}}{L}{\ms{L}}\), \(\lhaslb{\ms{W}}{[\mc{L}]L}{\ms{K}}\)
  \end{itemize}
\end{lemma}

\begin{proof} \
  \TODO{this}
\end{proof}

\subsection{Semantic Metatheory}

\begin{lemma}[Semantic Upgrade]
  For all \(\Gamma, a, A\), if \(\hasty{\Gamma}{1}{a}{A}{q}\), then
  \[\dnt{\hasty{\Gamma}{0}{a}{A}{q}} = \upg{\dnt{\hasty{\Gamma}{1}{a}{A}{q}}}{}\]
\end{lemma}

\begin{proof} \
  \TODO{this}
\end{proof}

\begin{lemma}[Semantic Weakening] 
  If \(\cwk{\Gamma}{\Delta}\), then 
  \begin{itemize}
    \item For all \(\hasty{\Delta}{p}{a}{A}{q}\),
    \[
      \upg{\dnt{\cwk{\Gamma}{\Delta}}}{p}
      ; \dnt{\hasty{\Delta}{p}{a}{A}{q}}
      = \dnt{\hasty{\Gamma}{p}{a}{A}{q}} 
    \]
    \item For all \(\haslb{\Delta}{t}{\ms{L}}\),
    \[
      \upg{\dnt{\cwk{\Gamma}{\Delta}}}{p}
      ; \dnt{\haslb{\Delta}{t}{\ms{L}}}
      = \dnt{\haslb{\Gamma}{t}{\ms{L}}}  
    \]
  \end{itemize}

  Similarly, if \(\lwk{\ms{L}}{\ms{K}}\) and \(\haslb{\Gamma}{t}{\ms{L}}\), then
  \[
    \dnt{\haslb{\Gamma}{t}{\ms{L}}}
    ; \dnt{\lwk{\ms{L}}{\ms{K}}}
    = \dnt{\haslb{\Gamma}{t}{\ms{K}}}
  \]
\end{lemma}

\begin{proof} \
  \TODO{this}
\end{proof}

\begin{lemma}[Substitution Coherence] \
  \begin{itemize}
    \item Given any two derivations \(D_1: \issubst{\gamma}{\Theta}{\Gamma}\), \(D_2: \issubst{\gamma'}{\Theta}{\Gamma}\) such that \(\forall x \in \Theta, \gamma(x) = \gamma'(x)\), we have \(\dnt{D_1} = \dnt{D_2}\)
    \item Given any two derivations \(D_1: \lbsubst{\mc{L}}{\ms{L}}{\ms{K}}\), \(D_2: \lbsubst{\mc{L}'}{\ms{L}}{\ms{K}}\) such that \(\forall \lbl{\ell} \in \ms{L}, \mc{L}(x) = \mc{L}'(x)\), we have \(\dnt{D_1} = \dnt{D_2}\)
  \end{itemize}
\end{lemma}

\begin{proof} \
  \TODO{this}
\end{proof}

\begin{lemma}[Semantic Substitution Splitting] 
  Given \(\csplits{\Gamma}{\Delta}{\Xi}\) and \(\issubst{\gamma}{\Delta}{\Xi}\),
  if \(\csplits{\Theta}{\Theta_\Delta}{\Theta_\Xi}\) and \(\issubst{\gamma}{\Theta_\Delta}{\Delta}\), \(\issubst{\gamma}{\Theta_\Xi}{\Xi}\), then
  \[
    \dnt{\issubst{\gamma}{\Theta}{\Gamma}};\dnt{\csplits{\Gamma}{\Delta}{\Xi}}
    = \dnt{\csplits{\Theta}{\Theta_\Delta}{\Theta_\Xi}};
      \dnt{\issubst{\gamma}{\Theta_\Delta}{\Delta}}
      \otimes \dnt{\issubst{\gamma}{\Theta_\Xi}{\Xi}}
  \]
\end{lemma}

\begin{proof} \
  \TODO{this}
\end{proof}

\begin{theorem}[Semantic Substitution] 
  Given \(\issubst{\gamma}{\Theta}{\Gamma}\),
  \begin{itemize}
    \item For all \(\hasty{\Gamma}{p}{a}{A}{q}\), 
    \[
      \upg{\dnt{\issubst{\gamma}{\Theta}{\Gamma}}}{p}
      ;\dnt{\hasty{\Gamma}{p}{a}{A}{q}} 
      = \dnt{\hasty{\Theta}{p}{[\gamma]a}{A}{q}}
    \]
    \item For all 
      \(\haslb{\Gamma}{t}{\ms{L}}\), 
      \(\lbsubst{\substlbs{\gamma}{\ms{L}}}{\ms{K}}{\ms{L}}\), 
      \(\haslb{\Theta}{[\gamma]t}{\ms{K}}\),
    \[
      \upg{\dnt{\issubst{\gamma}{\Theta}{\Gamma}}}{}
      ; \dnt{\haslb{\Gamma}{t}{\ms{L}}}
      = \dnt{\haslb{\Theta}{[\gamma]t}{\ms{K}}} 
      ; \dnt{\lbsubst{\substlbs{\gamma}{\ms{K}}}{\ms{K}}{\ms{L}}}
    \]
    \item For all 
      \(\lhaslb{\ms{W}}{L}{\ms{L}}\), 
      \(\lbsubst{\substlbs{\gamma}{\ms{L}}}{\ms{K}}{\ms{L}}\),
      \(\lbsubst{\substlbs{\gamma}{\ms{W'}}}{\ms{W}}{\ms{W'}}\),
      \[
        \dnt{\lbsubst{\substlbs{\gamma}{\ms{W'}}}{\ms{W}}{\ms{W'}}}
        ; \dnt{\lhaslb{\ms{W'}}{[\gamma]L}{\ms{K}}}
        = \dnt{\lhaslb{\ms{W}}{L}{\ms{L}}}
        ; \dnt{\lbsubst{\substlbs{\gamma}{\ms{L}}}{\ms{K}}{\ms{L}}}
        + \dnt{\lbsubst{\substlbs{\gamma}{\ms{W'}}}{\ms{W}}{\ms{W'}}}
      \]
  \end{itemize}
  Similarly, given \(\lbsubst{\mc{L}}{\ms{L}}{\ms{K}}\),
  \begin{itemize} 
    \item For all \(\haslb{\Gamma}{t}{\ms{L}}\), we have
    \[
      \dnt{\haslb{\Gamma}{t}{\ms{L}}};\dnt{\lbsubst{\mc{L}}{\ms{L}}{\ms{K}}}
      = \dnt{\haslb{\Gamma}{[\mc{L}]t}{\ms{K}}}
    \]
    \item For all \(\lhaslb{\ms{W}}{L}{\ms{L}}\), we have
    \[
      \dnt{\lhaslb{\ms{W}}{L}{\ms{L}}};\dnt{\lbsubst{\mc{L}}{\ms{L}}{\ms{K}}}
      = \dnt{\lhaslb{\ms{W}}{[\mc{L}]L}{\ms{L}}}
    \]
  \end{itemize}
\end{theorem}

\begin{proof} \
  \TODO{this}
\end{proof}

\section{Loop Hoisting}

\TODO{this?}

\section{Rewriting}

\subsection{Syntax}

To properly be able to reason about congruence and congruence-based
optimization, we need to introduce the concept of \textit{terms with holes},
which are defined using the grammar in Figure
\ref{fig:blocks-with-holes-grammar}. This simply extends the three main
syntactic categories with \textit{holes} of the form \(\lhole{X}\),
\(\lhole{T}\), \(\lhole{W}\). To \textit{type} terms with holes, we need to
introduce the concept of a \textit{hole-context} \(H\), which types a hole as being
one of:
\begin{itemize}
  \item \todo{expression}
  \item \todo{block}
  \item \todo{label-set}
\end{itemize}
\TODO{hole typing rules for expressions, blocks, label-sets; everything else is
the same}.

\begin{figure}
  \begin{grammar}
    <\(\mhole{a}, \mhole{b}, \mhole{c}, \mhole{e}\)> ::= \(\lhole{X}\)
    \;|\; \(x\) 
    \;|\; \(f\;\mhole{a}\)
    \;|\; \((\mhole{a}, \mhole{b})\) 
    \;|\; \(()\) 
    \;|\; \(\ctt\) 
    \;|\; \(\cff\)
    \;|\; \(\letexpr{x}{\mhole{a}}{\mhole{e}}\)
    \;|\; \(\letexpr{(x, y)}{\mhole{a}}{\mhole{e}}\)
    % \alt \(\lbsplice{\ell}{x: A}{\mhole{t}}\)

    <\(\mhole{s}, \mhole{t}\)> ::= \(\lhole{T}\) 
    \;|\; \(\lbrb{\ell}{\mhole{a}}\) 
    \;|\; \(\ite{\mhole{e}}{\mhole{s}}{\mhole{t}}\)
    \;|\; \(\letstmt{x}{\mhole{a}}{\mhole{t}}\)
    \;|\; \(\letstmt{(x, y)}{\mhole{a}}{\mhole{t}}\)
    \;|\; \(\ewhere{\mhole{t}}{\mhole{L}}\)

    <\(\mhole{L}\)> ::= \(\lhole{W}\) \;|\; \(\cdot\) \;|\; \(\lwbranch{\ell}{x: A}{\mhole{t}}, \mhole{L}\)

    <\(H\)> ::= \(\cdot\) 
    \;|\; \(H, \tyhole{\lhole{X}}{\Gamma}{p}{A}{q}\)
    \;|\; \(H, \blkhole{\lhole{T}}{\Gamma}{\ms{L}}\)
    \;|\; \(H, \cfghole{\lhole{W}}{\ms{L}}{\ms{K}}\)
  \end{grammar}
  \caption{Grammar for \isotopessa expressions and blocks with holes}
  \Description{Grammar for isotope-SSA blocks with holes}
  \label{fig:blocks-with-holes-grammar}
\end{figure}

\begin{figure}
  \begin{gather*}    
    \prftree[r]{\rle{hole-tm}}
      {\tyhole{\lhole{X}}{\Delta}{p}{A}{q}}
      {\cwk{\Gamma}{\Delta}}
      {\mhasty{H}{\Gamma}{p}{\lhole{X}}{A}{q}} \qquad
    \prftree[r]{\rle{var}}
      {\cwk{\Gamma}{\thyp{x}{A}{q}}}
      {\mhasty{H}{\Gamma}{p}{x}{A}{q}} \qquad
    \prftree[r]{\rle{app}}
      {f \in \mc{I}_p^q(A, B)}
      {\mhasty{H}{\Gamma}{1}{\mhole{a}}{A}{q}}
      {\mhasty{H}{\Gamma}{p}{f\;\mhole{a}}{B}{q}} \\
    \prftree[r]{\rle{pair}}
      {\csplits{\Gamma}{\Delta}{\Xi}}
      {\mhasty{H}{\Delta}{1}{\mhole{a}}{A}{q}}
      {\mhasty{H}{\Xi}{1}{\mhole{b}}{B}{q}}
      {\mhasty{H}{\Gamma}{p}{(\mhole{a}, \mhole{b})}{A \otimes B}{q}} \\
    \prftree[r]{\rle{unit}}
      {\cwk{\Gamma}{\cdot}}
      {\mhasty{H}{\Gamma}{p}{()}{\mb{1}}{q}} \qquad
    \prftree[r]{\rle{true}}
      {\cwk{\Gamma}{\cdot}}
      {\mhasty{H}{\Gamma}{p}{\ctt}{\mb{2}}{q}} \qquad
    \prftree[r]{\rle{false}}
      {\cwk{\Gamma}{\cdot}}
      {\mhasty{H}{\Gamma}{p}{\cff}{\mb{2}}{q}} \\
    \prftree[r]{\rle{let}}
      {\csplits{\Gamma}{\Delta}{\Xi}}
      {\mhasty{H}{\Delta, \thyp{x}{A}{}}{p}{\mhole{e}}{B}{q}}
      {\mhasty{H}{\Xi}{1}{\mhole{a}}{A}{q}}
      {\mhasty{H}{\Gamma}{p}{\letexpr{x}{\mhole{a}}{\mhole{e}}}{B}{q}} \qquad
    % \prftree[r]{\rle{blk}}
    %   {\mhaslb{H}{\Gamma}{\mhole{t}}{\llhyp{\ell}{\cdot}{A}}}
    %   {\mhasty{H}{\Gamma}{0}{\lbsplice{\ell}{A}{\mhole{t}}}{A}{\varnothing}} 
      \\
    \prftree[r]{\rle{let2}}
      {\csplits{\Gamma}{\Delta}{\Xi}}
      {\mhasty{H}{\Delta, \thyp{x}{A}{}, \thyp{y}{B}{}}{p}{\mhole{e}}{C}{q}}
      {\mhasty{H}{\Xi}{1}{\mhole{a}}{A \otimes B}{q}}
      {\mhasty{H}{\Gamma}{p}{\letexpr{(x, y)}{\mhole{a}}{\mhole{e}}}{C}{q}}
  \end{gather*}
  \caption{Typing rules for \isotopessa expressions with holes}
  \Description{Typing rules for isotope-SSA expressions with holes}
  \label{fig:terms-with-holes-typing}
\end{figure}

\begin{figure}
  \begin{gather*}    
    \prftree[r]{\rle{hole-blk}}
      {\blkhole{\lhole{T}}{\Delta}{\ms{L}} \in H}
      {\cwk{\Gamma}{\Delta}}
      {\mhaslb{H}{\Gamma}{\lhole{T}}{\ms{L}}} 
      \qquad
    \prftree[r]{\rle{br}}
      {\csplits{\Gamma}{\Delta}{\Xi}}
      {\lwk{\llhyp{\ell}{\Delta}{A}}{\ms{L}}}
      {\mhasty{H}{\Xi}{1}{\mhole{a}}{A}{\tint}}
      {\mhaslb{H}{\Gamma}{\lbrb{\ell}{\mhole{a}}}{\ms{L}}} 
      \\
    \prftree[r]{\rle{ite}}
      {\csplits{\Gamma}{\Delta}{\Xi}}
      {\mhasty{H}{\Delta}{1}{\mhole{e}}{\mb{2}}{\tint}}
      {\mhaslb{H}{\Xi}{\mhole{s}}{\ms{L}}}
      {\mhaslb{H}{\Xi}{\mhole{t}}{\ms{L}}}
      {\mhaslb{H}{\Gamma}{\ite{\mhole{e}}{\mhole{s}}{\mhole{t}}}{\ms{L}}} \\
    \prftree[r]{\rle{let-blk}}
      {\csplits{\Gamma}{\Delta}{\Xi}}
      {\mhaslb{H}{\Delta, \thyp{x}{A}{}}{\mhole{t}}{\ms{L}}}
      {\mhasty{H}{\Xi}{0}{\mhole{a}}{A}{q}}
      {\mhaslb{H}{\Gamma}{\letstmt{x}{\mhole{a}}{\mhole{t}}}{\ms{L}}} \\
    \prftree[r]{\rle{let2-blk}}
      {\csplits{\Gamma}{\Delta}{\Xi}}
      {\mhaslb{H}{\Delta, \thyp{x}{A}{}, \thyp{y}{B}{}}{\mhole{t}}{\ms{L}}}
      {\mhasty{H}{\Xi}{0}{\mhole{a}}{A \otimes B}{q}}
      {\mhaslb{H}{\Gamma}{\letstmt{(x, y)}{\mhole{a}}{\mhole{t}}}{\ms{L}}} \\
    \prftree[r]{\rle{where}}
      {\mhaslb{H}{\Gamma}{\mhole{t}}{\ms{L}}}
      {\mlhaslb{H}{\ms{L}}{\mhole{L}}{\ms{K}}}
      {\mhaslb{H}{\Gamma}{\ewhere{\mhole{t}}{\mhole{L}}}{\ms{K}}} 
      \qquad
    \prftree[r]{\rle{hole-cfg}}
      {\cfghole{\lhole{W}}{\ms{L}}{\ms{K}} \in H}
      {\lwk{\ms{M}}{\ms{K}}}
      {\mlhaslb{\ms{L}}{H}{\lhole{W}}{\ms{K}}}
      \\
    \prftree[r]{\rle{nil-br}}
      {\mlhaslb{H}{\ms{L}}{\cdot}{\ms{L}}} \qquad
    \prftree[r]{\rle{cons-br}}
      {\mlhaslb{H}{\ms{L}}{\mhole{L}}{\ms{K}, \llhyp{\ell}{\Gamma}{A}}}
      {\mhaslb{H}{\Gamma, \thyp{x}{A}{}}{\mhole{t}}{\ms{L}}}
      {\mlhaslb{H}{\ms{L}}{\mhole{L}, \lwbranch{\ell}{x: A}{\mhole{t}}}{\ms{K}}}
  \end{gather*}
  \caption{Typing rules for \isotopessa blocks with holes}
  \Description{Typing rules for isotope-SSA blocks with holes}
  \label{fig:blocks-with-holes-typing}
\end{figure}

\begin{figure}
  \begin{gather*}
    \boxed{H: \ms{Hole} \to \ms{MExpr} \sqcup \ms{MBlock} \sqcup \ms{MCfg}}
    \\
    \prftree[r]{\rle{rw-nil}}
      {\isrw{\mc{H}}{H}{\cdot}}
    \qquad
    \prftree[r]{\rle{rw-tm}}
      {\isrw{\mc{H}}{I}{H}}
      {\mhasty{I}{\Gamma}{p}{\mc{H}(\lhole{X})}{A}{q}}
      {\isrw{\mc{H}}{I}{H, \tyhole{\lhole{X}}{\Gamma}{p}{A}{q}}}
    \\
    \prftree[r]{\rle{rw-blk}}
      {\isrw{\mc{H}}{I}{H}}
      {\mhaslb{I}{\Gamma}{\mc{H}(\lhole{T})}{\ms{L}}}
      {\isrw{\mc{H}}{I}{H, \blkhole{\lhole{T}}{\Gamma}{\ms{L}}}}
    \qquad
    \prftree[r]{\rle{rw-cfg}}
      {\isrw{\mc{H}}{I}{H}}
      {\mlhaslb{I}{\ms{L}}{\mc{H}(\lhole{W})}{\ms{K}}}
      {\isrw{\mc{H}}{I}{H, \cfghole{\lhole{L}}{\ms{L}}{\ms{K}}}}
  \end{gather*}
  \caption{Typing rules for \isotopessa rewrites}
  \Description{Typing rules for isotope-SSA rewrites}
  \label{fig:rewrite-typing}
\end{figure}

\TODO{some text about rewrite typing rules}

\begin{theorem}[Rewriting]
  Given \(\isrw{\mc{H}}{H}{I}\), we have
  \begin{itemize}
    \item If \(\mhasty{H}{\Gamma}{p}{a}{A}{q}\), then \(\mhasty{I}{\Gamma}{p}{[\mc{H}]a}{A}{q}\)
    \item If \(\mhaslb{H}{\Gamma}{t}{\ms{L}}\), then \(\mhaslb{I}{\Gamma}{[\mc{H}]t}{\ms{L}}\)
  \end{itemize}
  Furthermore, we have that 
  \begin{itemize}
    \item \(\mhasty{\cdot}{\Gamma}{p}{a}{A}{q} \iff \hasty{\Gamma}{p}{a}{A}{q}\)
    \item \(\mhaslb{\cdot}{\Gamma}{t}{\ms{L}} \iff \haslb{\Gamma}{t}{\ms{L}}\)
  \end{itemize}
\end{theorem}

\subsection{Semantics}

\TODO{text}

\begin{figure}
  \begin{align*}
    \dnt{H, \tyhole{\lhole{X}}{\Gamma}{p}{A}{q}}
      &= \dnt{H} \times \mc{C}_p^q(\dnt{\Gamma}, \dnt{A}) 
      \\
    \dnt{H, \blkhole{\lhole{T}}{\Gamma}{\ms{L}}}
      &= \dnt{H} \times \mc{C}_0(\dnt{\Gamma}, \dnt{\ms{L}}) 
      \\
    \dnt{H, \cfghole{\lhole{W}}{\ms{L}}{\ms{K}}}
      &= \dnt{H} \times \mc{C}_0(\dnt{\ms{L}}, \dnt{\ms{K}} + \dnt{\ms{L}}) 
      \\
    \dnt{\cdot} 
      &= \mb{1}
  \end{align*}
  \caption{Semantics for \isotopessa holes}
  \Description{Semantics for isotope-SSA holes}
  \label{fig:holes-semantics}
\end{figure}

\TODO{some text about semantics of holes}

\begin{figure}
  \begin{gather*}
    \boxed{
      \dnt{\mhasty{H}{\Gamma}{p}{\mhole{a}}{A}{q}}
      : \dnt{H} \to \mc{C}_p^q(\dnt{\Gamma}, \dnt{A})}
      \\
    \sorry
  \end{gather*}
  \caption{Semantics for \isotopessa expressions with holes}
  \Description{Semantics for isotope-SSA expressions with holes}
  \label{fig:terms-with-holes-semantics}
\end{figure}

\TODO{fill}

\TODO{some text about semantics of terms with holes}

\begin{figure}
  \begin{gather*}
    \boxed{
      \dnt{\mhaslb{H}{\Gamma}{\mhole{t}}{\ms{L}}}
      : \dnt{H} \to \mc{C}_0^\varnothing(\dnt{\Gamma}, \dnt{\ms{L}})}
      \\
    \sorry
      \\  
    \boxed{
      \dnt{\mlhaslb{H}{\ms{L}}{\mhole{L}}{\ms{K}}}
      : \dnt{H} \to \mc{C}_0^\varnothing(\dnt{\ms{L}}, \dnt{\ms{K}} + \dnt{\ms{L}})}
      \\
    \sorry
  \end{gather*}
  \caption{Semantics for \isotopessa blocks with holes}
  \Description{Semantics for isotope-SSA blocks with holes}
  \label{fig:blocks-with-holes-semantics}
\end{figure}

\TODO{fill}

\TODO{some text about semantics of blocks with holes}

\begin{figure}
  \begin{gather*}
    \boxed{
      \dnt{\isrw{\mc{H}}{I}{H}}
      : \dnt{I} \to \dnt{H} 
    }
    \\
    \dnt{\isrw{\mc{H}}{I}{\cdot}}\;i = ()
    \qquad
    \dnt{\isrw{\mc{H}}{I}{H, \tyhole{\lhole{X}}{\Gamma}{p}{A}{q}}}\;i
    = (\dnt{\isrw{\mc{H}}{I}{H}}\;i, \dnt{\mhasty{I}{\Gamma}{p}{\mc{H}(\lhole{X})}{A}{q}}\;i)
    \\
    \dnt{\isrw{\mc{H}}{I}{H, \blkhole{\lhole{T}}{\Gamma}{\ms{L}}}}
    = (\dnt{\isrw{\mc{H}}{I}{H}}\;i, \dnt{\mhaslb{I}{\Gamma}{\mc{H}(\lhole{T})}{\ms{L}}}\;i)
    \\
    \dnt{\isrw{\mc{H}}{I}{H, \cfghole{\lhole{L}}{\ms{L}}{\ms{K}}}}
    = (\dnt{\isrw{\mc{H}}{I}{H}}\;i, \dnt{\mlhaslb{I}{\ms{L}}{\mc{H}(\lhole{W})}{\ms{K}}}\;i)
  \end{gather*}
  \caption{Semantics for \isotopessa rewrites}
  \Description{Semantics for isotope-SSA rewrites}
  \label{fig:rewrite-semantics}
\end{figure}

\TODO{fill}

\TODO{some text about semantics of rewrites}

\begin{lemma}[Rewriting Coherence] \
  \begin{itemize}
    \item Given any two derivations \(D_1: \mhasty{H}{\Gamma}{p}{a}{A}{q}\), \(D_2: \mhasty{\Gamma}{p}{H}{a}{A}{r}\), \(\dnt{D_1} = \dnt{D_2}\)
    \item Given any two derivations \(D_1, D_2: \mhaslb{H}{\Gamma}{t}{\ms{L}}\), \(\dnt{D_1} = \dnt{D_2}\)
    \item Given any two derivations \(D_1, D_2: \isrw{\mc{H}}{I}{H}\), \(\dnt{D_1} = \dnt{D_2}\)
  \end{itemize}
\end{lemma}

\begin{theorem}[Semantic Rewriting]
  Given \(\isrw{\mc{H}}{I}{H}\), we have that
  \begin{itemize}
    \item For all \(\mhasty{H}{\Gamma}{p}{a}{A}{q}\), \(m \in \dnt{I}\), we have
    \[
      \dnt{\mhasty{I}{\Gamma}{p}{[\mc{H}]a}{A}{q}}(m)
      = \dnt{\mhasty{H}{\Gamma}{p}{a}{A}{q}}(\dnt{\isrw{\mc{H}}{I}{H}}(m))
    \]
    \item For all \(\mhaslb{H}{\Gamma}{t}{\ms{L}}\), \(m \in \dnt{I}\), we have
    \[
      \dnt{\mhaslb{I}{\Gamma}{[\mc{H}]t}{\ms{L}}}(m)
      = \dnt{\mhaslb{H}{\Gamma}{t}{\ms{L}}}(\dnt{\isrw{\mc{H}}{I}{H}}(m))
    \]
  \end{itemize}

  Furthermore, we have that
  \[
    \dnt{\mhasty{\cdot}{\Gamma}{p}{a}{A}{q}}() = \dnt{\hasty{\Gamma}{p}{a}{A}{q}}
    \qquad
    \dnt{\mhaslb{\cdot}{\Gamma}{t}{\ms{L}}}() = \dnt{\haslb{\Gamma}{t}{\ms{L}}}
  \]
\end{theorem}

% We can further use this example to get a substructural \textit{monoidal} category by considering a primitive form of separation logic: let's introduce a grammar of predicates \(\varphi, \psi ::= \top, \ms{S}, \bot\) for programs which cannot access state, can access state, and are invalid, respectively. The separating conjunction operator on these predicates is then given by
% \begin{equation}
%   \top * \varphi = \varphi * \top = \varphi
%   \qquad
%   \bot * \varphi = \varphi * \bot = \bot
%   \qquad
%   \ms{S} * \ms{S} = \bot
% \end{equation}
% We'll then define a category \(\ms{SRel}\) with objects of the form \((A, \varphi)\) and morphisms
% \begin{equation}
%   \begin{gathered}
%     \boxed{
%       \ms{SRel}((A, \varphi), (B, \psi)) 
%       \subseteq A \to \ms{S} \to \mc{P}(A \times S) 
%     }
%     \\
%     \begin{aligned}
%       \ms{SRel}((A, \top), (B, \top)) &= \{f \in A \to S \to \mc{P}(A \times S) \mid \exists g. f = \lambda a, s. g(a) \times \{s\}\} \\
%       \ms{SRel}((A, \ms{S}), (B, \ms{S})) &= A \to S \to \mc{P}(A \times S) \\
%       \ms{SRel}((A, \varphi), (B, \psi)) &= \{\lambda a, s. \varnothing\} \quad \text{otherwise}
%     \end{aligned}
%   \end{gathered}
% \end{equation}
% Note that it is precisely the \textit{central} morphisms \(\ms{C}_0(A, B)\) which lie in \(\ms{SRel}((A, \top), (B, \top))\). We can then define the tensor product on objects as \((A, \varphi) \otimes (B, \psi) = (A \otimes B, \varphi * \psi)\), and, for \(f \in \mathsf{SRel}((A, \varphi), (B, \psi))\), and tensor product functors
% \begin{equation}
%   \begin{aligned}
%   f \otimes (C, \phi) &= f \otimes C,
%   & (C, \phi) \otimes f &= C \otimes f
%   && \text{if}\; \varphi * \phi \neq \bot, \psi * \phi \neq \bot \\
%   f \otimes (C, \phi) &= \lambda a, s. \varnothing,
%   & (C, \phi) \otimes f &=  \lambda a, s. \varnothing
%   && \text{otherwise}
%   \end{aligned} 
% \end{equation}
% Lifting associators, unitors, and symmetries from \(\mc{C}_0\) in the obvious manner (setting them to \(\lambda a, s. \varnothing\) where necessary), we can verify that \(\ms{SRel}\) is a \textit{premonoidal} category. we can see that it is in fact \textit{monoidal} by nothing that, for any \(f, g\), \(f \ltimes g = f \rtimes g\) since either \(f\) is central, \(g\) is central, or their product is the empty morphism \(\lambda a, s. \varnothing\). Hence, to get a \textit{substructural} monoidal category, we simply need to define:
% \begin{equation}
%   \begin{gathered}
%   \ms{Aff}(\ms{SRel}) = \ms{Rel}(\ms{SRel}) = \{(A, \top) | A \in \ms{Set}\} \neq |\ms{SRel}|
%   \\
%   \begin{aligned}
%     \ms{SRel}^\varnothing((A, \phi), (B, \psi))
%     &= \ms{SRel}((A, \phi), (B, \psi)) \\
%     \ms{SRel}^{\{\ms{a}\}}((A, \phi), (B, \psi)) 
%     &= \{f \in \ms{SRel}((A, \phi), (B, \psi)) \mid \forall a, s. |f(a, s)| \geq 1 \lor \phi \neq \psi \lor \phi = \bot \} \\
%     \ms{SRel}^{\{\ms{r}\}}((A, \phi), (B, \psi)) 
%     &= \{f \in \ms{SRel}((A, \phi), (B, \psi)) \mid \forall a, s. |f(a, s)| \leq 1 \lor \phi \neq \psi \lor \phi = \bot \} \\
%     \implies \ms{SRel}^{\{\ms{a}, \ms{r}\}}((A, \phi), (B, \psi)) 
%     &= \{f \in \ms{SRel}((A, \phi), (B, \psi)) \mid \forall a, s. |f(a, s)| = 1 \lor \phi \neq \psi \lor \phi = \bot \} \\
%   \end{aligned}
%   \end{gathered}
% \end{equation}
% We can verify that the associators, unitors, and symmetry lie in \(\ms{SRel}^{\{\ms{a}, \ms{r}\}}\) and that all other desired axioms are satisfied.

% \TODO{relate separation logic trick to \cite{promonad}, \cite{linear-state-usage}, \cite{mellies-ftrs}}

% \TODO{clean up, and generalize to arbitrary premonoidal categories (w/ initial object? how to deal w/ coproducts?)}

\end{document}
\endinput
