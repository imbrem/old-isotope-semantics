%% Commands for TeXCount
%TC:macro \cite [option:text,text]
%TC:macro \citep [option:text,text]
%TC:macro \citet [option:text,text]
%TC:envir table 0 1
%TC:envir table* 0 1
%TC:envir tabular [ignore] word
%TC:envir displaymath 0 word
%TC:envir math 0 word
%TC:envir comment 0 0

\documentclass[acmsmall,screen,review]{acmart}

\usepackage{syntax}
\renewcommand{\syntleft}{\normalfont\itshape}
\renewcommand{\syntright}{\normalfont\itshape}

\usepackage{prftree}

\newcounter{todos}
\newcommand{\TODO}[1]{{
  \stepcounter{todos}
  \begin{center}\large{\textcolor{red}{\textbf{TODO \arabic{todos}:} #1}}\end{center}
}}

% Math fonts
\newcommand{\mc}[1]{\ensuremath{\mathcal{#1}}}
\newcommand{\mb}[1]{\ensuremath{\mathbf{#1}}}
\newcommand{\ms}[1]{\ensuremath{\mathsf{#1}}}

% Syntax atoms
\newcommand{\lbl}[1]{{`#1}}
\newcommand{\lto}{\Rightarrow}
\newcommand{\ctt}{\ms{tt}}
\newcommand{\cff}{\ms{ff}}

% Syntax
\newcommand{\letexpr}[3]{\ensuremath{\ms{let}\;#1 = #2\;\ms{in}\;#3}}
\newcommand{\letstmt}[3]{\ensuremath{\ms{let}\;#1 = #2; #3}}
\newcommand{\brb}[2]{\ms{br}\;#1\;#2}
\newcommand{\lbrb}[2]{\brb{\lbl{#1}}{#2}}
\newcommand{\ite}[3]{\ms{if}\;#1\;\{#2\}\;\ms{else}\;\{#3\}}
\newcommand{\ewhere}[2]{\ms{then}\;#1\;\ms{where}\;#2}
\newcommand{\where}[2]{#1\;\ms{where}\;#2}
\newcommand{\wbranch}[3]{#1(#2) \lto #3}
\newcommand{\lwbranch}[3]{\wbranch{\lbl{#1}}{#2}{#3}}
\newcommand{\bsplice}[3]{#1(#2)\;\{#3\}}
\newcommand{\lbsplice}[3]{\bsplice{\lbl{#1}}{#2}{#3}}
\newcommand{\csplits}[3]{#1 \mapsto #2;#3}
\newcommand{\cwk}[2]{#1 \mapsto #2}
\newcommand{\lwk}[2]{#1 \rightsquigarrow #2}
\newcommand{\tlin}[2]{\ms{lin}_{#2}(#1)}
\newcommand{\ltlin}[3]{\ms{lin}_{#3}(#1^{#2})}
\newcommand{\thyp}[3]{#1: {#2}^{#3}}
\newcommand{\lhyp}[3]{#1[#2](#3)}
\newcommand{\llhyp}[3]{\lhyp{\lbl{#1}}{#2}{#3}}
\newcommand{\rle}[1]{{\scriptsize\textsf{#1}}}
\newcommand{\taff}{\ms{a}}
\newcommand{\trel}{\ms{r}}
\newcommand{\tint}{\infty}
\newcommand{\hasty}[4]{#1 \vdash_{#2} #3: #4}
\newcommand{\haslb}[3]{#1 \vdash #2 \rhd #3}
\newcommand{\lhaslb}[3]{#1 \vdash #2 \rhd #3}

% Branding
\newcommand{\isotopessa}{\ms{isotope_{SSA}}}

%% Rights management information.  This information is sent to you
%% when you complete the rights form.  These commands have SAMPLE
%% values in them; it is your responsibility as an author to replace
%% the commands and values with those provided to you when you
%% complete the rights form.
\setcopyright{acmcopyright}
\copyrightyear{2018}
\acmYear{2018}
\acmDOI{XXXXXXX.XXXXXXX}

%%
%% These commands are for a JOURNAL article.
% \acmJournal{JACM}
% \acmVolume{37}
% \acmNumber{4}
% \acmArticle{111}
% \acmMonth{8}

%%
%% Submission ID.
%% Use this when submitting an article to a sponsored event. You'll
%% receive a unique submission ID from the organizers
%% of the event, and this ID should be used as the parameter to this command.
%%\acmSubmissionID{123-A56-BU3}

%%
%% The majority of ACM publications use numbered citations and
%% references.  The command \citestyle{authoryear} switches to the
%% "author year" style.
%%
%% If you are preparing content for an event
%% sponsored by ACM SIGGRAPH, you must use the "author year" style of
%% citations and references.
%% Uncommenting
%% the next command will enable that style.
%%\citestyle{acmauthoryear}

\begin{document}

\title{Denotational Semantics for SSA with Weak Memory Operations}

\author{Neel Krishnaswami}
\email{nk480@cl.cam.ac.uk}
\orcid{0000-0003-2838-5865}

\author{Jad Ghalayini}
\email{jeg74@cl.cam.ac.uk}
\orcid{0000-0002-6905-1303}

\begin{abstract}
  TODO THIS
\end{abstract}

%%
%% The code below is generated by the tool at http://dl.acm.org/ccs.cfm.
%% Please copy and paste the code instead of the example below.
%%
\begin{CCSXML}
<ccs2012>
 <concept>
  <concept_id>00000000.0000000.0000000</concept_id>
  <concept_desc>Do Not Use This Code, Generate the Correct Terms for Your Paper</concept_desc>
  <concept_significance>500</concept_significance>
 </concept>
 <concept>
  <concept_id>00000000.00000000.00000000</concept_id>
  <concept_desc>Do Not Use This Code, Generate the Correct Terms for Your Paper</concept_desc>
  <concept_significance>300</concept_significance>
 </concept>
 <concept>
  <concept_id>00000000.00000000.00000000</concept_id>
  <concept_desc>Do Not Use This Code, Generate the Correct Terms for Your Paper</concept_desc>
  <concept_significance>100</concept_significance>
 </concept>
 <concept>
  <concept_id>00000000.00000000.00000000</concept_id>
  <concept_desc>Do Not Use This Code, Generate the Correct Terms for Your Paper</concept_desc>
  <concept_significance>100</concept_significance>
 </concept>
</ccs2012>
\end{CCSXML}

\ccsdesc[500]{Do Not Use This Code~Generate the Correct Terms for Your Paper}
\ccsdesc[300]{Do Not Use This Code~Generate the Correct Terms for Your Paper}
\ccsdesc{Do Not Use This Code~Generate the Correct Terms for Your Paper}
\ccsdesc[100]{Do Not Use This Code~Generate the Correct Terms for Your Paper}

%%
%% Keywords. The author(s) should pick words that accurately describe
%% the work being presented. Separate the keywords with commas.
\keywords{TODO PUT KEYWORDS HERE}

% \received{20 February 2007}
% \received[revised]{12 March 2009}
% \received[accepted]{5 June 2009}

\maketitle

\section{Introduction}

TODO THIS

\section{SSA Syntax}

\TODO{text}

\begin{figure}
  \begin{center}
    \begin{grammar}
      <\(A, B, C\)> ::= \(X\)
      \;|\; \(A \otimes B\)

      <\(a, b, c, e\)> ::= \(x\) 
      \;|\; \(f\;a\)
      \;|\; \((a, b)\) 
      \;|\; \(()\) 
      \;|\; \(\ctt\) 
      \;|\; \(\cff\)
      \;|\; \(\letexpr{x}{a}{e}\)
      \;|\; \(\letexpr{(x, y)}{a}{e}\)
      \;|\; \(\lbsplice{\ell}{x: A}{t}\)
      
      <\(s, t\)> ::= \(\lbrb{\ell}{a}\) 
      \;|\; \(\ite{e}{s}{t}\)
      \;|\; \(\letstmt{x}{a}{t}\)
      \;|\; \(\letstmt{(x, y)}{a}{t}\)
      \;|\; \(\ewhere{t}{L}\)

      <\(L\)> ::= \(\cdot\) \;|\; \(\lwbranch{\ell}{x: A}{t}, L\)

      <\(\Gamma\)> ::= \(\cdot\) \;|\; \(\Gamma, \thyp{x}{A}{q}\)

      <\(\ms{L}\)> ::= \(\cdot\) \;|\; \(\ms{L}, \lbl{\ell}[\Gamma](x: A)\)
    \end{grammar}
  \end{center}
  \caption{Grammar for \isotopessa}
  \Description{Grammar for isotope-SSA}
  \label{fig:ssa-grammar}
\end{figure}

\TODO{late \ms{where}-binding, and other sugar (?)}

\TODO{top-level functions}

\TODO{table of typing judgements}

\TODO{contexts, label-contexts}

\begin{figure}
  \begin{center}        
    \begingroup
    \renewcommand{\arraystretch}{1.5}
    \setlength{\tabcolsep}{2em}
    \begin{tabular}{rl}
        \multicolumn{1}{c}{Judgment} & \multicolumn{1}{c}{Meaning} \\ \hline
        \(\hasty{\Gamma}{p}{x}{A}\) &
        \(a\) is a term of type \(A\) in context \(\Gamma\) with purity \(p \in \{0, 1\}\) \\
        \(\haslb{\Gamma}{t}{\ms{L}}\) &
        \(t\) is a block targeting label-set \(\ms{L}\) in context \(\Gamma\) \\
        \(\lhaslb{\ms{L}}{L}{\ms{K}}\) &
        The labels \(L\) send label-set \(\ms{L}\) to label-set \(\ms{K}\) \\
        \(\csplits{\Gamma}{\Delta}{\Xi}\) &
        The context \(\Gamma\) splits into \(\Delta\) and \(\Xi\) \\
        \(\tlin{A}{q}\) &
        The type \(A\) has linearity \(q\) \\
        \(\ltlin{A}{r}{q}\) &
        \(\tlin{A}{q} \land q \subseteq r\) \\
        \(\tlin{\Gamma}{q}\) &
        The context \(\Gamma\) has linearity \(q\) \\
        \(\cwk{\Gamma}{\Delta}\) &
        \(\Gamma\) is a weakening of \(\Delta\) (i.e. \(\csplits{\Gamma}{\Delta}{\cdot}\)) \\
        \(\lwk{\ms{L}}{\ms{K}}\) &
        The label-set \(\ms{L}\) weakens to the label-set \(\ms{K}\)
    \end{tabular}
    \endgroup
  \end{center}
  \caption{Typing judgements for \isotopessa}
  \Description{Typing judgements for isotope-SSA}
  \label{fig:ssa-judgements}
\end{figure}

\begin{figure}
  \begin{gather*}    
    \prftree[r]{\rle{base-lin}}{q \subseteq \ms{lin}(X)}{\tlin{X}{q}} \qquad
    \prftree[r]{\rle{pair-lin}}{\tlin{A}{q}}{\tlin{B}{q}}{\tlin{A \otimes B}{q}} \qquad
    \prftree[r]{\rle{nil-lin}}{\tlin{\cdot}{q}} \qquad
    \prftree[r]{\rle{cons-lin}}{\ltlin{A}{r}{q}}{\tlin{\Gamma}{q}}
      {\tlin{\Gamma, \thyp{x}{A}{r}}{q}} \\
    \prftree[r]{\rle{split-nil}}{\csplits{\cdot}{\cdot}{\cdot}} \qquad
    \prftree[r]{\rle{split-left}}
      {\csplits{\Gamma}{\Delta}{\Xi}}
      {r \subseteq q}
      {\csplits{\Gamma, \thyp{x}{A}{q}}{\Delta, \thyp{x}{A}{r}}{\Xi}} \qquad
    \prftree[r]{\rle{split-right}}
      {\csplits{\Gamma}{\Delta}{\Xi}}
      {r \subseteq q}
      {\csplits{\Gamma, \thyp{x}{A}{q}}{\Delta}{\Xi, \thyp{x}{A}{r}}} \\
    \prftree[r]{\rle{split-dup}}
      {\csplits{\Gamma}{\Delta}{\Xi}}
      {\ltlin{A}{q}{\trel}}
      {r, s \subseteq q}
      {\csplits{\Gamma, \thyp{x}{A}{q}}{\Delta, \thyp{x}{A}{r}}{\Xi, \thyp{x}{A}{s}}}
      \qquad
    \prftree[r]{\rle{split-drop}}
      {\csplits{\Gamma}{\Delta}{\Xi}}
      {\ltlin{A}{q}{\taff}}
      {\csplits{\Gamma, \thyp{x}{A}{q}}{\Delta}{\Xi}}
      \\
    \prftree[r]{\rle{join-nil}}{\lwk{\cdot}{\cdot}} \qquad
    \prftree[r]{\rle{join-cons}}
      {\lwk{\ms{L}}{\ms{K}}}
      {\lwk{\ms{L}, \llhyp{\ell}{\Gamma}{A}}{\ms{K}, \llhyp{\ell}{\Gamma}{A}}} 
      \qquad
    \prftree[r]{\rle{join-zero}}
      {\lwk{\ms{L}}{\ms{K}}}
      {\lwk{\ms{L}}{\ms{K}, \llhyp{\ell}{\Gamma}{A}}} 
  \end{gather*}
  \caption{Structural rules for \isotopessa}
  \Description{Structural rules for isotope-SSA}
  \label{fig:ssa-structural}
\end{figure}

\begin{figure}
  \begin{gather*}    
    \prftree[r]{\rle{var}}
      {\cwk{\Gamma}{\thyp{x}{A}{q}}}
      {\hasty{\Gamma}{p}{x}{A}} \qquad
    \prftree[r]{\rle{app}}
      {f \in \mc{I}_p(A, B)}
      {\hasty{\Gamma}{1}{a}{A}}
      {\hasty{\Gamma}{p}{f\;a}{B}} \qquad
    \prftree[r]{\rle{pair}}
      {\csplits{\Gamma}{\Delta}{\Xi}}
      {\hasty{\Delta}{1}{a}{A}}
      {\hasty{\Xi}{1}{b}{B}}
      {\hasty{\Gamma}{1}{(a, b)}{A \otimes B}} \\
    \prftree[r]{\rle{unit}}
      {\cwk{\Gamma}{\cdot}}
      {\hasty{\Gamma}{1}{()}{\mb{1}}} \qquad
    \prftree[r]{\rle{true}}
      {\cwk{\Gamma}{\cdot}}
      {\hasty{\Gamma}{1}{\ctt}{\mb{2}}} \qquad
    \prftree[r]{\rle{false}}
      {\cwk{\Gamma}{\cdot}}
      {\hasty{\Gamma}{1}{\cff}{\mb{2}}} \\
    \prftree[r]{\rle{let}}
      {\csplits{\Gamma}{\Delta}{\Xi}}
      {\hasty{\Delta, \thyp{x}{A}{}}{p}{e}{B}}
      {\hasty{\Xi}{1}{a}{A}}
      {\hasty{\Gamma}{p}{\letexpr{x}{a}{e}}{B}} \qquad
    \prftree[r]{\rle{blk}}
      {\haslb{\Gamma}{t}{\llhyp{\ell}{\cdot}{A}}}
      {\hasty{\Gamma}{0}{\lbsplice{\ell}{A}{t}}{A}} \\
    \prftree[r]{\rle{let2}}
      {\csplits{\Gamma}{\Delta}{\Xi}}
      {\hasty{\Delta, \thyp{x}{A}{}, \thyp{y}{B}{}}{p}{e}{C}}
      {\hasty{\Xi}{1}{a}{A \otimes B}}
      {\hasty{\Gamma}{p}{\letexpr{(x, y)}{a}{e}}{C}}
  \end{gather*}
  \caption{Typing rules for \isotopessa terms}
  \Description{Typing rules for isotope-SSA terms}
  \label{fig:ssa-term-typing}
\end{figure}


\begin{figure}
  \begin{gather*}    
    \prftree[r]{\rle{br}}
      {\csplits{\Gamma}{\Delta}{\Xi}}
      {\lwk{\llhyp{\ell}{\Delta}{A}}{\ms{L}}}
      %TODO: consider allowing binding here, or should it be purely for lets?
      {\hasty{\Xi}{1}{a}{A}}
      {\haslb{\Gamma}{\lbrb{\ell}{a}}{\ms{L}}} \qquad
    \prftree[r]{\rle{ite}}
      {\csplits{\Gamma}{\Delta}{\Xi}}
      {\hasty{\Delta}{1}{e}{\mb{2}}}
      {\haslb{\Xi}{s}{\ms{L}}}
      {\haslb{\Xi}{t}{\ms{L}}}
      {\haslb{\Gamma}{\ite{e}{s}{t}}{\ms{L}}} \\
    \prftree[r]{\rle{let-blk}}
      {\csplits{\Gamma}{\Delta}{\Xi}}
      {\haslb{\Delta, \thyp{x}{A}{}}{t}{\ms{L}}}
      {\hasty{\Xi}{p}{a}{A}}
      {\haslb{\Gamma}{\letstmt{x}{a}{t}}{\ms{L}}} \\
    \prftree[r]{\rle{let2-blk}}
      {\csplits{\Gamma}{\Delta}{\Xi}}
      {\haslb{\Delta, \thyp{x}{A}{}, \thyp{y}{B}{}}{t}{\ms{L}}}
      {\hasty{\Xi}{p}{a}{A \otimes B}}
      {\haslb{\Gamma}{\letstmt{(x, y)}{a}{t}}{\ms{L}}} \\
    \prftree[r]{\rle{where}}
      {\haslb{\Gamma}{t}{\ms{L}}}
      {\lhaslb{\ms{L}}{L}{\ms{K}}}
      {\haslb{\Gamma}{\ewhere{t}{L}}{\ms{K}}} \qquad
    \prftree[r]{\rle{nil-br}}
      {\lwk{\ms{L}}{\ms{K}}}
      {\lhaslb{\ms{L}}{\cdot}{\ms{K}}} \qquad
    \prftree[r]{\rle{cons-br}}
      {\lhaslb{\ms{L}}{L}{\ms{K}, \llhyp{\ell}{\Gamma}{A}}}
      {\haslb{\Gamma, \thyp{x}{A}{}}{t}{\ms{L}}}
      {\lhaslb{\ms{L}}{L, \lwbranch{\ell}{x: A}{t}}{\ms{K}}}
  \end{gather*}
  \caption{Typing rules for \isotopessa blocks}
  \Description{Typing rules for isotope-SSA blocks}
  \label{fig:ssa-block-typing}
\end{figure}

\section{SSA Semantics}

\TODO{premonoidal categories}

\TODO{Elgot structure}

\begin{figure}
  \begin{center}
    \TODO{this}
  \end{center}
  \caption{Semantics for \isotopessa}
  \Description{Semantics for isotope-SSA}
  \label{fig:ssa-semantics}
\end{figure}

\TODO{metatheory: substitution, rewriting, etc}
\TODO{\(\implies\) E-graph optimization}

\section{Basic Models}

\TODO{Ye Olde Trace Monad}

\TODO{Ye Olde State Transformer}

\TODO{Ye Olde Nondeterministic Trace Monad}

\section{Weak Memory}

\TODO{Pomsets}

\TODO{SC Monad}

\TODO{Building The Weak Memory Monad}

\TODO{Karoubi envelope as premonoidal demonstration?}

\section{Implementation}

\TODO{this}

\section{Related Work}

\TODO{\cite{promonad}}

\TODO{\cite{linear-state-usage}}

\TODO{\cite{ssa-is-fun}}

\TODO{\cite{sparky}}

\bibliographystyle{ACM-Reference-Format}
\bibliography{references}

\end{document}
\endinput
