%% Commands for TeXCount
%TC:macro \cite [option:text,text]
%TC:macro \citep [option:text,text]
%TC:macro \citet [option:text,text]
%TC:envir table 0 1
%TC:envir table* 0 1
%TC:envir tabular [ignore] word
%TC:envir displaymath 0 word
%TC:envir math 0 word
%TC:envir comment 0 0

\documentclass[acmsmall,screen,review]{acmart}

\usepackage{syntax}
\renewcommand{\syntleft}{\normalfont\itshape}
\renewcommand{\syntright}{\normalfont\itshape}

\usepackage{prftree}

\usepackage{listings}
\usepackage{xcolor}
\usepackage{subcaption}
\usepackage{fancyvrb}

\definecolor{codegreen}{rgb}{0,0.6,0}
\definecolor{codegray}{rgb}{0.5,0.5,0.5}
\definecolor{codepurple}{rgb}{0.58,0,0.82}
\definecolor{backcolour}{rgb}{0.95,0.95,0.92}

\lstdefinestyle{mystyle}{
%    backgroundcolor=\color{backcolour},   
    commentstyle=\color{codegreen},
    keywordstyle=\color{magenta},
    numberstyle=\tiny\color{codegray},
    stringstyle=\color{codepurple},
    basicstyle=\ttfamily\footnotesize,
    breakatwhitespace=false,         
    breaklines=true,                 
    captionpos=b,                    
    keepspaces=true,                 
    numbers=left,                    
    numbersep=5pt,                  
    showspaces=false,                
    showstringspaces=false,
    showtabs=false,                  
    tabsize=2
}

\lstset{style=mystyle}

\newcounter{todos}
\newcommand{\TODO}[1]{{
  \stepcounter{todos}
  \begin{center}\large{\textcolor{red}{\textbf{TODO \arabic{todos}:} #1}}\end{center}
}}
\newcommand{\sorry}{\textcolor{red}{\textbf{sorry}}}

\newcommand{\todo}[1]{\stepcounter{todos} \textcolor{red}{TODO \arabic{todos}: #1}}

% Math fonts
\newcommand{\mc}[1]{\ensuremath{\mathcal{#1}}}
\newcommand{\mb}[1]{\ensuremath{\mathbf{#1}}}
\newcommand{\ms}[1]{\ensuremath{\mathsf{#1}}}

% Math
\newcommand{\nats}{\mathbb{N}}

% Syntax atoms
\newcommand{\lbl}[1]{{`#1}}
\newcommand{\lto}{\Rightarrow}
\newcommand{\ctt}{\ms{tt}}
\newcommand{\cff}{\ms{ff}}

% Syntax
\newcommand{\letexpr}[3]{\ensuremath{\ms{let}\;#1 = #2\;\ms{in}\;#3}}
\newcommand{\letstmt}[3]{\ensuremath{\ms{let}\;#1 = #2; #3}}
\newcommand{\brb}[2]{\ms{br}\;#1\;#2}
\newcommand{\lbrb}[2]{\brb{\lbl{#1}}{#2}}
\newcommand{\ite}[3]{\ms{if}\;#1\;\{#2\}\;\ms{else}\;\{#3\}}
\newcommand{\ewhere}[2]{\ms{then}\;#1\;\ms{where}\;#2}
\newcommand{\where}[2]{#1\;\ms{where}\;#2}
\newcommand{\wbranch}[3]{#1(#2) \lto #3}
\newcommand{\lwbranch}[3]{\wbranch{\lbl{#1}}{#2}{#3}}
\newcommand{\bsplice}[3]{#1(#2)\;\{#3\}}
%\newcommand{\lbsplice}[3]{\bsplice{\lbl{#1}}{#2}{#3}}
\newcommand{\csplits}[3]{#1 \mapsto #2;#3}
\newcommand{\cwk}[2]{#1 \mapsto #2}
\newcommand{\lwk}[2]{#1 \rightsquigarrow #2}
\newcommand{\tlin}[2]{#2 \subseteq \ms{lin}(#1)}
\newcommand{\ltlin}[3]{#3 \subseteq \ms{lin}(#1) \cap #2}
\newcommand{\thyp}[3]{#1: {#2}^{#3}}
\newcommand{\lhyp}[3]{#1[#2](#3)}
\newcommand{\llhyp}[3]{\lhyp{\lbl{#1}}{#2}{#3}}
\newcommand{\rle}[1]{{\scriptsize\textsf{#1}}}
\newcommand{\taff}{{\{\ms{a}\}}}
\newcommand{\trel}{{\{\ms{r}\}}}
\newcommand{\tint}{{\{\ms{a}, \ms{r}\}}}
\newcommand{\hasty}[5]{#1 \vdash_{#2} #3: {#4}^{#5}}
\newcommand{\haslb}[3]{#1 \vdash #2 \rhd #3}
\newcommand{\lhaslb}[3]{#1 \vdash #2 \rhd #3}
\newcommand{\issubst}[3]{#1: #2 \mapsto #3}
\newcommand{\lbsubst}[3]{#1: #2 \rightsquigarrow #3}
\newcommand{\exprletsubst}[2]{{#1};{#2}}
\newcommand{\stmtletsubst}[2]{{#1};{#2}}
\newcommand{\mhole}[1]{{#1}^?}
\newcommand{\lhole}[1]{?#1}
\newcommand{\mhasty}[6]{#1;#2 \vdash_{#3} #4: {#5}^{#6}}
\newcommand{\mhaslb}[4]{#1;#2 \vdash #3 \rhd #4}
\newcommand{\mlhaslb}[4]{#1;#2 \vdash #3 \rhd #4}
\newcommand{\tyhole}[5]{#1: #2 \mapsto_{#3} {#4}^{#5}}
\newcommand{\blkhole}[3]{#1: #2 \mapsto #3}
\newcommand{\cfghole}[3]{#1: #2 \mapsto #3}
\newcommand{\substctx}[2]{{#1}^{#2}}
\newcommand{\substlbs}[2]{{#1}^{#2}}
\newcommand{\restrictsubst}[2]{{#1}_{#2}}
\newcommand{\subsubst}[2]{#1 \subseteq #2}
\newcommand{\isrw}[3]{#1: #2 \mapsto #3}
\newcommand{\mbind}{\mathbin{{>}\hspace{-0.1em}{>}\hspace{-0.1em}{=}}}
% \newcommand{\strictlbsubst}[3]{#1: #2 \rightsquigarrow_= #3}

% Denotational semantics
\newcommand{\dnt}[1]{\llbracket{#1}\rrbracket}
\newcommand{\ednt}[1]{\left\llbracket{#1}\right\rrbracket}
\newcommand{\upg}[2]{{#1}^{\uparrow #2}}

% Weak memory
\newcommand{\bufloc}[1]{\overline{#1}}

% Branding
\newcommand{\isotopessa}{\ms{isotope_{SSA}}}

%% Rights management information.  This information is sent to you
%% when you complete the rights form.  These commands have SAMPLE
%% values in them; it is your responsibility as an author to replace
%% the commands and values with those provided to you when you
%% complete the rights form.
\setcopyright{acmcopyright}
\copyrightyear{2023}
\acmYear{2023}
\acmDOI{XXXXXXX.XXXXXXX}

%%
%% These commands are for a JOURNAL article.
% \acmJournal{JACM}
% \acmVolume{37}
% \acmNumber{4}
% \acmArticle{111}
% \acmMonth{8}

%%
%% Submission ID.
%% Use this when submitting an article to a sponsored event. You'll
%% receive a unique submission ID from the organizers
%% of the event, and this ID should be used as the parameter to this command.
%%\acmSubmissionID{123-A56-BU3}

\begin{document}

\title{Isotope With Resources}

\author{Jad Ghalayini}
\email{jeg74@cl.cam.ac.uk}
\orcid{0000-0002-6905-1303}

\author{Neel Krishnaswami}
\email{nk480@cl.cam.ac.uk}
\orcid{0000-0003-2838-5865}

\maketitle

\TODO{go cite Iris}

A \textbf{resource algebra} \(R\) is a partial commutative ordered (additive)
monoid equipped with a distinguished element \(1 \in R\) (note that the
identity here is represented as \(0\)) such that addition is nondecreasing in
both arguments. \TODO{does it need to be nondecreasing in both arguments?} We
define the following important resource algebras:
\begin{itemize}
    \item The \textbf{unit resource algebra} \(\mb{1}\) simply the unit monoid
    \item The \textbf{linear resource algebra} \(\mb{2_L}\) is the set \(\{0,
    1\}\) equipped with the discrete order, where \(1 + 1\) is undefined
    \item The \textbf{affine resource algebra} \(\mb{2_A}\) is the set \(\{0,
    1\}\) equipped with the usual order, where \(1 + 1\) is undefined
    \item The \textbf{relevant resource algebra} \(\mb{2_R}\) is the set \(\{0,
    1\}\) equipped with the discrete order, where \(1 + 1 = 1\)
    \item The \textbf{pure} or \textbf{intuitionistic resource algebra}
    \(\mb{2_P}\) is the set \(\{0, 1\}\) equipped with the usual order, where
    \(1 + 1\) is undefined
\end{itemize}
A resource algebra is \textbf{discrete} if it is equipped with the discrete
order. A resource algebra is \textbf{usable} if \(1 \neq 0\). A resource algebra
isomorphism is a monoid isomorphism that is also an order isomorphism. We define
the \textbf{product} of two resource algebras \(L \times R\) to consist of pairs
\((l, r)\) where \((l, r) \leq (l', r') \iff l \leq l' \land r \leq r'\) and
\((l, r) + (l', r') = (l + l', r + r')\); in particular, we note that 
\begin{itemize}
    \item The product of discrete resource algebras is discrete
    \item \((L \times M) \times R \simeq L \times (M \times R)\) and \(R \times
    \mb{1} \simeq \mb{1} \times R \simeq R\)
\end{itemize}
Given an assignment \(R_A\) of a resource algebra to every base type \(A \in
\mc{T}\), we can assign the algebra \(R_{A \otimes B} = R_A \times R_B\) to the
tensor product \(A \otimes B\).

\TODO{resource algebra based strict splits, weakening \(\implies\) splits}

\TODO{isotope with resources}

\TODO{substitution + weakening theorems}

\TODO{resource algebra semantics}

\TODO{resource algebra semantic substitution + weakening}

\end{document}